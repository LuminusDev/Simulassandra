%% Beamer theme to be used by LIP6 teams.
%% This is a skeleton file demonstrating
%% the use of the theme Frederiksberg version 2.2
%% with the UPMC visual guidelines applied.
%%
%% Version 1.1
%% May 19, 2012
%% by S�bastien Heymann <sebastien.heymann@lip6.fr>, http://sebastien.pro/
%% from LIP6 ComplexNetworks
%%
%% REQUIREMENTS:
%% - compile with pdflatex (tested on Windows7 with Texmaker 3.2)
%% - install http://matdat.life.ku.dk/LaTeX/Frederiksberg
%%
%% WARNINGS:
%% - load this file in ISO-8859-1 and keep using this encoding

%%*************************************************************************
%% Legal Notice:
%% Copyright (c) 2012 S�bastien Heymann <sebastien.heymann@lip6.fr>
%% 
%% Permission is hereby granted, free of charge, to any person obtaining a copy of 
%% this software and associated documentation files (the "Software"), to deal in the 
%% Software without restriction, including without limitation the rights to use, copy, 
%% modify, merge, publish, distribute, sublicense, and/or sell copies of the Software, 
%% and to permit persons to whom the Software is furnished to do so, subject to the 
%% following conditions:
%% 
%% The above copyright notice and this permission notice shall be included in all 
%% copies or substantial portions of the Software.
%% 
%% THE SOFTWARE IS PROVIDED "AS IS", WITHOUT WARRANTY OF ANY KIND, EXPRESS OR IMPLIED, 
%% INCLUDING BUT NOT LIMITED TO THE WARRANTIES OF MERCHANTABILITY, FITNESS FOR A 
%% PARTICULAR PURPOSE AND NONINFRINGEMENT. IN NO EVENT SHALL THE AUTHORS OR COPYRIGHT 
%% HOLDERS BE LIABLE FOR ANY CLAIM, DAMAGES OR OTHER LIABILITY, WHETHER IN AN ACTION 
%% OF CONTRACT, TORT OR OTHERWISE, ARISING FROM, OUT OF OR IN CONNECTION WITH THE 
%% SOFTWARE OR THE USE OR OTHER DEALINGS IN THE SOFTWARE.
%%*************************************************************************



% *** COMPILATION ***
% Use pdflatex.
%
% Use one of these document classes while working on the presentation:
% Commment these lines below once the presentation
% is ready to export in various formats:
\documentclass{beamer}
%\documentclass[handout]{beamer} % no overlay
%
% Once done, export the presentation, handout and printable handout files using:
% beamer.tex ; handout.tex ; print.tex



% *** PACKAGES ***
%
\usepackage[utf8]{inputenc}
\usepackage[cyr]{aeguill} % support French guillemets
\usepackage[francais]{babel} % support French language
\usepackage{hyperref} % support PDF bookmarks
\usepackage{url}
%\usepackage{caption}
%\usepackage{media9}
%\addmediapath{./videos/}



% *** GRAPHICS RELATED PACKAGES ***
%
% declare the path(s) where your graphic files are
\graphicspath{{img/}}
% and their extensions so you won't have to specify these with
% every instance of \includegraphics
\DeclareGraphicsExtensions{.pdf,.png,.jpg}

% no figure caption label
%\captionsetup[figure]{labelformat=empty}


% *** ABSOLUTE POSITIONNING PACKAGE ***
%
\usepackage[absolute,overlay]{textpos}



% *** COLORS ***
%
% Official colors defined in the visual guidelines on
% http://www.upmc.fr/fr/espace_des_personnels/pour_votre_laboratoire/communiquer2/logos_et_chartes.html
\usepackage{color}
\definecolor{UPMCEngagementBlueA}   {RGB}{140,184,198}
\definecolor{UPMCEngagementBlueB}   {RGB}{92,127,146}
\definecolor{UPMCEngagementBlueC}   {RGB}{75,146,219}
\definecolor{UPMCEngagementBlueD}   {RGB}{33,49,77}
\definecolor{UPMCEngagementYellowA} {RGB}{254,209,0}
\definecolor{UPMCEngagementYellowB} {RGB}{198,172,0}
\definecolor{UPMCEngagementGreen}   {RGB}{64,74,41}
\definecolor{UPMCCorporateGreen}    {RGB}{182,191,0}
\definecolor{UPMCExcellenceOrangeA} {RGB}{224,82,6}
\definecolor{UPMCExcellenceOrangeB} {RGB}{225,160,47}
\definecolor{UPMCCorporateMarron}   {RGB}{145,120,91}
\definecolor{UPMCInnovationCoolGray}{RGB}{97,99,101}

\colorlet{BgTransition}{UPMCInnovationCoolGray}



% *** THEME ***
% Based on the theme Frederiksberg version 2.2
% March 9, 2012
% by Morten Larsen <ml@life.ku.dk>
% See the user guide at http://matdat.life.ku.dk/LaTeX/Frederiksberg
% for more info about the options and additional features like sidebar.
%
\usetheme[not@ku={}, wide, TPomitframeno, FTalign=center, greyfoot, seriftitles, fnolabel=, basecolour=UPMCEngagementBlueB,  topbarcolour=UPMCEngagementBlueA]{Frederiksberg}
\setbeamercolor{block title}{fg=white}
%\setbeamercolor{block body}{bg=white}
\setbeamercolor{block title example}{bg=UPMCCorporateGreen}
\setbeamercolor{block title alerted}{bg=UPMCExcellenceOrangeA}
\setbeamercolor{alerted text}{fg=UPMCExcellenceOrangeA}

% Removes header and footer in handout mode
\mode<handout>{
  \setbeamertemplate{headline}{}
  \setbeamertemplate{footline}[page number]{}
}



% *** PDF SETUP ***
%
\hypersetup{pagebackref,  
%pdfpagemode=FullScreen, % open PDF in fullscreen
pdfnonfullscreenpagemode=UseThumbs, % show thumbnails on exiting full-screen mode; other option: UseOutlines
colorlinks=false}



% *** META ***
%
\title[Projet de Programmation]{Simulation d'algorithmes \\d'équilibrage de charge dans un \\environnement distribué}
\author{Kevin Barreau \and Guillaume Marques \and Corentin Salingue}
\institute{Université de Bordeaux}
\date{15 avril 2015}
\logo{
    \hspace{9cm}
    \includegraphics[scale=0.13]{ubordeaux}
}



% *** SPECIAL COMMANDS ***
%
\newcommand{\SlideTransition}[2][]{
  % Change background color
  \mode<all> {
    \setbeamercolor{background canvas}{bg=BgTransition}
  }

  % Use [plain] mode to get a slide without header nor footer.
  % Use <handout:0> to hide the slide on the handout.
  \begin{frame}<handout:0>[plain]
%  \only<1| handout:0>{  
%  \begin{textblock*}{\paperwidth}(-5cm,0pt)
%    \raggedleft \includegraphics[scale=0.3]{bg}
%\end{textblock*}
%  }
  \vspace{3.5cm}
  \begin{flushright}
    {\Huge \rmfamily \slshape \textcolor{white}{#2}} \linebreak
    #1
    \end{flushright}
  \end{frame}

  % Reset background color
  \mode<all> {
    \setbeamercolor{background canvas}{bg=white}
  }
}



% *** BEGIN DOCUMENT ***
%
\begin{document}
\mode* % ignore text outside frames, useful to write notes



% *** FRAME #1 ***
%
% use:
%\frame[plain]{\titlepage}
% or:
\begin{frame}
\titlepage
(additional stuff here)
\end{frame}


\section{Introduction}

\begin{frame}{Introduction}

\begin{alertblock}{Alert}
    Parler de ce qu'on doit faire sans s'attacher à Cassandra (Besoins fonctionnels ?)
\end{alertblock}
\end{frame}


\section{Cassandra}

\begin{frame}{Cassandra}

\begin{alertblock}{Alert}
    Se positionner par rapport à Cassandra et pourquoi
\end{alertblock}
\end{frame}


\section{Fonctionnement}

\begin{frame}{Fonctionnement}

\begin{alertblock}{Alert}
    Expliquer comment fonctionnent le client et cassandra avec des schemas, capture d'écran et tout...
\end{alertblock}
\end{frame}


\section{Architecture}

\begin{frame}{Architecture}

\begin{alertblock}{Alert}
    Architecture du client et de cassandra
\end{alertblock}
\end{frame}


\section{Simulateur}

\begin{frame}{Simulateur}

\begin{alertblock}{Alert}
    Expliquer pourquoi simulateur + archi + fonctionnement
\end{alertblock}
\end{frame}


\section{Points techniques}

\begin{frame}{Points techniques}

\begin{alertblock}{Alert}
    Si on veut rentrer dans des détails d'implémentation...
\end{alertblock}
\end{frame}


\section{Tests}

\begin{frame}{Tests}

\begin{alertblock}{Alert}
    Présentation des tests
\end{alertblock}
\end{frame}


\section{Améliorations}

\begin{frame}{Améliorations}

\begin{alertblock}{Alert}
    En guise de conclusion, qu'est ce qu'il reste à faire ou à améliorer
\end{alertblock}
\end{frame}




\section{Blocks}

\begin{frame}{Blocks}

\begin{block}{Standard}
This is a standard block.
\end{block}

\begin{exampleblock}{Example}
This is an example.
\end{exampleblock}

\begin{alertblock}{Alert}
This is important.
\end{alertblock}
\end{frame}


\section{Example}

\begin{frame}{Example}

\begin{block}{Complex networks}
\begin{itemize}
\item Sociology : social networks, call networks
\item Informatics : Internet, Web, peer-to-peer networks
\item Biology, linguistics, etc.
\end{itemize}
\end{block}

\vspace{1.6cm}

\begin{textblock*}{\paperwidth}(0pt,6cm)
    \raggedleft \includegraphics[width=0.5\textwidth]{static-graph1}
\end{textblock*}
\end{frame}


\begin{frame}{Example}
\begin{columns}
\begin{column}[c]{3.3cm}
\begin{block}{Evolving network}
Nodes and links appearing over time.
\end{block}
\end{column}

\begin{column}[c]{7cm}
\includegraphics[scale=0.4]{dynamic-graph-draft} 
%\linebreak
%{\tiny This is a caption.}
\end{column}
\end{columns}
\end{frame}



\mode<all>{ 
\SlideTransition{Questions?} 
}



\mode<all>{ 
\SlideTransition[Title \linebreak
\href{mailto:first.lastname@lip6.fr}{<first.lastname@lip6.fr>}
]{Thank You!} 
}


% that's all folks
\end{document}
