\documentclass[12pt]{article}
\usepackage[utf8]{inputenc}
\usepackage[french]{babel}
\usepackage[tikz]{bclogo}
\usepackage{geometry}
\usepackage{array}
\usepackage{graphics}
\usepackage{graphicx}
\usepackage{pgfgantt}
\usepackage{url}
\bibliographystyle{alpha}
\usepackage[counterclockwise]{rotating}
\geometry{hmargin=2.5cm,vmargin=1.5cm}

\setlength{\parskip}{1ex plus 2ex minus 1ex}
\newcolumntype{M}[1]{
    >{\raggedright}m{#1}
}

\title{
 \begin{minipage}\linewidth
        \centering
        Simulation d'algorithmes d'équilibrage de charge dans un environnement distribué 
        \vskip3pt
        \large Architecture
    \end{minipage}
 }
 
\bibliographystyle{alpha}
\author{Kevin Barreau \and Guillaume Marques \and Corentin Salingue}

\begin{document}

\maketitle

\newpage

\renewcommand{\contentsname}{Sommaire} 
\tableofcontents

\newpage

\section{Architecture de Cassandra}

\paragraph{} Cassandra est une base de données distribuées, écrite en langage Java. Dans sa version 2.1.2, elle est composée de 963 fichiers, répartis dans 62 dossiers, pour un total de 126 502 lignes de codes et 36 287 lignes de commentaires.

\paragraph{} Cassandra est un projet riche et complet. Nous ne nous intéresserons qu'aux parties de son architecture sur lesquelles nous allons travailler.

\subsection{Staged event-driven architecture (SEDA)}

\paragraph{} Cassandra est basée sur une architecture de type Staged Event Driven Architecture (SEDA). Cela permet de séparer des tâches dans différents emplacements, appelés \textit{stages}, qui sont connectés par un service de messages. Chaque stage possède une file d'attente pour les messages (un message correspondant à une tâche à traiter), ainsi qu'un ensemble de threads pour traiter les tâches. \ref{fig:stages}

\begin{figure}[p]
	\centering
		% Graphic for TeX using PGF
% Title: C:\Users\Kéké\Pictures\stages.dia
% Creator: Dia v0.97.2
% CreationDate: Sat Feb 14 15:53:37 2015
% For: Kéké
% \usepackage{tikz}
% The following commands are not supported in PSTricks at present
% We define them conditionally, so when they are implemented,
% this pgf file will use them.
\ifx\du\undefined
  \newlength{\du}
\fi
\setlength{\du}{15\unitlength}
\begin{tikzpicture}
\pgftransformxscale{1.000000}
\pgftransformyscale{-1.000000}
\definecolor{dialinecolor}{rgb}{0.000000, 0.000000, 0.000000}
\pgfsetstrokecolor{dialinecolor}
\definecolor{dialinecolor}{rgb}{1.000000, 1.000000, 1.000000}
\pgfsetfillcolor{dialinecolor}
\pgfsetlinewidth{0.100000\du}
\pgfsetdash{}{0pt}
\pgfsetdash{}{0pt}
\pgfsetbuttcap
\pgfsetmiterjoin
\pgfsetlinewidth{0.100000\du}
\pgfsetbuttcap
\pgfsetmiterjoin
\pgfsetdash{}{0pt}
\definecolor{dialinecolor}{rgb}{0.960784, 0.960784, 0.960784}
\pgfsetfillcolor{dialinecolor}
\fill (4.832258\du,6.556250\du)--(4.832258\du,17.476250\du)--(15.400000\du,17.476250\du)--(15.400000\du,6.556250\du)--cycle;
\definecolor{dialinecolor}{rgb}{0.000000, 0.000000, 0.000000}
\pgfsetstrokecolor{dialinecolor}
\draw (4.832258\du,6.556250\du)--(4.832258\du,17.476250\du)--(15.400000\du,17.476250\du)--(15.400000\du,6.556250\du)--cycle;
\pgfsetbuttcap
\pgfsetmiterjoin
\pgfsetdash{}{0pt}
\definecolor{dialinecolor}{rgb}{0.000000, 0.000000, 0.000000}
\pgfsetstrokecolor{dialinecolor}
\draw (4.832258\du,6.556250\du)--(4.832258\du,17.476250\du)--(15.400000\du,17.476250\du)--(15.400000\du,6.556250\du)--cycle;
\pgfsetlinewidth{0.000000\du}
\pgfsetdash{}{0pt}
\pgfsetdash{}{0pt}
\pgfsetbuttcap
\pgfsetmiterjoin
\pgfsetlinewidth{0.000000\du}
\pgfsetbuttcap
\pgfsetmiterjoin
\pgfsetdash{}{0pt}
\definecolor{dialinecolor}{rgb}{0.247059, 0.317647, 0.709804}
\pgfsetfillcolor{dialinecolor}
\fill (1.689994\du,8.035212\du)--(1.689994\du,8.939378\du)--(2.564994\du,8.939378\du)--(2.564994\du,8.035212\du)--cycle;
\definecolor{dialinecolor}{rgb}{0.247059, 0.317647, 0.709804}
\pgfsetstrokecolor{dialinecolor}
\draw (1.689994\du,8.035212\du)--(1.689994\du,8.939378\du)--(2.564994\du,8.939378\du)--(2.564994\du,8.035212\du)--cycle;
\pgfsetbuttcap
\pgfsetmiterjoin
\pgfsetdash{}{0pt}
\definecolor{dialinecolor}{rgb}{0.247059, 0.317647, 0.709804}
\pgfsetstrokecolor{dialinecolor}
\draw (1.689994\du,8.035212\du)--(1.689994\du,8.939378\du)--(2.564994\du,8.939378\du)--(2.564994\du,8.035212\du)--cycle;
\pgfsetlinewidth{0.000000\du}
\pgfsetdash{}{0pt}
\pgfsetdash{}{0pt}
\pgfsetbuttcap
\pgfsetmiterjoin
\pgfsetlinewidth{0.000000\du}
\pgfsetbuttcap
\pgfsetmiterjoin
\pgfsetdash{}{0pt}
\definecolor{dialinecolor}{rgb}{0.247059, 0.317647, 0.709804}
\pgfsetfillcolor{dialinecolor}
\fill (2.844574\du,8.035212\du)--(2.844574\du,8.939379\du)--(3.719574\du,8.939379\du)--(3.719574\du,8.035212\du)--cycle;
\definecolor{dialinecolor}{rgb}{0.247059, 0.317647, 0.709804}
\pgfsetstrokecolor{dialinecolor}
\draw (2.844574\du,8.035212\du)--(2.844574\du,8.939379\du)--(3.719574\du,8.939379\du)--(3.719574\du,8.035212\du)--cycle;
\pgfsetbuttcap
\pgfsetmiterjoin
\pgfsetdash{}{0pt}
\definecolor{dialinecolor}{rgb}{0.247059, 0.317647, 0.709804}
\pgfsetstrokecolor{dialinecolor}
\draw (2.844574\du,8.035212\du)--(2.844574\du,8.939379\du)--(3.719574\du,8.939379\du)--(3.719574\du,8.035212\du)--cycle;
\pgfsetlinewidth{0.000000\du}
\pgfsetdash{}{0pt}
\pgfsetdash{}{0pt}
\pgfsetbuttcap
\pgfsetmiterjoin
\pgfsetlinewidth{0.000000\du}
\pgfsetbuttcap
\pgfsetmiterjoin
\pgfsetdash{}{0pt}
\definecolor{dialinecolor}{rgb}{0.247059, 0.317647, 0.709804}
\pgfsetfillcolor{dialinecolor}
\fill (3.994323\du,8.026865\du)--(3.994323\du,8.931031\du)--(4.869323\du,8.931031\du)--(4.869323\du,8.026865\du)--cycle;
\definecolor{dialinecolor}{rgb}{0.247059, 0.317647, 0.709804}
\pgfsetstrokecolor{dialinecolor}
\draw (3.994323\du,8.026865\du)--(3.994323\du,8.931031\du)--(4.869323\du,8.931031\du)--(4.869323\du,8.026865\du)--cycle;
\pgfsetbuttcap
\pgfsetmiterjoin
\pgfsetdash{}{0pt}
\definecolor{dialinecolor}{rgb}{0.247059, 0.317647, 0.709804}
\pgfsetstrokecolor{dialinecolor}
\draw (3.994323\du,8.026865\du)--(3.994323\du,8.931031\du)--(4.869323\du,8.931031\du)--(4.869323\du,8.026865\du)--cycle;
\pgfsetlinewidth{0.000000\du}
\pgfsetdash{}{0pt}
\pgfsetdash{}{0pt}
\pgfsetbuttcap
\pgfsetmiterjoin
\pgfsetlinewidth{0.000000\du}
\pgfsetbuttcap
\pgfsetmiterjoin
\pgfsetdash{}{0pt}
\definecolor{dialinecolor}{rgb}{0.247059, 0.317647, 0.709804}
\pgfsetfillcolor{dialinecolor}
\fill (5.123068\du,8.033115\du)--(5.123068\du,8.937281\du)--(5.998068\du,8.937281\du)--(5.998068\du,8.033115\du)--cycle;
\definecolor{dialinecolor}{rgb}{0.247059, 0.317647, 0.709804}
\pgfsetstrokecolor{dialinecolor}
\draw (5.123068\du,8.033115\du)--(5.123068\du,8.937281\du)--(5.998068\du,8.937281\du)--(5.998068\du,8.033115\du)--cycle;
\pgfsetbuttcap
\pgfsetmiterjoin
\pgfsetdash{}{0pt}
\definecolor{dialinecolor}{rgb}{0.247059, 0.317647, 0.709804}
\pgfsetstrokecolor{dialinecolor}
\draw (5.123068\du,8.033115\du)--(5.123068\du,8.937281\du)--(5.998068\du,8.937281\du)--(5.998068\du,8.033115\du)--cycle;
\pgfsetlinewidth{0.100000\du}
\pgfsetdash{}{0pt}
\pgfsetdash{}{0pt}
\pgfsetbuttcap
\pgfsetmiterjoin
\pgfsetlinewidth{0.100000\du}
\pgfsetbuttcap
\pgfsetmiterjoin
\pgfsetdash{}{0pt}
\definecolor{dialinecolor}{rgb}{0.247059, 0.317647, 0.709804}
\pgfsetfillcolor{dialinecolor}
\fill (6.388782\du,8.069167\du)--(7.040160\du,8.069167\du)--(7.365849\du,8.484369\du)--(7.040160\du,8.899571\du)--(6.388782\du,8.899571\du)--cycle;
\definecolor{dialinecolor}{rgb}{0.247059, 0.317647, 0.709804}
\pgfsetstrokecolor{dialinecolor}
\draw (6.388782\du,8.069167\du)--(7.040160\du,8.069167\du)--(7.365849\du,8.484369\du)--(7.040160\du,8.899571\du)--(6.388782\du,8.899571\du)--cycle;
\pgfsetbuttcap
\pgfsetmiterjoin
\pgfsetdash{}{0pt}
\definecolor{dialinecolor}{rgb}{0.247059, 0.317647, 0.709804}
\pgfsetstrokecolor{dialinecolor}
\draw (6.388782\du,8.069167\du)--(7.040160\du,8.069167\du)--(7.365849\du,8.484369\du)--(7.040160\du,8.899571\du)--(6.388782\du,8.899571\du)--cycle;
\pgfsetlinewidth{0.100000\du}
\pgfsetdash{}{0pt}
\pgfsetdash{}{0pt}
\pgfsetbuttcap
{
\definecolor{dialinecolor}{rgb}{0.000000, 0.000000, 0.000000}
\pgfsetfillcolor{dialinecolor}
% was here!!!
\pgfsetarrowsend{latex}
\definecolor{dialinecolor}{rgb}{0.000000, 0.000000, 0.000000}
\pgfsetstrokecolor{dialinecolor}
\pgfpathmoveto{\pgfpoint{9.783875\du}{9.050405\du}}
\pgfpatharc{353}{252}{1.549399\du and 1.549399\du}
\pgfusepath{stroke}
}
\definecolor{dialinecolor}{rgb}{1.000000, 1.000000, 1.000000}
\pgfsetfillcolor{dialinecolor}
\pgfpathellipse{\pgfpoint{10.750896\du}{10.500934\du}}{\pgfpoint{3.237421\du}{0\du}}{\pgfpoint{0\du}{1.072133\du}}
\pgfusepath{fill}
\pgfsetlinewidth{0.100000\du}
\pgfsetdash{}{0pt}
\pgfsetdash{}{0pt}
\definecolor{dialinecolor}{rgb}{0.000000, 0.000000, 0.000000}
\pgfsetstrokecolor{dialinecolor}
\pgfpathellipse{\pgfpoint{10.750896\du}{10.500934\du}}{\pgfpoint{3.237421\du}{0\du}}{\pgfpoint{0\du}{1.072133\du}}
\pgfusepath{stroke}
% setfont left to latex
\definecolor{dialinecolor}{rgb}{0.000000, 0.000000, 0.000000}
\pgfsetstrokecolor{dialinecolor}
\node at (10.750896\du,10.314122\du){Traitement};
% setfont left to latex
\definecolor{dialinecolor}{rgb}{0.000000, 0.000000, 0.000000}
\pgfsetstrokecolor{dialinecolor}
\node at (10.750896\du,11.114122\du){de tâche};
\pgfsetlinewidth{0.080000\du}
\pgfsetmiterjoin
\pgfsetdash{}{0pt}
\definecolor{dialinecolor}{rgb}{0.000000, 0.000000, 0.000000}
\pgfsetfillcolor{dialinecolor}
\fill (8.270275\du,13.633244\du)--(8.175675\du,14.276524\du)--(8.610993\du,14.340541\du)--(8.060053\du,15.062754\du)--(8.154653\du,14.419475\du)--(7.719335\du,14.355457\du)--cycle;
\definecolor{dialinecolor}{rgb}{0.000000, 0.000000, 0.000000}
\pgfsetstrokecolor{dialinecolor}
\draw (8.270275\du,13.633244\du)--(8.175675\du,14.276524\du)--(8.610993\du,14.340541\du)--(8.060053\du,15.062754\du)--(8.154653\du,14.419475\du)--(7.719335\du,14.355457\du)--cycle;
\pgfsetlinewidth{0.080000\du}
\pgfsetmiterjoin
\pgfsetdash{}{0pt}
\definecolor{dialinecolor}{rgb}{0.000000, 0.000000, 0.000000}
\pgfsetfillcolor{dialinecolor}
\fill (7.404318\du,13.822444\du)--(7.309718\du,14.465723\du)--(7.745036\du,14.529741\du)--(7.194096\du,15.251954\du)--(7.288696\du,14.608675\du)--(6.853378\du,14.544657\du)--cycle;
\definecolor{dialinecolor}{rgb}{0.000000, 0.000000, 0.000000}
\pgfsetstrokecolor{dialinecolor}
\draw (7.404318\du,13.822444\du)--(7.309718\du,14.465723\du)--(7.745036\du,14.529741\du)--(7.194096\du,15.251954\du)--(7.288696\du,14.608675\du)--(6.853378\du,14.544657\du)--cycle;
\pgfsetlinewidth{0.080000\du}
\pgfsetmiterjoin
\pgfsetdash{}{0pt}
\definecolor{dialinecolor}{rgb}{0.000000, 0.000000, 0.000000}
\pgfsetfillcolor{dialinecolor}
\fill (9.056664\du,14.263910\du)--(8.962064\du,14.907190\du)--(9.397382\du,14.971207\du)--(8.846442\du,15.693421\du)--(8.941042\du,15.050141\du)--(8.505724\du,14.986124\du)--cycle;
\definecolor{dialinecolor}{rgb}{0.000000, 0.000000, 0.000000}
\pgfsetstrokecolor{dialinecolor}
\draw (9.056664\du,14.263910\du)--(8.962064\du,14.907190\du)--(9.397382\du,14.971207\du)--(8.846442\du,15.693421\du)--(8.941042\du,15.050141\du)--(8.505724\du,14.986124\du)--cycle;
\pgfsetlinewidth{0.080000\du}
\pgfsetmiterjoin
\pgfsetdash{}{0pt}
\definecolor{dialinecolor}{rgb}{0.000000, 0.000000, 0.000000}
\pgfsetfillcolor{dialinecolor}
\fill (9.910166\du,13.948577\du)--(9.815566\du,14.591857\du)--(10.250884\du,14.655874\du)--(9.699944\du,15.378088\du)--(9.794544\du,14.734808\du)--(9.359226\du,14.670790\du)--cycle;
\definecolor{dialinecolor}{rgb}{0.000000, 0.000000, 0.000000}
\pgfsetstrokecolor{dialinecolor}
\draw (9.910166\du,13.948577\du)--(9.815566\du,14.591857\du)--(10.250884\du,14.655874\du)--(9.699944\du,15.378088\du)--(9.794544\du,14.734808\du)--(9.359226\du,14.670790\du)--cycle;
\pgfsetlinewidth{0.080000\du}
\pgfsetmiterjoin
\pgfsetdash{}{0pt}
\definecolor{dialinecolor}{rgb}{0.000000, 0.000000, 0.000000}
\pgfsetfillcolor{dialinecolor}
\fill (10.889801\du,14.179821\du)--(10.795201\du,14.823101\du)--(11.230519\du,14.887118\du)--(10.679579\du,15.609332\du)--(10.774179\du,14.966052\du)--(10.338861\du,14.902035\du)--cycle;
\definecolor{dialinecolor}{rgb}{0.000000, 0.000000, 0.000000}
\pgfsetstrokecolor{dialinecolor}
\draw (10.889801\du,14.179821\du)--(10.795201\du,14.823101\du)--(11.230519\du,14.887118\du)--(10.679579\du,15.609332\du)--(10.774179\du,14.966052\du)--(10.338861\du,14.902035\du)--cycle;
\pgfsetlinewidth{0.080000\du}
\pgfsetmiterjoin
\pgfsetdash{}{0pt}
\definecolor{dialinecolor}{rgb}{0.000000, 0.000000, 0.000000}
\pgfsetfillcolor{dialinecolor}
\fill (11.659214\du,13.780399\du)--(11.564614\du,14.423679\du)--(11.999932\du,14.487696\du)--(11.448992\du,15.209910\du)--(11.543592\du,14.566630\du)--(11.108274\du,14.502613\du)--cycle;
\definecolor{dialinecolor}{rgb}{0.000000, 0.000000, 0.000000}
\pgfsetstrokecolor{dialinecolor}
\draw (11.659214\du,13.780399\du)--(11.564614\du,14.423679\du)--(11.999932\du,14.487696\du)--(11.448992\du,15.209910\du)--(11.543592\du,14.566630\du)--(11.108274\du,14.502613\du)--cycle;
\pgfsetlinewidth{0.080000\du}
\pgfsetmiterjoin
\pgfsetdash{}{0pt}
\definecolor{dialinecolor}{rgb}{0.000000, 0.000000, 0.000000}
\pgfsetfillcolor{dialinecolor}
\fill (12.596805\du,13.633244\du)--(12.502205\du,14.276524\du)--(12.937523\du,14.340541\du)--(12.386582\du,15.062754\du)--(12.481182\du,14.419475\du)--(12.045864\du,14.355457\du)--cycle;
\definecolor{dialinecolor}{rgb}{0.000000, 0.000000, 0.000000}
\pgfsetstrokecolor{dialinecolor}
\draw (12.596805\du,13.633244\du)--(12.502205\du,14.276524\du)--(12.937523\du,14.340541\du)--(12.386582\du,15.062754\du)--(12.481182\du,14.419475\du)--(12.045864\du,14.355457\du)--cycle;
% setfont left to latex
\definecolor{dialinecolor}{rgb}{0.000000, 0.000000, 0.000000}
\pgfsetstrokecolor{dialinecolor}
\node[anchor=west] at (4.832258\du,6.406250\du){Stage};
% setfont left to latex
\definecolor{dialinecolor}{rgb}{0.000000, 0.000000, 0.000000}
\pgfsetstrokecolor{dialinecolor}
\node[anchor=west] at (1.689994\du,9.534378\du){File d'attente};
% setfont left to latex
\definecolor{dialinecolor}{rgb}{0.000000, 0.000000, 0.000000}
\pgfsetstrokecolor{dialinecolor}
\node[anchor=west] at (6.584397\du,16.346826\du){Ensemble de threads};
\pgfsetlinewidth{0.100000\du}
\pgfsetdash{}{0pt}
\pgfsetdash{}{0pt}
\pgfsetbuttcap
{
\definecolor{dialinecolor}{rgb}{0.000000, 0.000000, 0.000000}
\pgfsetfillcolor{dialinecolor}
% was here!!!
\pgfsetarrowsend{latex}
\definecolor{dialinecolor}{rgb}{0.000000, 0.000000, 0.000000}
\pgfsetstrokecolor{dialinecolor}
\pgfpathmoveto{\pgfpoint{6.336231\du}{9.933251\du}}
\pgfpatharc{180}{81}{1.589460\du and 1.589460\du}
\pgfusepath{stroke}
}
\pgfsetlinewidth{0.100000\du}
\pgfsetdash{{\pgflinewidth}{0.200000\du}}{0cm}
\pgfsetdash{{\pgflinewidth}{0.200000\du}}{0cm}
\pgfsetbuttcap
{
\definecolor{dialinecolor}{rgb}{0.000000, 0.000000, 0.000000}
\pgfsetfillcolor{dialinecolor}
% was here!!!
\pgfsetarrowsend{latex}
\definecolor{dialinecolor}{rgb}{0.000000, 0.000000, 0.000000}
\pgfsetstrokecolor{dialinecolor}
\draw (7.681653\du,13.591199\du)--(8.985030\du,11.573067\du);
}
\pgfsetlinewidth{0.100000\du}
\pgfsetdash{{\pgflinewidth}{0.200000\du}}{0cm}
\pgfsetdash{{\pgflinewidth}{0.200000\du}}{0cm}
\pgfsetbuttcap
{
\definecolor{dialinecolor}{rgb}{0.000000, 0.000000, 0.000000}
\pgfsetfillcolor{dialinecolor}
% was here!!!
\pgfsetarrowsend{latex}
\definecolor{dialinecolor}{rgb}{0.000000, 0.000000, 0.000000}
\pgfsetstrokecolor{dialinecolor}
\draw (9.571305\du,13.664096\du)--(9.573652\du,11.825334\du);
}
\pgfsetlinewidth{0.100000\du}
\pgfsetdash{{\pgflinewidth}{0.200000\du}}{0cm}
\pgfsetdash{{\pgflinewidth}{0.200000\du}}{0cm}
\pgfsetbuttcap
{
\definecolor{dialinecolor}{rgb}{0.000000, 0.000000, 0.000000}
\pgfsetfillcolor{dialinecolor}
% was here!!!
\pgfsetarrowsend{latex}
\definecolor{dialinecolor}{rgb}{0.000000, 0.000000, 0.000000}
\pgfsetstrokecolor{dialinecolor}
\draw (10.761162\du,13.895341\du)--(10.582718\du,11.993511\du);
}
\pgfsetlinewidth{0.100000\du}
\pgfsetdash{{\pgflinewidth}{0.200000\du}}{0cm}
\pgfsetdash{{\pgflinewidth}{0.200000\du}}{0cm}
\pgfsetbuttcap
{
\definecolor{dialinecolor}{rgb}{0.000000, 0.000000, 0.000000}
\pgfsetfillcolor{dialinecolor}
% was here!!!
\pgfsetarrowsend{latex}
\definecolor{dialinecolor}{rgb}{0.000000, 0.000000, 0.000000}
\pgfsetstrokecolor{dialinecolor}
\draw (11.824886\du,13.495919\du)--(11.633829\du,11.867378\du);
}
\pgfsetlinewidth{0.100000\du}
\pgfsetdash{}{0pt}
\pgfsetdash{}{0pt}
\pgfsetbuttcap
\pgfsetmiterjoin
\pgfsetlinewidth{0.100000\du}
\pgfsetbuttcap
\pgfsetmiterjoin
\pgfsetdash{}{0pt}
\definecolor{dialinecolor}{rgb}{0.960784, 0.960784, 0.960784}
\pgfsetfillcolor{dialinecolor}
\fill (6.768893\du,22.149985\du)--(6.768893\du,33.069985\du)--(17.336635\du,33.069985\du)--(17.336635\du,22.149985\du)--cycle;
\definecolor{dialinecolor}{rgb}{0.000000, 0.000000, 0.000000}
\pgfsetstrokecolor{dialinecolor}
\draw (6.768893\du,22.149985\du)--(6.768893\du,33.069985\du)--(17.336635\du,33.069985\du)--(17.336635\du,22.149985\du)--cycle;
\pgfsetbuttcap
\pgfsetmiterjoin
\pgfsetdash{}{0pt}
\definecolor{dialinecolor}{rgb}{0.000000, 0.000000, 0.000000}
\pgfsetstrokecolor{dialinecolor}
\draw (6.768893\du,22.149985\du)--(6.768893\du,33.069985\du)--(17.336635\du,33.069985\du)--(17.336635\du,22.149985\du)--cycle;
\pgfsetlinewidth{0.000000\du}
\pgfsetdash{}{0pt}
\pgfsetdash{}{0pt}
\pgfsetbuttcap
\pgfsetmiterjoin
\pgfsetlinewidth{0.000000\du}
\pgfsetbuttcap
\pgfsetmiterjoin
\pgfsetdash{}{0pt}
\definecolor{dialinecolor}{rgb}{0.247059, 0.317647, 0.709804}
\pgfsetfillcolor{dialinecolor}
\fill (3.626629\du,23.628947\du)--(3.626629\du,24.533114\du)--(4.501629\du,24.533114\du)--(4.501629\du,23.628947\du)--cycle;
\definecolor{dialinecolor}{rgb}{0.247059, 0.317647, 0.709804}
\pgfsetstrokecolor{dialinecolor}
\draw (3.626629\du,23.628947\du)--(3.626629\du,24.533114\du)--(4.501629\du,24.533114\du)--(4.501629\du,23.628947\du)--cycle;
\pgfsetbuttcap
\pgfsetmiterjoin
\pgfsetdash{}{0pt}
\definecolor{dialinecolor}{rgb}{0.247059, 0.317647, 0.709804}
\pgfsetstrokecolor{dialinecolor}
\draw (3.626629\du,23.628947\du)--(3.626629\du,24.533114\du)--(4.501629\du,24.533114\du)--(4.501629\du,23.628947\du)--cycle;
\pgfsetlinewidth{0.000000\du}
\pgfsetdash{}{0pt}
\pgfsetdash{}{0pt}
\pgfsetbuttcap
\pgfsetmiterjoin
\pgfsetlinewidth{0.000000\du}
\pgfsetbuttcap
\pgfsetmiterjoin
\pgfsetdash{}{0pt}
\definecolor{dialinecolor}{rgb}{0.247059, 0.317647, 0.709804}
\pgfsetfillcolor{dialinecolor}
\fill (4.781209\du,23.628947\du)--(4.781209\du,24.533114\du)--(5.656209\du,24.533114\du)--(5.656209\du,23.628947\du)--cycle;
\definecolor{dialinecolor}{rgb}{0.247059, 0.317647, 0.709804}
\pgfsetstrokecolor{dialinecolor}
\draw (4.781209\du,23.628947\du)--(4.781209\du,24.533114\du)--(5.656209\du,24.533114\du)--(5.656209\du,23.628947\du)--cycle;
\pgfsetbuttcap
\pgfsetmiterjoin
\pgfsetdash{}{0pt}
\definecolor{dialinecolor}{rgb}{0.247059, 0.317647, 0.709804}
\pgfsetstrokecolor{dialinecolor}
\draw (4.781209\du,23.628947\du)--(4.781209\du,24.533114\du)--(5.656209\du,24.533114\du)--(5.656209\du,23.628947\du)--cycle;
\pgfsetlinewidth{0.000000\du}
\pgfsetdash{}{0pt}
\pgfsetdash{}{0pt}
\pgfsetbuttcap
\pgfsetmiterjoin
\pgfsetlinewidth{0.000000\du}
\pgfsetbuttcap
\pgfsetmiterjoin
\pgfsetdash{}{0pt}
\definecolor{dialinecolor}{rgb}{0.247059, 0.317647, 0.709804}
\pgfsetfillcolor{dialinecolor}
\fill (5.930958\du,23.620600\du)--(5.930958\du,24.524767\du)--(6.805958\du,24.524767\du)--(6.805958\du,23.620600\du)--cycle;
\definecolor{dialinecolor}{rgb}{0.247059, 0.317647, 0.709804}
\pgfsetstrokecolor{dialinecolor}
\draw (5.930958\du,23.620600\du)--(5.930958\du,24.524767\du)--(6.805958\du,24.524767\du)--(6.805958\du,23.620600\du)--cycle;
\pgfsetbuttcap
\pgfsetmiterjoin
\pgfsetdash{}{0pt}
\definecolor{dialinecolor}{rgb}{0.247059, 0.317647, 0.709804}
\pgfsetstrokecolor{dialinecolor}
\draw (5.930958\du,23.620600\du)--(5.930958\du,24.524767\du)--(6.805958\du,24.524767\du)--(6.805958\du,23.620600\du)--cycle;
\pgfsetlinewidth{0.000000\du}
\pgfsetdash{}{0pt}
\pgfsetdash{}{0pt}
\pgfsetbuttcap
\pgfsetmiterjoin
\pgfsetlinewidth{0.000000\du}
\pgfsetbuttcap
\pgfsetmiterjoin
\pgfsetdash{}{0pt}
\definecolor{dialinecolor}{rgb}{0.247059, 0.317647, 0.709804}
\pgfsetfillcolor{dialinecolor}
\fill (7.059703\du,23.626850\du)--(7.059703\du,24.531017\du)--(7.934703\du,24.531017\du)--(7.934703\du,23.626850\du)--cycle;
\definecolor{dialinecolor}{rgb}{0.247059, 0.317647, 0.709804}
\pgfsetstrokecolor{dialinecolor}
\draw (7.059703\du,23.626850\du)--(7.059703\du,24.531017\du)--(7.934703\du,24.531017\du)--(7.934703\du,23.626850\du)--cycle;
\pgfsetbuttcap
\pgfsetmiterjoin
\pgfsetdash{}{0pt}
\definecolor{dialinecolor}{rgb}{0.247059, 0.317647, 0.709804}
\pgfsetstrokecolor{dialinecolor}
\draw (7.059703\du,23.626850\du)--(7.059703\du,24.531017\du)--(7.934703\du,24.531017\du)--(7.934703\du,23.626850\du)--cycle;
\pgfsetlinewidth{0.100000\du}
\pgfsetdash{}{0pt}
\pgfsetdash{}{0pt}
\pgfsetbuttcap
\pgfsetmiterjoin
\pgfsetlinewidth{0.100000\du}
\pgfsetbuttcap
\pgfsetmiterjoin
\pgfsetdash{}{0pt}
\definecolor{dialinecolor}{rgb}{0.247059, 0.317647, 0.709804}
\pgfsetfillcolor{dialinecolor}
\fill (8.325417\du,23.662902\du)--(8.976795\du,23.662902\du)--(9.302484\du,24.078104\du)--(8.976795\du,24.493306\du)--(8.325417\du,24.493306\du)--cycle;
\definecolor{dialinecolor}{rgb}{0.247059, 0.317647, 0.709804}
\pgfsetstrokecolor{dialinecolor}
\draw (8.325417\du,23.662902\du)--(8.976795\du,23.662902\du)--(9.302484\du,24.078104\du)--(8.976795\du,24.493306\du)--(8.325417\du,24.493306\du)--cycle;
\pgfsetbuttcap
\pgfsetmiterjoin
\pgfsetdash{}{0pt}
\definecolor{dialinecolor}{rgb}{0.247059, 0.317647, 0.709804}
\pgfsetstrokecolor{dialinecolor}
\draw (8.325417\du,23.662902\du)--(8.976795\du,23.662902\du)--(9.302484\du,24.078104\du)--(8.976795\du,24.493306\du)--(8.325417\du,24.493306\du)--cycle;
\pgfsetlinewidth{0.100000\du}
\pgfsetdash{}{0pt}
\pgfsetdash{}{0pt}
\pgfsetbuttcap
{
\definecolor{dialinecolor}{rgb}{0.000000, 0.000000, 0.000000}
\pgfsetfillcolor{dialinecolor}
% was here!!!
\pgfsetarrowsend{latex}
\definecolor{dialinecolor}{rgb}{0.000000, 0.000000, 0.000000}
\pgfsetstrokecolor{dialinecolor}
\pgfpathmoveto{\pgfpoint{11.720510\du}{24.644140\du}}
\pgfpatharc{353}{252}{1.549399\du and 1.549399\du}
\pgfusepath{stroke}
}
\definecolor{dialinecolor}{rgb}{1.000000, 1.000000, 1.000000}
\pgfsetfillcolor{dialinecolor}
\pgfpathellipse{\pgfpoint{12.687531\du}{26.094669\du}}{\pgfpoint{3.237421\du}{0\du}}{\pgfpoint{0\du}{1.072133\du}}
\pgfusepath{fill}
\pgfsetlinewidth{0.100000\du}
\pgfsetdash{}{0pt}
\pgfsetdash{}{0pt}
\definecolor{dialinecolor}{rgb}{0.000000, 0.000000, 0.000000}
\pgfsetstrokecolor{dialinecolor}
\pgfpathellipse{\pgfpoint{12.687531\du}{26.094669\du}}{\pgfpoint{3.237421\du}{0\du}}{\pgfpoint{0\du}{1.072133\du}}
\pgfusepath{stroke}
% setfont left to latex
\definecolor{dialinecolor}{rgb}{0.000000, 0.000000, 0.000000}
\pgfsetstrokecolor{dialinecolor}
\node at (12.687531\du,25.917169\du){Traitement};
% setfont left to latex
\definecolor{dialinecolor}{rgb}{0.000000, 0.000000, 0.000000}
\pgfsetstrokecolor{dialinecolor}
\node at (12.687531\du,26.717169\du){de tâche};
\pgfsetlinewidth{0.080000\du}
\pgfsetmiterjoin
\pgfsetdash{}{0pt}
\definecolor{dialinecolor}{rgb}{0.000000, 0.000000, 0.000000}
\pgfsetfillcolor{dialinecolor}
\fill (10.206910\du,29.226979\du)--(10.112310\du,29.870259\du)--(10.547628\du,29.934276\du)--(9.996688\du,30.656489\du)--(10.091288\du,30.013210\du)--(9.655970\du,29.949192\du)--cycle;
\definecolor{dialinecolor}{rgb}{0.000000, 0.000000, 0.000000}
\pgfsetstrokecolor{dialinecolor}
\draw (10.206910\du,29.226979\du)--(10.112310\du,29.870259\du)--(10.547628\du,29.934276\du)--(9.996688\du,30.656489\du)--(10.091288\du,30.013210\du)--(9.655970\du,29.949192\du)--cycle;
\pgfsetlinewidth{0.080000\du}
\pgfsetmiterjoin
\pgfsetdash{}{0pt}
\definecolor{dialinecolor}{rgb}{0.000000, 0.000000, 0.000000}
\pgfsetfillcolor{dialinecolor}
\fill (9.340953\du,29.416179\du)--(9.246353\du,30.059459\du)--(9.681671\du,30.123476\du)--(9.130731\du,30.845689\du)--(9.225331\du,30.202410\du)--(8.790013\du,30.138392\du)--cycle;
\definecolor{dialinecolor}{rgb}{0.000000, 0.000000, 0.000000}
\pgfsetstrokecolor{dialinecolor}
\draw (9.340953\du,29.416179\du)--(9.246353\du,30.059459\du)--(9.681671\du,30.123476\du)--(9.130731\du,30.845689\du)--(9.225331\du,30.202410\du)--(8.790013\du,30.138392\du)--cycle;
\pgfsetlinewidth{0.080000\du}
\pgfsetmiterjoin
\pgfsetdash{}{0pt}
\definecolor{dialinecolor}{rgb}{0.000000, 0.000000, 0.000000}
\pgfsetfillcolor{dialinecolor}
\fill (10.993299\du,29.857645\du)--(10.898699\du,30.500925\du)--(11.334017\du,30.564942\du)--(10.783077\du,31.287156\du)--(10.877677\du,30.643876\du)--(10.442359\du,30.579859\du)--cycle;
\definecolor{dialinecolor}{rgb}{0.000000, 0.000000, 0.000000}
\pgfsetstrokecolor{dialinecolor}
\draw (10.993299\du,29.857645\du)--(10.898699\du,30.500925\du)--(11.334017\du,30.564942\du)--(10.783077\du,31.287156\du)--(10.877677\du,30.643876\du)--(10.442359\du,30.579859\du)--cycle;
\pgfsetlinewidth{0.080000\du}
\pgfsetmiterjoin
\pgfsetdash{}{0pt}
\definecolor{dialinecolor}{rgb}{0.000000, 0.000000, 0.000000}
\pgfsetfillcolor{dialinecolor}
\fill (11.846801\du,29.542312\du)--(11.752201\du,30.185592\du)--(12.187519\du,30.249609\du)--(11.636579\du,30.971823\du)--(11.731179\du,30.328543\du)--(11.295861\du,30.264526\du)--cycle;
\definecolor{dialinecolor}{rgb}{0.000000, 0.000000, 0.000000}
\pgfsetstrokecolor{dialinecolor}
\draw (11.846801\du,29.542312\du)--(11.752201\du,30.185592\du)--(12.187519\du,30.249609\du)--(11.636579\du,30.971823\du)--(11.731179\du,30.328543\du)--(11.295861\du,30.264526\du)--cycle;
\pgfsetlinewidth{0.080000\du}
\pgfsetmiterjoin
\pgfsetdash{}{0pt}
\definecolor{dialinecolor}{rgb}{0.000000, 0.000000, 0.000000}
\pgfsetfillcolor{dialinecolor}
\fill (12.826436\du,29.773557\du)--(12.731836\du,30.416836\du)--(13.167154\du,30.480854\du)--(12.616214\du,31.203067\du)--(12.710814\du,30.559787\du)--(12.275496\du,30.495770\du)--cycle;
\definecolor{dialinecolor}{rgb}{0.000000, 0.000000, 0.000000}
\pgfsetstrokecolor{dialinecolor}
\draw (12.826436\du,29.773557\du)--(12.731836\du,30.416836\du)--(13.167154\du,30.480854\du)--(12.616214\du,31.203067\du)--(12.710814\du,30.559787\du)--(12.275496\du,30.495770\du)--cycle;
\pgfsetlinewidth{0.080000\du}
\pgfsetmiterjoin
\pgfsetdash{}{0pt}
\definecolor{dialinecolor}{rgb}{0.000000, 0.000000, 0.000000}
\pgfsetfillcolor{dialinecolor}
\fill (13.595849\du,29.374134\du)--(13.501249\du,30.017414\du)--(13.936567\du,30.081432\du)--(13.385627\du,30.803645\du)--(13.480227\du,30.160365\du)--(13.044909\du,30.096348\du)--cycle;
\definecolor{dialinecolor}{rgb}{0.000000, 0.000000, 0.000000}
\pgfsetstrokecolor{dialinecolor}
\draw (13.595849\du,29.374134\du)--(13.501249\du,30.017414\du)--(13.936567\du,30.081432\du)--(13.385627\du,30.803645\du)--(13.480227\du,30.160365\du)--(13.044909\du,30.096348\du)--cycle;
\pgfsetlinewidth{0.080000\du}
\pgfsetmiterjoin
\pgfsetdash{}{0pt}
\definecolor{dialinecolor}{rgb}{0.000000, 0.000000, 0.000000}
\pgfsetfillcolor{dialinecolor}
\fill (14.533440\du,29.226979\du)--(14.438840\du,29.870259\du)--(14.874158\du,29.934276\du)--(14.323218\du,30.656489\du)--(14.417817\du,30.013210\du)--(13.982499\du,29.949192\du)--cycle;
\definecolor{dialinecolor}{rgb}{0.000000, 0.000000, 0.000000}
\pgfsetstrokecolor{dialinecolor}
\draw (14.533440\du,29.226979\du)--(14.438840\du,29.870259\du)--(14.874158\du,29.934276\du)--(14.323218\du,30.656489\du)--(14.417817\du,30.013210\du)--(13.982499\du,29.949192\du)--cycle;
% setfont left to latex
\definecolor{dialinecolor}{rgb}{0.000000, 0.000000, 0.000000}
\pgfsetstrokecolor{dialinecolor}
\node[anchor=west] at (6.768893\du,21.999985\du){Stage};
% setfont left to latex
\definecolor{dialinecolor}{rgb}{0.000000, 0.000000, 0.000000}
\pgfsetstrokecolor{dialinecolor}
\node[anchor=west] at (3.626629\du,25.128114\du){File d'attente};
% setfont left to latex
\definecolor{dialinecolor}{rgb}{0.000000, 0.000000, 0.000000}
\pgfsetstrokecolor{dialinecolor}
\node[anchor=west] at (8.521032\du,31.940561\du){Ensemble de threads};
\pgfsetlinewidth{0.100000\du}
\pgfsetdash{}{0pt}
\pgfsetdash{}{0pt}
\pgfsetbuttcap
{
\definecolor{dialinecolor}{rgb}{0.000000, 0.000000, 0.000000}
\pgfsetfillcolor{dialinecolor}
% was here!!!
\pgfsetarrowsend{latex}
\definecolor{dialinecolor}{rgb}{0.000000, 0.000000, 0.000000}
\pgfsetstrokecolor{dialinecolor}
\pgfpathmoveto{\pgfpoint{8.272866\du}{25.526986\du}}
\pgfpatharc{180}{81}{1.589460\du and 1.589460\du}
\pgfusepath{stroke}
}
\pgfsetlinewidth{0.100000\du}
\pgfsetdash{{\pgflinewidth}{0.200000\du}}{0cm}
\pgfsetdash{{\pgflinewidth}{0.200000\du}}{0cm}
\pgfsetbuttcap
{
\definecolor{dialinecolor}{rgb}{0.000000, 0.000000, 0.000000}
\pgfsetfillcolor{dialinecolor}
% was here!!!
\pgfsetarrowsend{latex}
\definecolor{dialinecolor}{rgb}{0.000000, 0.000000, 0.000000}
\pgfsetstrokecolor{dialinecolor}
\draw (9.618288\du,29.184935\du)--(10.921665\du,27.166802\du);
}
\pgfsetlinewidth{0.100000\du}
\pgfsetdash{{\pgflinewidth}{0.200000\du}}{0cm}
\pgfsetdash{{\pgflinewidth}{0.200000\du}}{0cm}
\pgfsetbuttcap
{
\definecolor{dialinecolor}{rgb}{0.000000, 0.000000, 0.000000}
\pgfsetfillcolor{dialinecolor}
% was here!!!
\pgfsetarrowsend{latex}
\definecolor{dialinecolor}{rgb}{0.000000, 0.000000, 0.000000}
\pgfsetstrokecolor{dialinecolor}
\draw (11.507940\du,29.257832\du)--(11.510287\du,27.419069\du);
}
\pgfsetlinewidth{0.100000\du}
\pgfsetdash{{\pgflinewidth}{0.200000\du}}{0cm}
\pgfsetdash{{\pgflinewidth}{0.200000\du}}{0cm}
\pgfsetbuttcap
{
\definecolor{dialinecolor}{rgb}{0.000000, 0.000000, 0.000000}
\pgfsetfillcolor{dialinecolor}
% was here!!!
\pgfsetarrowsend{latex}
\definecolor{dialinecolor}{rgb}{0.000000, 0.000000, 0.000000}
\pgfsetstrokecolor{dialinecolor}
\draw (12.697797\du,29.489076\du)--(12.519353\du,27.587246\du);
}
\pgfsetlinewidth{0.100000\du}
\pgfsetdash{{\pgflinewidth}{0.200000\du}}{0cm}
\pgfsetdash{{\pgflinewidth}{0.200000\du}}{0cm}
\pgfsetbuttcap
{
\definecolor{dialinecolor}{rgb}{0.000000, 0.000000, 0.000000}
\pgfsetfillcolor{dialinecolor}
% was here!!!
\pgfsetarrowsend{latex}
\definecolor{dialinecolor}{rgb}{0.000000, 0.000000, 0.000000}
\pgfsetstrokecolor{dialinecolor}
\draw (13.761521\du,29.089654\du)--(13.570464\du,27.461113\du);
}
\pgfsetlinewidth{0.200000\du}
\pgfsetdash{}{0pt}
\pgfsetdash{}{0pt}
\pgfsetmiterjoin
\pgfsetbuttcap
{
\definecolor{dialinecolor}{rgb}{0.000000, 0.000000, 0.000000}
\pgfsetfillcolor{dialinecolor}
% was here!!!
\pgfsetarrowsend{latex}
\definecolor{dialinecolor}{rgb}{0.000000, 0.000000, 0.000000}
\pgfsetstrokecolor{dialinecolor}
\pgfpathmoveto{\pgfpoint{13.988317\du}{10.500934\du}}
\pgfpathcurveto{\pgfpoint{18.441040\du}{10.922577\du}}{\pgfpoint{21.439908\du}{10.759002\du}}{\pgfpoint{21.167284\du}{15.502667\du}}
\pgfpathcurveto{\pgfpoint{20.894660\du}{20.246331\du}}{\pgfpoint{10.044208\du}{20.191807\du}}{\pgfpoint{6.336516\du}{20.137282\du}}
\pgfpathcurveto{\pgfpoint{2.628824\du}{20.082757\du}}{\pgfpoint{0.175204\du}{24.390222\du}}{\pgfpoint{3.626629\du}{24.081030\du}}
\pgfusepath{stroke}
}
% setfont left to latex
\definecolor{dialinecolor}{rgb}{0.000000, 0.000000, 0.000000}
\pgfsetstrokecolor{dialinecolor}
\node[anchor=west] at (16.423619\du,10.540902\du){Nouvelle tâche};
\end{tikzpicture}

	\caption{Schéma de deux stages dans une architecture SEDA \label{fig:stages}}
\end{figure}

\end{document}