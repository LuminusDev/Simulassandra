\documentclass[12pt]{article}
\usepackage[utf8]{inputenc}
\usepackage[french]{babel}
\usepackage[tikz]{bclogo}

\title{
 \begin{minipage}\linewidth
        \centering
        Simulation d'algorithmes d'équilibrage de charge dans un environnement distribué 
        \vskip3pt
        \large Identifications des besoins
    \end{minipage}
 }
    
\author{Kevin Barreau \and Guillaume Marques \and Corentin Salingue}

\begin{document}

\maketitle

\abstract
Ce document dégage une première identification des besoins.
Il s'agit d'un document support pour l'élaboration du cahier des charges

\newpage

Nous devons développer une solution logicielle permettant de tester des algorithmes d'équilibrage de charge et de réplication des objets.

\section{Besoins fonctionnels}

L'environnement de simulation voulu est un système distribué constitué de $n$ noeuds de stockage dans lequel on souhaite stocker $m$ objets.

\subsection{Environnement distribué}

\begin{bclogo}[ombre = true, epOmbre = 0.25, couleurOmbre = black!30, logo = \bcinfo, arrondi = 0.1]{Environnement distribué}
	Un environnement distribué est constitué de plusieurs machines appelées $noeuds$ sur lesquelles sont stockées les données et sont distribuées les tâches à effectuer, appelées aussi $requetes$.%
\end{bclogo}

	Le client souhaite l'utilisation de l'environnement distribué Cassandra car il pense que tous les aspects qu'il souhaite tester sont facilement implémentables ici.

\subsection{Gestion des noeuds}


\begin{itemize}
 \item Creer un noeud
 \item Supprimer un noeud
 \item Sauvegarder un ensemble de noeuds avec leurs objets
 \item Importer un ensemble de noeuds avec leurs objets afin de simuler sur un environemment précedemment crée.
\end{itemize}

\vspace{0.5cm}

\textbf{Question } Quelle format pour le stockage d'un ensemble de noeuds ? \newline
\textbf{Question } Combien de noeuds maximum ? \newline
\textbf{Question } Position des noeuds sur le ring ? \newline


\subsection{Gestion des objets}

\textbf{Question } Sur quel type d'objet allons-nous travailler ? \newline

\begin{itemize}
 \item Creer un objet
 \item Supprimer un objet
 \item Gérer la popularité d'un objet
 \begin{itemize}
  \item Implémenter l'algorithme d'approximation Space-Saving Algorithm
  \item Ajouter un vecteur de taille $n$ à chaque noeud dans le lequel on stockera les objets les plus populaires
  \item Protocole de communication pour le calcul des objets les plus populaires \textbf{(Modifier gossip? Nouveau ?)}
 \end{itemize}
\end{itemize}

\vspace{0.5cm}

\textbf{Question } Gestion de la popularité : quelle période (fixe ou fenêtre glissante) ? \newline
\textbf{Question } Position de l'objet sur le noeud ? \newline
\textbf{Note } Bien définir l'algorithme. \newline

\subsection{Gestion des requêtes}

\begin{itemize}
 \item Creer une requête
 \item Envoyer une requête sur le réseau
 \item Sauvegarder un jeu de requête
 \item Importer un jeu de requête
\end{itemize}

\vspace{0.5cm}

\textbf{Question } Quelle format pour le stockage d'un jeu de requêtes ? \newline

\subsection{Visualisation des données}

\begin{itemize}
 \item Temps de réponse moyen sur les requêtes passées.
 \item Charge d'un noeud
 \item Popularité des objets
\end{itemize}

\vspace{0.5cm}

\textbf{Note } Bien définir ces items. \newline
\end{document}
