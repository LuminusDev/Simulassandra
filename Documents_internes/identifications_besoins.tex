\documentclass[12pt]{article}
\usepackage[utf8]{inputenc}
\usepackage[french]{babel}
\usepackage[tikz]{bclogo}
\usepackage{geometry}
\usepackage{array}
\usepackage{graphics}
\usepackage{graphicx}
\usepackage{pgfgantt}
\usepackage{url}
\bibliographystyle{alpha}
\usepackage[counterclockwise]{rotating}
\geometry{hmargin=2.5cm,vmargin=1.5cm}

\setlength{\parskip}{1ex plus 2ex minus 1ex}
\newcolumntype{M}[1]{
    >{\raggedright}m{#1}
}

\newcommand{\besoin}[2] {
  (\textit{Priorité} : #1, \textit{criticité} : #2)
}

\title{
 \begin{minipage}\linewidth
        \centering
        Simulation d'algorithmes d'équilibrage de charge dans un environnement distribué 
        \vskip3pt
        \large Identifications des besoins
    \end{minipage}
 }
 
\bibliographystyle{alpha}
\author{Kevin Barreau \and Guillaume Marques \and Corentin Salingue}

\begin{document}

\maketitle

\abstract
Dans une première partie, nous présentons le projet, le contexte et les hyothèses. Ensuite, nous développons les besoins fonctionnels et les besoins non-fonctionnels.
De plus, nous dégageons une première version de la plannification du projet (GANTT). Enfin, nous présentons les livrables.



\newpage


\renewcommand{\contentsname}{Sommaire} 
\tableofcontents



\newpage

\section{Définition du projet}

\subsection{Contexte}

\paragraph{Définition} Un \textit{environnement distribué} est constitué de plusieurs machines (ordinateurs généralement), appelées \textit{noeuds}, sur lesquelles sont stockées des données et pouvant traiter des requêtes. Chaque noeud possède des informations locales propres à son fonctionnement (exemple : une liste des noeuds voisins).

\paragraph{Définition} Une \textit{base de données} est une entité permettant de stocker des données afin d'en faciliter l'exploitation (ajout, mise à jour, recherche de données).

\paragraph{Définition} Une \textit{donnée} ou un \textit{objet} est un codage (une représentation sous forme binaire), propre au système de base de données, d'une information quelconque.

%Assez précis ?%

%\paragraph{} Diff BD et BD distrib et maitre esclave (faire, apparaitre file d'attente & def charge )

\paragraph{Définition} Une \textit{requête} est une interrogation d'une base de données afin de récupérer ou modifier des données.

\paragraph{Définition} Une \textit{charge} est associée à un noeud et désigne le nombre de requêtes restantes que le noeud doit traiter à l'instant $T$.

\paragraph{Définition} La \textit{réplication} d'une donnée est une action qui réalise des copies de cette donnée sur d'autres noeuds.

\paragraph{Définition} Un \textit{réseau} est un ensemble de noeuds qui sont reliés entre eux (par exemple par Internet) et qui communiquent ensemble.

\paragraph{} L'expansion, au cours des deux dernières décennies, des réseaux et notamment d'Internet a engendré une importante création de données, massives par leur nombre et leur taille.
Stocker ces informations sur un seul point de stockage (ordinateur par exemple) n'est bien sûr plus envisageable, que ce soit pour des raisons techniques ou pour des raisons de sûreté (pannes potentielles par exemple).
Pour cela des systèmes de stockages dits distribués sont utilisés en pratique afin des les répartir sur différentes unités de stockages.

\paragraph{} Pour répartir toutes ces données, notre client a développé de nouveaux algorithmes d'équilibrage de charge basés sur la réplication qu'il souhaite tester dans un environnement distribué.

%%%%%%%%%%%%%%%%%%%%%%%%%%%%%%%%%%%%%%%%

\subsection{Finalité}

\paragraph{} Nous devons développer une solution logicielle permettant de tester ces nouveaux algorithmes d'équilibrage de charge et de réplication proposés par le client dans un environnement distribué.

\paragraph{Définition} Une \textit{charge minimum} d'un environnement distribué est la plus petite charge trouvée sur un noeud parmi l'ensemble des noeuds. La \textit{charge moyenne}, est la moyenne des charges de l'ensemble des noeuds.

\paragraph{}Il s'agira d'implémenter les algorithmes développés par le client.
On distingue les algorithmes d'affectation de requête :

\begin{itemize}
 \item \textbf{SLVO} 
    Si la charge du noeud est inférieure ou égale à la charge minimum, il s'affecte toutes les requêtes en attente et en avertit les autres noeuds.
 \item \textbf{AverageDegree}
    Si la charge du noeud est inférieure ou égale à la charge moyenne, il s'affecte toutes les requêtes en attente et en avertit les autres noeuds.
\end{itemize}

\paragraph{Définition} La \textit{popularité} d'un objet est le nombre de requête que va recevoir un objet durant un intervalle de temps $T$ défini par l'utilisateur.

\paragraph{}Ainsi que l'algorithme de gestion de copie, permettant d'établir le nombre de réplicas d'un objet en fonction de sa popularité.

\paragraph{} Pour comparer l'efficacité de ces algorithmes, on peut visualiser l'état du réseau \textit{en temps réel}.


\subsection{Hypothèses}

\paragraph{} Nous évoluerons dans un environnement distribué constitué de $n$ noeuds de stockage dans lequel on souhaite stocker $m$ objets. C'est un réseau statique : on ne peut pas ajouter ou supprimer de noeuds après création du réseau.

\paragraph{Définition} Un \textit{message} est un envoi d'information d'un noeud vers un autre noeud pour mettre à jour ses données locales ou effectuer des actions particulières (autre que des requêtes, comme par exemple, "donne à tel noeud ta charge actuelle").

\paragraph{Données locales d'un noeud} Un noeud contient les données locales suivantes :
\begin{itemize}
 \item la charge de tous les noeuds du réseau
 \item la popularité de chaque objet stocké sur ce noeud
 \item une file d'attente de message à traiter
 \item la requête en cours de traitement
\end{itemize}

\paragraph{Requêtes} Nous supposerons que les requêtes seront effectuées en un temps fixe.

%\newpage
\section{Ordonnancement des besoins}

\paragraph{} Nous avons dégagé une liste de besoins fonctionnels et non-fonctionnels. 
Pour mieux les comparer, nous les avons ordonnés en fonction de leur priorité et de leur criticité.


\subsection{Priorité}

\paragraph{} La priorité est un indicateur de l'ordre dans lequel nous devrons implémenter les fonctionnalités afin de satisfaire les besoins du client.

\paragraph{}
\begin{tabular}{| l | M{3cm} | M{9cm} |}
    \hline
    Valeur & Signification & Description \tabularnewline
    \hline
    1 & Priorité haute & A implémenter en premier. \tabularnewline
    \hline
    2 & Priorité moyenne & A implémenter.  \tabularnewline
    \hline
    3 & Priorité faible & A implémenter (en fonction du temps restant).  \tabularnewline
    \hline
 \end{tabular}
 

\subsection{Criticité}

\paragraph{} Le niveau de criticité d'un besoin est un indicateur de l'impact qu'aura la non-implémentation de ce besoin sur le bon fonctionnement de l'application.

\paragraph{}
\begin{tabular}{| l | M{3cm} | M{9cm} |}
    \hline
    Valeur & Signification & Description \tabularnewline
    \hline
    1 & Criticité extrême & L'application ne sera pas utilisable par le client. \tabularnewline
    \hline
    2 & Criticité haute & L'application est utilisable par le client. En revanche, certaines fonctionnalités de l'application ne seront pas utilisables. \tabularnewline
    \hline
    3 & Criticité moyenne & L'application est utilisable par le client. En revanche, certaines fonctionnalités de l'application n'améneront pas au résultat attendu.  \tabularnewline
    \hline
    4 & Criticité faible & L'application peut fonctionner sans l'ajout de ces fonctionnalités  \tabularnewline
    \hline
 \end{tabular}



\newpage

\section{Besoins fonctionnels}

\subsection{Gestion d'un réseau}

\paragraph{} Un réseau est un ensemble de noeuds qui sont reliés entre eux (par exemple par Internet) et qui communiquent ensemble afin de traiter toutes les requêtes reçues.


\subsubsection{Gestion des noeuds}

\paragraph{Création d'un noeud} \besoin{1}{1} Il est possible de séparer ce besoin en plusieurs sous-besoins :
 \begin{itemize}
 	\item créer un noeud dans l'environnement
 	\item initialiser les données locales d'un noeud
 \end{itemize}


\paragraph{Mise à jour des données locales} \besoin{1}{1} Afin de connaître l'état du réseau de manière précise, les données locales doivent être mise à jour à chaque action,
c'est à dire lors du traitement d'un message dans la file d'attente.

\paragraph{Mise à jour des données locales} \besoin{1}{1} Afin de connaître l'état du réseau de manière précise, les données locales doivent être mise à jour à chaque action. Une mise à jour a donc lieu lors du traitement d'un message dans la file d'attente.


\paragraph{Communication des données locales} \besoin{1}{1} Un noeud doit être capable de communiquer ses données locales à d'autres noeuds du réseau.

\paragraph{Récupération de l'état du réseau} \besoin{1}{1} L'application doit permettre la description de l'état du réseau.
On souhaite connaître : 
\begin{itemize}
 \item le nombre de requêtes en attente
 \item la popularité des objets
\end{itemize}

\subsubsection{Réplication d'un objet}

\paragraph{} Il s'agit de copier un objet sur un autre noeud. Il est possible de définir le nombre de copie d'un objet au sein d'un ensemble de noeud, appelé \textit{data center}. 
Pour savoir quel noeud stocke l'objet, on utilise une fonction de hachage dans laquelle on fait passer la clé de l'objet (la clé de l'objet est une donnée permettant d'identifier un objet de manière unique).
On obtient ainsi un \textit{token}.


\paragraph{Définition} Une \textit{fonction de hachage} est une fonction mathématique déterministe (c'est à dire, si on lui donne la même entrée, elle renvoie la même sortie). Nous définissons ses entrées et sorties dans le paragraphe suivant.

\paragraph{Définition} Un \textit{token} permet comme une étiquette sur un produit, de désigner une donnée.

\paragraph{} Il s'agit de copier un objet sur un autre noeud. Il est possible de définir le nombre de copie d'un objet au sein d'un ensemble de noeud, appelé \textit{data center}. Pour savoir quel noeud stocke l'objet, on utilise une fonction de hachage dans laquelle on fait passer la clé de l'objet (la clé de l'objet est une donnée permettant d'identifier un objet de manière unique). On obtient ainsi un \textit{token}.


\paragraph{} Tous les noeuds possèdent un intervalle de token dont ils sont responsables. 
On regarde le token de l'objet pour savoir quel noeud va le prendre en charge (voir la figure~\ref{fig:partitionning}).

\paragraph{} Une stratégie de réplication est la méthode qui permet de placer les copies d'un objet dans un data center.
La stratégie consiste à utiliser une fonction de hachage différente pour chaque copie (voir la figure~\ref{fig:multi_hash_partitionning}). 
Le numéro de la copie définit la fonction à utiliser. 
Ainsi, sur le schéma, la deuxième copie de tous les objets utilisera la fonction de hachage \texttt{Hash2} pour obtenir un token et placer la copie.

\paragraph{Définition des fonctions de hachage} \besoin{1}{3}
\paragraph{Mise en place de la stratégie de réplication} \besoin{1}{3}


\subsubsection{Popularité d'un objet}

\paragraph{} Les algorithmes à implémenter nécessitent de connaître la popularité d'un objet dans le réseau.

La popularité d'un objet est fonction du nombre de requêtes sur cet objet.
Plus ce nombre de requêtes est grand, plus l'objet est populaire.

La popularité d'un objet est défini par le nombre de requêtes sur cet objet. Plus le nombre de requêtes est grand, plus l'objet est populaire.


\paragraph{Calcul de la popularité} \besoin{1}{1}

\paragraph{Stockage de la popularité} \besoin{1}{1} Chaque noeud stocke la popularité des objets qu'il contient.

\paragraph{Communication de la popularité} \besoin{1}{1} Un noeud stockant des copies d'un objet doit communiquer la popularité de ces derniers au noeud possédant l'objet original.


\subsection{Protocole de réaffectation}

\paragraph{} \besoin{1}{1} Les algorithmes d'équilibrage de charge à implémenter sont \textbf{SLVO} et \textbf{AverageDegree}.


\subsection{Protocole de test}

\paragraph{} La conformité des algorithmes implémentés est assurée par un protocole de test suivant la démarche :

\begin{itemize}
	\item Définir un réseau $R$, un ensemble d'objets $O$ et un ensemble de requêtes $Q$
	\item Faire tourner l'algorithme à la main avec $R$, $O$ et $Q$
	\item Stocker l'état final du réseau
	\item Faire valider ce processus par le client
	\item Exécuter l'algorithme sur ordinateur avec $R$, $O$ et $Q$
	\item Vérifier les résultats constatés avec les résultats attendus
\end{itemize}
	
\paragraph{} S'il y a une différence entre les deux résultats, une vérification par le client peut être envisagée dans le cas de résultats \textit{presque} similaires. 
La notion de similitude est laissée à l'appréciation de l'équipe en charge du projet, lors de la vérification.



\subsection{Requêtes}

\subsubsection{Générations de requêtes}

\paragraph{} Pour tester la validité des algorithmes, l'application devra posséder une fonction de génération de requêtes. Si l'utilisateur ne détient pas de suites de requêtes, il pourra demander à l'application d'en créer pour lui. L'application ne connaissant pas la nature des données ne pourra qu'effectuer un nombre restreint de requêtes différentes. Elle peut par exemple, compter le nombre de données sauvegardées, chercher si une donnée existe réellement, mais ne peut pas en modifier une.


\subsubsection{Importation d'un jeu de requête}

\paragraph{} Pour comparer l'efficacité des algorithmes, il doit être possible d'envoyer sur le réseau une même suite de requêtes, un jeu de requête.

\paragraph{Importation} \besoin{1}{2} L'application doit pouvoir lire un fichier contenant une suite de requête et envoyer ces requêtes sur le réseau.


\subsection{Visualisation des données}

\paragraph{} Afin de suivre l'évolution des charges de chaque noeud lors de l'éxecution des algorithmes, on enregistre les données locales de chaque noeud à chaque modifications de celles-ci.

\subsubsection{Enregistrement des données}

\paragraph{Ecriture dans un fichier} \besoin{1}{2} Lorsque les données locales d'un noeud sont modifiés, on enregistre dans un fichier ces données.
L'écriture est de la forme \texttt{itération de l'algorithme; identifiant du noeud; charge du noeud;}

\paragraph{}

\subsubsection{Affichage des données}

\paragraph{Définition} Un \textit{graphe} est un ensemble de points appelés \textit{sommets}, dont certaines paires sont directement reliées par un (ou plusieurs) lien(s) appelé(s) \textit{arêtes} \cite{graphe}. 

\paragraph{Noeuds} \besoin{3}{2} L'application doit permettre la représentation de chaque noeud par un sommet.

\paragraph{Analyse syntaxique} \besoin{2}{1} Lors de l'éxecution d'un algorithme, la charge de chaque noeud est enregistrée dans un fichier.
Un analyseur syntaxique découpe chaque ligne du fichier pour récuperer le moment auquel a été enregistré l'information (\texttt{itération de l'algorithme}),
le noeud concerné ( \texttt{ identifiant du noeud} ) et la charge de ce noeud à ce moment ( \texttt{charge du noeud} ).

\paragraph{Charge des noeuds}  \besoin{3}{3} A chaque sommet est associé une valeur correspondant à la charge de ce noeud.
Ces données sont récupérées grâce à l'analyseur syntaxique.

\paragraph{Film de l'éxecution} \besoin{3}{3} Cela consiste à afficher la charge des noeuds dans l'ordre chronologique, c'est à dire dans l'ordre des itérations croissant.


\newpage

\section{Besoins non fonctionnels}

\subsection{Cassandra}

\paragraph{} Cassandra est une base de données distribuée.
Nous créons notre environnement de simulation à partir de la dernière version stable de Cassandra.

\paragraph{} Le choix de cette solution nous a été fortement recommandé par le client.
En effet, celui-ci dispose de connaissances sur cette application et pourra donc plus facilement intervenir s'il souhaite faire évoluer le projet en implémentant par exemple de nouveaux algorithmes.



\subsection{Maintenabilité du code}

\paragraph{} Nous ne pensons pas que le projet sera totalement terminé le 8 Avril 2015, date de rendu du code et du mémoire.
Pour cela, nous avons défini quelques normes pour que le projet puisse être repris. (a détailler)

\paragraph{}



\subsection{Gestion d'un réseau}

\subsubsection{Communication entre noeuds}


\paragraph{Algorithme} Le calcul de la popularité nécessite l'implémentation de l'algorithme d'approximation Space-Saving Algorithm \cite{SpaceSaving}.

\paragraph{}Pour connaître l'état du réseau, il faut regrouper les données locales des noeuds.
Nous cherchons donc a récupérer ces données en un temps raisonnable ($O(log(n))$ pour $n$ noeuds).

\paragraph{}Pour cela, nous nous appuierons sur le protocole \texttt{Gossip} \cite{gossip}.
Périodiquement, chaque noeud choisi $n$ noeuds aléatoirement dont un noeud \textit{seed}, noeud en mesure d'avoir une connaissance globale du système, et il communique à ces noeuds ses statistiques (valeur de sa charge, objets les plus populaires...).

\paragraph{}Ainsi, la connaissance globale du système se fait, dans la théorie, en $O(log(n))$.


\subsubsection{Taille des données}

\paragraph{} La taille de chaque donnée est laissée à l'appréciation de l'équipe. 
Néanmoins, celle-ci doit être suffisamment importante, afin de permettre de créer des requêtes qui "stressent" le système pour avoir des résultats cohérents (sur la base de l'hypothèse : chaque requête prend un même temps à être traitée).


\subsection{Visualisation des données}

\subsubsection{Etat du réseau}

\paragraph{} La vue permet de montrer l'état du réseau.

\paragraph{} Le réseau est représenté par un graphe, les machines par des noeuds. 
Pour chaque machine, les données affichées sont la charge ainsi que le contenu de la file d'attente.

\subsubsection{Actualisation de la vue}

\paragraph{} L'état du réseau doit être visible en temps réel.

\paragraph{} La vue peut donc être actualisée toutes les $0.5$ secondes. 
Un délai plus faible risquerait de ralentir le système, étant donné que l'obtention des données nécessaires à la visualisation se fait sur la même base de données que celle qui est testée.


\newpage
\section{Répartitions des tâches}

\subsection{Diagramme de Gantt}

\rotatebox{270}{
\begin{ganttchart}[
hgrid,
vgrid={*6{blue,dashed}, *1{red}},
inline = true,
x unit=4.5mm,
time slot format=isodate,
milestone inline label node/.append style={left=3mm},
y unit chart = 0.65cm,
bar top shift= 0.1,
bar height = 0.8
]{2015-02-16}{2015-04-08}
\gantttitlecalendar{month=name, day} \\

\ganttgroup{Gestion du réseau (A)}{2015-02-16}{2015-03-06} \\
\ganttbar[name=a1]{A1}{2015-02-16}{2015-02-23} \\ %Création noeuds via script ?
\ganttbar[name=a2]{A2}{2015-02-16}{2015-02-23} \\ %Implémentation et initialisations données noeuds
\ganttbar[name=a3]{A3}{2015-02-24}{2015-03-28} \\ %Communication entre noeuds
\ganttbar[name=a4]{A4}{2015-02-27}{2015-03-02} \\ %Gestion des replicats
\ganttbar{AT}{2015-03-03}{2015-03-06} \\ %Tests

\ganttgroup{Simulation de requête (B)}{2015-03-01}{2015-03-08} \\
\ganttbar{B1}{2015-03-01}{2015-03-06} \\ %Création d'un interpreteur pour lire un fichier jeu de requête
\ganttbar{BT}{2015-03-07}{2015-03-08} \\ %Tests

\ganttgroup{Gestion de la popularité (C)}{2015-03-07}{2015-03-16} \\
\ganttbar{C1}{2015-03-07}{2015-03-10} \\ %Calcul de la popularité pour un objet
\ganttbar{C2}{2015-03-07}{2015-03-10} \\ %Implémentation de SpaceSaving
\ganttbar{C3}{2015-03-10}{2015-03-14} \\ %Amélioration protocole Gossip
\ganttbar{CT}{2015-03-15}{2015-03-16} \\ %Tests


\ganttgroup{Equilibrage des charges (D)}{2015-03-15}{2015-03-29} \\
\ganttbar{D1}{2015-03-15}{2015-03-24} \\ %Algorithme SLVO
\ganttbar{D2}{2015-03-15}{2015-03-24} \\ %Algorithme AvergeDegree
\ganttbar{DT}{2015-03-25}{2015-03-29} \\ %Tests 

\ganttgroup{Visualisation de données (E)}{2015-03-29}{2015-04-08} \\
\ganttbar{E1}{2015-03-29}{2015-04-01} \\ %Prise en main tulip
\ganttbar{E2}{2015-04-01}{2015-04-04} \\ %Affichage réseau
\ganttbar{E3}{2015-04-04}{2015-04-06} \\ %AFfichage informations (charges)

\ganttbar{T}{2015-04-06}{2015-04-07} \\

\ganttmilestone{Livrable final}{2015-04-08} 

\end{ganttchart}
}



\newpage

\subsection{Affectation des tâches}

\paragraph{}
\begin{tabular}{| l | M{5cm} | M{3cm} | M{5.5cm} |}
    \hline
    Fct & Description & Développeur(s) & Commentaire \tabularnewline
    \hline
    A1 & Création des noeuds &  &   \tabularnewline
    \hline
    A2 & Données locales des noeuds &  & Initialisation et implémentation \tabularnewline
    \hline
    A3 & Communication entre noeuds &  &  \tabularnewline
    \hline
    A4 & Gestion des replicats &  &  \tabularnewline
    \hline
    AT & Tests groupe A &  &  Vérification, tests, mémoire \tabularnewline
    \hline 
    \hline
    B1 & Générateur de requêtes &  & A détailler  \tabularnewline
    \hline
    BT & Tests groupe B &  &  Vérification, tests, mémoire \tabularnewline
    \hline
    \hline
    C1 & Popularité objet sur noeud &  &  \tabularnewline
    \hline
    C2 & Space-Saving Algorithm &  &  \tabularnewline
    \hline
    C3 & Popularité d'un objet &  &  \tabularnewline
    \hline
    CT & Tests groupe C &  & Vérification, tests, mémoire \tabularnewline
    \hline
    \hline
    D1 & Implémentation SLVO &  &  \tabularnewline
    \hline
    D2 & Implémentation AverageDegree &  &  \tabularnewline
    \hline
    DT & Tests groupe D &  &  Avec client \tabularnewline
    \hline
    \hline
    E1 & Prise en main Tulip &  &  \tabularnewline
    \hline
    E2 & Représentation réseau &  &  \tabularnewline
    \hline
    E3 & Représentation données &  &  \tabularnewline
    \hline
    \hline
    T & Tests finaux &  &  Vérification, tests, mémoire \tabularnewline
    \hline
 \end{tabular}
 


\newpage

\section{Livrables}

\subsection{Livrable ``final''}

\paragraph{} Il devra être remis le 8 Avril 2015.


\begin{figure}[p]
	\centering
		% Graphic for TeX using PGF
% Title: C:\Users\Kéké\Pictures\distribued_database.dia
% Creator: Dia v0.97.2
% CreationDate: Thu Feb 05 10:46:09 2015
% For: Kéké
% \usepackage{tikz}
% The following commands are not supported in PSTricks at present
% We define them conditionally, so when they are implemented,
% this pgf file will use them.
\ifx\du\undefined
  \newlength{\du}
\fi
\setlength{\du}{15\unitlength}
\begin{tikzpicture}
\pgftransformxscale{1.000000}
\pgftransformyscale{-1.000000}
\definecolor{dialinecolor}{rgb}{0.000000, 0.000000, 0.000000}
\pgfsetstrokecolor{dialinecolor}
\definecolor{dialinecolor}{rgb}{1.000000, 1.000000, 1.000000}
\pgfsetfillcolor{dialinecolor}
\pgfsetlinewidth{0.100000\du}
\pgfsetdash{}{0pt}
\pgfsetdash{}{0pt}
\pgfsetbuttcap
\pgfsetmiterjoin
\pgfsetlinewidth{0.100000\du}
\pgfsetbuttcap
\pgfsetmiterjoin
\pgfsetdash{}{0pt}
\definecolor{dialinecolor}{rgb}{1.000000, 1.000000, 1.000000}
\pgfsetfillcolor{dialinecolor}
\pgfpathellipse{\pgfpoint{14.237500\du}{16.387500\du}}{\pgfpoint{10.637500\du}{0\du}}{\pgfpoint{0\du}{10.637500\du}}
\pgfusepath{fill}
\definecolor{dialinecolor}{rgb}{0.000000, 0.000000, 0.000000}
\pgfsetstrokecolor{dialinecolor}
\pgfpathellipse{\pgfpoint{14.237500\du}{16.387500\du}}{\pgfpoint{10.637500\du}{0\du}}{\pgfpoint{0\du}{10.637500\du}}
\pgfusepath{stroke}
\pgfsetbuttcap
\pgfsetmiterjoin
\pgfsetdash{}{0pt}
\definecolor{dialinecolor}{rgb}{0.000000, 0.000000, 0.000000}
\pgfsetstrokecolor{dialinecolor}
\pgfpathellipse{\pgfpoint{14.237500\du}{16.387500\du}}{\pgfpoint{10.637500\du}{0\du}}{\pgfpoint{0\du}{10.637500\du}}
\pgfusepath{stroke}
\pgfsetlinewidth{0.100000\du}
\pgfsetdash{}{0pt}
\pgfsetdash{}{0pt}
\pgfsetbuttcap
\pgfsetmiterjoin
\pgfsetlinewidth{0.100000\du}
\pgfsetbuttcap
\pgfsetmiterjoin
\pgfsetdash{}{0pt}
\definecolor{dialinecolor}{rgb}{0.960784, 0.960784, 0.960784}
\pgfsetfillcolor{dialinecolor}
\pgfpathellipse{\pgfpoint{10.362500\du}{11.612500\du}}{\pgfpoint{3.562500\du}{0\du}}{\pgfpoint{0\du}{3.562500\du}}
\pgfusepath{fill}
\definecolor{dialinecolor}{rgb}{0.000000, 0.000000, 0.000000}
\pgfsetstrokecolor{dialinecolor}
\pgfpathellipse{\pgfpoint{10.362500\du}{11.612500\du}}{\pgfpoint{3.562500\du}{0\du}}{\pgfpoint{0\du}{3.562500\du}}
\pgfusepath{stroke}
\pgfsetbuttcap
\pgfsetmiterjoin
\pgfsetdash{}{0pt}
\definecolor{dialinecolor}{rgb}{0.000000, 0.000000, 0.000000}
\pgfsetstrokecolor{dialinecolor}
\pgfpathellipse{\pgfpoint{10.362500\du}{11.612500\du}}{\pgfpoint{3.562500\du}{0\du}}{\pgfpoint{0\du}{3.562500\du}}
\pgfusepath{stroke}
\pgfsetlinewidth{0.000000\du}
\pgfsetdash{}{0pt}
\pgfsetdash{}{0pt}
\pgfsetbuttcap
\pgfsetmiterjoin
\pgfsetlinewidth{0.000000\du}
\pgfsetbuttcap
\pgfsetmiterjoin
\pgfsetdash{}{0pt}
\definecolor{dialinecolor}{rgb}{0.247059, 0.317647, 0.709804}
\pgfsetfillcolor{dialinecolor}
\pgfpathellipse{\pgfpoint{10.225000\du}{8.275000\du}}{\pgfpoint{1.025000\du}{0\du}}{\pgfpoint{0\du}{1.025000\du}}
\pgfusepath{fill}
\definecolor{dialinecolor}{rgb}{0.000000, 0.000000, 0.000000}
\pgfsetstrokecolor{dialinecolor}
\pgfpathellipse{\pgfpoint{10.225000\du}{8.275000\du}}{\pgfpoint{1.025000\du}{0\du}}{\pgfpoint{0\du}{1.025000\du}}
\pgfusepath{stroke}
\pgfsetbuttcap
\pgfsetmiterjoin
\pgfsetdash{}{0pt}
\definecolor{dialinecolor}{rgb}{0.000000, 0.000000, 0.000000}
\pgfsetstrokecolor{dialinecolor}
\pgfpathellipse{\pgfpoint{10.225000\du}{8.275000\du}}{\pgfpoint{1.025000\du}{0\du}}{\pgfpoint{0\du}{1.025000\du}}
\pgfusepath{stroke}
\pgfsetlinewidth{0.000000\du}
\pgfsetdash{}{0pt}
\pgfsetdash{}{0pt}
\pgfsetbuttcap
\pgfsetmiterjoin
\pgfsetlinewidth{0.000000\du}
\pgfsetbuttcap
\pgfsetmiterjoin
\pgfsetdash{}{0pt}
\definecolor{dialinecolor}{rgb}{0.247059, 0.317647, 0.709804}
\pgfsetfillcolor{dialinecolor}
\pgfpathellipse{\pgfpoint{13.770000\du}{11.630000\du}}{\pgfpoint{1.025000\du}{0\du}}{\pgfpoint{0\du}{1.025000\du}}
\pgfusepath{fill}
\definecolor{dialinecolor}{rgb}{0.000000, 0.000000, 0.000000}
\pgfsetstrokecolor{dialinecolor}
\pgfpathellipse{\pgfpoint{13.770000\du}{11.630000\du}}{\pgfpoint{1.025000\du}{0\du}}{\pgfpoint{0\du}{1.025000\du}}
\pgfusepath{stroke}
\pgfsetbuttcap
\pgfsetmiterjoin
\pgfsetdash{}{0pt}
\definecolor{dialinecolor}{rgb}{0.000000, 0.000000, 0.000000}
\pgfsetstrokecolor{dialinecolor}
\pgfpathellipse{\pgfpoint{13.770000\du}{11.630000\du}}{\pgfpoint{1.025000\du}{0\du}}{\pgfpoint{0\du}{1.025000\du}}
\pgfusepath{stroke}
\pgfsetlinewidth{0.000000\du}
\pgfsetdash{}{0pt}
\pgfsetdash{}{0pt}
\pgfsetbuttcap
\pgfsetmiterjoin
\pgfsetlinewidth{0.000000\du}
\pgfsetbuttcap
\pgfsetmiterjoin
\pgfsetdash{}{0pt}
\definecolor{dialinecolor}{rgb}{0.247059, 0.317647, 0.709804}
\pgfsetfillcolor{dialinecolor}
\pgfpathellipse{\pgfpoint{10.265000\du}{14.985000\du}}{\pgfpoint{1.025000\du}{0\du}}{\pgfpoint{0\du}{1.025000\du}}
\pgfusepath{fill}
\definecolor{dialinecolor}{rgb}{0.000000, 0.000000, 0.000000}
\pgfsetstrokecolor{dialinecolor}
\pgfpathellipse{\pgfpoint{10.265000\du}{14.985000\du}}{\pgfpoint{1.025000\du}{0\du}}{\pgfpoint{0\du}{1.025000\du}}
\pgfusepath{stroke}
\pgfsetbuttcap
\pgfsetmiterjoin
\pgfsetdash{}{0pt}
\definecolor{dialinecolor}{rgb}{0.000000, 0.000000, 0.000000}
\pgfsetstrokecolor{dialinecolor}
\pgfpathellipse{\pgfpoint{10.265000\du}{14.985000\du}}{\pgfpoint{1.025000\du}{0\du}}{\pgfpoint{0\du}{1.025000\du}}
\pgfusepath{stroke}
\pgfsetlinewidth{0.000000\du}
\pgfsetdash{}{0pt}
\pgfsetdash{}{0pt}
\pgfsetbuttcap
\pgfsetmiterjoin
\pgfsetlinewidth{0.000000\du}
\pgfsetbuttcap
\pgfsetmiterjoin
\pgfsetdash{}{0pt}
\definecolor{dialinecolor}{rgb}{0.247059, 0.317647, 0.709804}
\pgfsetfillcolor{dialinecolor}
\pgfpathellipse{\pgfpoint{7.010000\du}{11.640000\du}}{\pgfpoint{1.025000\du}{0\du}}{\pgfpoint{0\du}{1.025000\du}}
\pgfusepath{fill}
\definecolor{dialinecolor}{rgb}{0.000000, 0.000000, 0.000000}
\pgfsetstrokecolor{dialinecolor}
\pgfpathellipse{\pgfpoint{7.010000\du}{11.640000\du}}{\pgfpoint{1.025000\du}{0\du}}{\pgfpoint{0\du}{1.025000\du}}
\pgfusepath{stroke}
\pgfsetbuttcap
\pgfsetmiterjoin
\pgfsetdash{}{0pt}
\definecolor{dialinecolor}{rgb}{0.000000, 0.000000, 0.000000}
\pgfsetstrokecolor{dialinecolor}
\pgfpathellipse{\pgfpoint{7.010000\du}{11.640000\du}}{\pgfpoint{1.025000\du}{0\du}}{\pgfpoint{0\du}{1.025000\du}}
\pgfusepath{stroke}
\pgfsetlinewidth{0.100000\du}
\pgfsetdash{}{0pt}
\pgfsetdash{}{0pt}
\pgfsetbuttcap
\pgfsetmiterjoin
\pgfsetlinewidth{0.100000\du}
\pgfsetbuttcap
\pgfsetmiterjoin
\pgfsetdash{}{0pt}
\definecolor{dialinecolor}{rgb}{0.960784, 0.960784, 0.960784}
\pgfsetfillcolor{dialinecolor}
\pgfpathellipse{\pgfpoint{18.522500\du}{13.917500\du}}{\pgfpoint{3.562500\du}{0\du}}{\pgfpoint{0\du}{3.562500\du}}
\pgfusepath{fill}
\definecolor{dialinecolor}{rgb}{0.000000, 0.000000, 0.000000}
\pgfsetstrokecolor{dialinecolor}
\pgfpathellipse{\pgfpoint{18.522500\du}{13.917500\du}}{\pgfpoint{3.562500\du}{0\du}}{\pgfpoint{0\du}{3.562500\du}}
\pgfusepath{stroke}
\pgfsetbuttcap
\pgfsetmiterjoin
\pgfsetdash{}{0pt}
\definecolor{dialinecolor}{rgb}{0.000000, 0.000000, 0.000000}
\pgfsetstrokecolor{dialinecolor}
\pgfpathellipse{\pgfpoint{18.522500\du}{13.917500\du}}{\pgfpoint{3.562500\du}{0\du}}{\pgfpoint{0\du}{3.562500\du}}
\pgfusepath{stroke}
\pgfsetlinewidth{0.000000\du}
\pgfsetdash{}{0pt}
\pgfsetdash{}{0pt}
\pgfsetbuttcap
\pgfsetmiterjoin
\pgfsetlinewidth{0.000000\du}
\pgfsetbuttcap
\pgfsetmiterjoin
\pgfsetdash{}{0pt}
\definecolor{dialinecolor}{rgb}{1.000000, 0.341176, 0.133333}
\pgfsetfillcolor{dialinecolor}
\pgfpathellipse{\pgfpoint{18.385000\du}{10.580000\du}}{\pgfpoint{1.025000\du}{0\du}}{\pgfpoint{0\du}{1.025000\du}}
\pgfusepath{fill}
\definecolor{dialinecolor}{rgb}{0.000000, 0.000000, 0.000000}
\pgfsetstrokecolor{dialinecolor}
\pgfpathellipse{\pgfpoint{18.385000\du}{10.580000\du}}{\pgfpoint{1.025000\du}{0\du}}{\pgfpoint{0\du}{1.025000\du}}
\pgfusepath{stroke}
\pgfsetbuttcap
\pgfsetmiterjoin
\pgfsetdash{}{0pt}
\definecolor{dialinecolor}{rgb}{0.000000, 0.000000, 0.000000}
\pgfsetstrokecolor{dialinecolor}
\pgfpathellipse{\pgfpoint{18.385000\du}{10.580000\du}}{\pgfpoint{1.025000\du}{0\du}}{\pgfpoint{0\du}{1.025000\du}}
\pgfusepath{stroke}
\pgfsetlinewidth{0.000000\du}
\pgfsetdash{}{0pt}
\pgfsetdash{}{0pt}
\pgfsetbuttcap
\pgfsetmiterjoin
\pgfsetlinewidth{0.000000\du}
\pgfsetbuttcap
\pgfsetmiterjoin
\pgfsetdash{}{0pt}
\definecolor{dialinecolor}{rgb}{1.000000, 0.341176, 0.133333}
\pgfsetfillcolor{dialinecolor}
\pgfpathellipse{\pgfpoint{21.930000\du}{13.935000\du}}{\pgfpoint{1.025000\du}{0\du}}{\pgfpoint{0\du}{1.025000\du}}
\pgfusepath{fill}
\definecolor{dialinecolor}{rgb}{0.000000, 0.000000, 0.000000}
\pgfsetstrokecolor{dialinecolor}
\pgfpathellipse{\pgfpoint{21.930000\du}{13.935000\du}}{\pgfpoint{1.025000\du}{0\du}}{\pgfpoint{0\du}{1.025000\du}}
\pgfusepath{stroke}
\pgfsetbuttcap
\pgfsetmiterjoin
\pgfsetdash{}{0pt}
\definecolor{dialinecolor}{rgb}{0.000000, 0.000000, 0.000000}
\pgfsetstrokecolor{dialinecolor}
\pgfpathellipse{\pgfpoint{21.930000\du}{13.935000\du}}{\pgfpoint{1.025000\du}{0\du}}{\pgfpoint{0\du}{1.025000\du}}
\pgfusepath{stroke}
\pgfsetlinewidth{0.000000\du}
\pgfsetdash{}{0pt}
\pgfsetdash{}{0pt}
\pgfsetbuttcap
\pgfsetmiterjoin
\pgfsetlinewidth{0.000000\du}
\pgfsetbuttcap
\pgfsetmiterjoin
\pgfsetdash{}{0pt}
\definecolor{dialinecolor}{rgb}{1.000000, 0.341176, 0.133333}
\pgfsetfillcolor{dialinecolor}
\pgfpathellipse{\pgfpoint{18.425000\du}{17.290000\du}}{\pgfpoint{1.025000\du}{0\du}}{\pgfpoint{0\du}{1.025000\du}}
\pgfusepath{fill}
\definecolor{dialinecolor}{rgb}{0.000000, 0.000000, 0.000000}
\pgfsetstrokecolor{dialinecolor}
\pgfpathellipse{\pgfpoint{18.425000\du}{17.290000\du}}{\pgfpoint{1.025000\du}{0\du}}{\pgfpoint{0\du}{1.025000\du}}
\pgfusepath{stroke}
\pgfsetbuttcap
\pgfsetmiterjoin
\pgfsetdash{}{0pt}
\definecolor{dialinecolor}{rgb}{0.000000, 0.000000, 0.000000}
\pgfsetstrokecolor{dialinecolor}
\pgfpathellipse{\pgfpoint{18.425000\du}{17.290000\du}}{\pgfpoint{1.025000\du}{0\du}}{\pgfpoint{0\du}{1.025000\du}}
\pgfusepath{stroke}
\pgfsetlinewidth{0.000000\du}
\pgfsetdash{}{0pt}
\pgfsetdash{}{0pt}
\pgfsetbuttcap
\pgfsetmiterjoin
\pgfsetlinewidth{0.000000\du}
\pgfsetbuttcap
\pgfsetmiterjoin
\pgfsetdash{}{0pt}
\definecolor{dialinecolor}{rgb}{1.000000, 0.341176, 0.133333}
\pgfsetfillcolor{dialinecolor}
\pgfpathellipse{\pgfpoint{15.170000\du}{13.945000\du}}{\pgfpoint{1.025000\du}{0\du}}{\pgfpoint{0\du}{1.025000\du}}
\pgfusepath{fill}
\definecolor{dialinecolor}{rgb}{0.000000, 0.000000, 0.000000}
\pgfsetstrokecolor{dialinecolor}
\pgfpathellipse{\pgfpoint{15.170000\du}{13.945000\du}}{\pgfpoint{1.025000\du}{0\du}}{\pgfpoint{0\du}{1.025000\du}}
\pgfusepath{stroke}
\pgfsetbuttcap
\pgfsetmiterjoin
\pgfsetdash{}{0pt}
\definecolor{dialinecolor}{rgb}{0.000000, 0.000000, 0.000000}
\pgfsetstrokecolor{dialinecolor}
\pgfpathellipse{\pgfpoint{15.170000\du}{13.945000\du}}{\pgfpoint{1.025000\du}{0\du}}{\pgfpoint{0\du}{1.025000\du}}
\pgfusepath{stroke}
\pgfsetlinewidth{0.100000\du}
\pgfsetdash{}{0pt}
\pgfsetdash{}{0pt}
\pgfsetbuttcap
\pgfsetmiterjoin
\pgfsetlinewidth{0.100000\du}
\pgfsetbuttcap
\pgfsetmiterjoin
\pgfsetdash{}{0pt}
\definecolor{dialinecolor}{rgb}{0.960784, 0.960784, 0.960784}
\pgfsetfillcolor{dialinecolor}
\pgfpathellipse{\pgfpoint{12.917500\du}{21.022500\du}}{\pgfpoint{3.562500\du}{0\du}}{\pgfpoint{0\du}{3.562500\du}}
\pgfusepath{fill}
\definecolor{dialinecolor}{rgb}{0.000000, 0.000000, 0.000000}
\pgfsetstrokecolor{dialinecolor}
\pgfpathellipse{\pgfpoint{12.917500\du}{21.022500\du}}{\pgfpoint{3.562500\du}{0\du}}{\pgfpoint{0\du}{3.562500\du}}
\pgfusepath{stroke}
\pgfsetbuttcap
\pgfsetmiterjoin
\pgfsetdash{}{0pt}
\definecolor{dialinecolor}{rgb}{0.000000, 0.000000, 0.000000}
\pgfsetstrokecolor{dialinecolor}
\pgfpathellipse{\pgfpoint{12.917500\du}{21.022500\du}}{\pgfpoint{3.562500\du}{0\du}}{\pgfpoint{0\du}{3.562500\du}}
\pgfusepath{stroke}
\pgfsetlinewidth{0.000000\du}
\pgfsetdash{}{0pt}
\pgfsetdash{}{0pt}
\pgfsetbuttcap
\pgfsetmiterjoin
\pgfsetlinewidth{0.000000\du}
\pgfsetbuttcap
\pgfsetmiterjoin
\pgfsetdash{}{0pt}
\definecolor{dialinecolor}{rgb}{0.298039, 0.686275, 0.313726}
\pgfsetfillcolor{dialinecolor}
\pgfpathellipse{\pgfpoint{12.780000\du}{17.685000\du}}{\pgfpoint{1.025000\du}{0\du}}{\pgfpoint{0\du}{1.025000\du}}
\pgfusepath{fill}
\definecolor{dialinecolor}{rgb}{0.000000, 0.000000, 0.000000}
\pgfsetstrokecolor{dialinecolor}
\pgfpathellipse{\pgfpoint{12.780000\du}{17.685000\du}}{\pgfpoint{1.025000\du}{0\du}}{\pgfpoint{0\du}{1.025000\du}}
\pgfusepath{stroke}
\pgfsetbuttcap
\pgfsetmiterjoin
\pgfsetdash{}{0pt}
\definecolor{dialinecolor}{rgb}{0.000000, 0.000000, 0.000000}
\pgfsetstrokecolor{dialinecolor}
\pgfpathellipse{\pgfpoint{12.780000\du}{17.685000\du}}{\pgfpoint{1.025000\du}{0\du}}{\pgfpoint{0\du}{1.025000\du}}
\pgfusepath{stroke}
\pgfsetlinewidth{0.000000\du}
\pgfsetdash{}{0pt}
\pgfsetdash{}{0pt}
\pgfsetbuttcap
\pgfsetmiterjoin
\pgfsetlinewidth{0.000000\du}
\pgfsetbuttcap
\pgfsetmiterjoin
\pgfsetdash{}{0pt}
\definecolor{dialinecolor}{rgb}{0.298039, 0.686275, 0.313726}
\pgfsetfillcolor{dialinecolor}
\pgfpathellipse{\pgfpoint{16.325000\du}{21.040000\du}}{\pgfpoint{1.025000\du}{0\du}}{\pgfpoint{0\du}{1.025000\du}}
\pgfusepath{fill}
\definecolor{dialinecolor}{rgb}{0.000000, 0.000000, 0.000000}
\pgfsetstrokecolor{dialinecolor}
\pgfpathellipse{\pgfpoint{16.325000\du}{21.040000\du}}{\pgfpoint{1.025000\du}{0\du}}{\pgfpoint{0\du}{1.025000\du}}
\pgfusepath{stroke}
\pgfsetbuttcap
\pgfsetmiterjoin
\pgfsetdash{}{0pt}
\definecolor{dialinecolor}{rgb}{0.000000, 0.000000, 0.000000}
\pgfsetstrokecolor{dialinecolor}
\pgfpathellipse{\pgfpoint{16.325000\du}{21.040000\du}}{\pgfpoint{1.025000\du}{0\du}}{\pgfpoint{0\du}{1.025000\du}}
\pgfusepath{stroke}
\pgfsetlinewidth{0.000000\du}
\pgfsetdash{}{0pt}
\pgfsetdash{}{0pt}
\pgfsetbuttcap
\pgfsetmiterjoin
\pgfsetlinewidth{0.000000\du}
\pgfsetbuttcap
\pgfsetmiterjoin
\pgfsetdash{}{0pt}
\definecolor{dialinecolor}{rgb}{0.298039, 0.686275, 0.313726}
\pgfsetfillcolor{dialinecolor}
\pgfpathellipse{\pgfpoint{12.820000\du}{24.395000\du}}{\pgfpoint{1.025000\du}{0\du}}{\pgfpoint{0\du}{1.025000\du}}
\pgfusepath{fill}
\definecolor{dialinecolor}{rgb}{0.000000, 0.000000, 0.000000}
\pgfsetstrokecolor{dialinecolor}
\pgfpathellipse{\pgfpoint{12.820000\du}{24.395000\du}}{\pgfpoint{1.025000\du}{0\du}}{\pgfpoint{0\du}{1.025000\du}}
\pgfusepath{stroke}
\pgfsetbuttcap
\pgfsetmiterjoin
\pgfsetdash{}{0pt}
\definecolor{dialinecolor}{rgb}{0.000000, 0.000000, 0.000000}
\pgfsetstrokecolor{dialinecolor}
\pgfpathellipse{\pgfpoint{12.820000\du}{24.395000\du}}{\pgfpoint{1.025000\du}{0\du}}{\pgfpoint{0\du}{1.025000\du}}
\pgfusepath{stroke}
\pgfsetlinewidth{0.000000\du}
\pgfsetdash{}{0pt}
\pgfsetdash{}{0pt}
\pgfsetbuttcap
\pgfsetmiterjoin
\pgfsetlinewidth{0.000000\du}
\pgfsetbuttcap
\pgfsetmiterjoin
\pgfsetdash{}{0pt}
\definecolor{dialinecolor}{rgb}{0.298039, 0.686275, 0.313726}
\pgfsetfillcolor{dialinecolor}
\pgfpathellipse{\pgfpoint{9.565000\du}{21.050000\du}}{\pgfpoint{1.025000\du}{0\du}}{\pgfpoint{0\du}{1.025000\du}}
\pgfusepath{fill}
\definecolor{dialinecolor}{rgb}{0.000000, 0.000000, 0.000000}
\pgfsetstrokecolor{dialinecolor}
\pgfpathellipse{\pgfpoint{9.565000\du}{21.050000\du}}{\pgfpoint{1.025000\du}{0\du}}{\pgfpoint{0\du}{1.025000\du}}
\pgfusepath{stroke}
\pgfsetbuttcap
\pgfsetmiterjoin
\pgfsetdash{}{0pt}
\definecolor{dialinecolor}{rgb}{0.000000, 0.000000, 0.000000}
\pgfsetstrokecolor{dialinecolor}
\pgfpathellipse{\pgfpoint{9.565000\du}{21.050000\du}}{\pgfpoint{1.025000\du}{0\du}}{\pgfpoint{0\du}{1.025000\du}}
\pgfusepath{stroke}
% setfont left to latex
\definecolor{dialinecolor}{rgb}{0.000000, 0.000000, 0.000000}
\pgfsetstrokecolor{dialinecolor}
\node at (10.362500\du,11.835000\du){Data Center 1};
% setfont left to latex
\definecolor{dialinecolor}{rgb}{0.000000, 0.000000, 0.000000}
\pgfsetstrokecolor{dialinecolor}
\node at (18.522500\du,14.140000\du){Data Center 2};
% setfont left to latex
\definecolor{dialinecolor}{rgb}{0.000000, 0.000000, 0.000000}
\pgfsetstrokecolor{dialinecolor}
\node at (12.917500\du,21.245000\du){Data Center 3};
% setfont left to latex
\definecolor{dialinecolor}{rgb}{0.000000, 0.000000, 0.000000}
\pgfsetstrokecolor{dialinecolor}
\node[anchor=west] at (4.687500\du,17.937500\du){Cluster};
\end{tikzpicture}

	\caption{Visualisation d'une base de données distribuées sous forme de cluster possédant quatre data center\label{fig:distributed_database}}
\end{figure}

\begin{figure}[p]
	\centering
		% Graphic for TeX using PGF
% Title: C:\Users\Kéké\Pictures\Diagramme2.dia
% Creator: Dia v0.97.2
% CreationDate: Wed Feb 04 15:16:29 2015
% For: Kéké
% \usepackage{tikz}
% The following commands are not supported in PSTricks at present
% We define them conditionally, so when they are implemented,
% this pgf file will use them.
\ifx\du\undefined
  \newlength{\du}
\fi
\setlength{\du}{15\unitlength}
\begin{tikzpicture}
\pgftransformxscale{1.000000}
\pgftransformyscale{-1.000000}
\definecolor{dialinecolor}{rgb}{0.000000, 0.000000, 0.000000}
\pgfsetstrokecolor{dialinecolor}
\definecolor{dialinecolor}{rgb}{1.000000, 1.000000, 1.000000}
\pgfsetfillcolor{dialinecolor}
\pgfsetlinewidth{0.100000\du}
\pgfsetdash{}{0pt}
\pgfsetdash{}{0pt}
\pgfsetbuttcap
\pgfsetmiterjoin
\pgfsetlinewidth{0.100000\du}
\pgfsetbuttcap
\pgfsetmiterjoin
\pgfsetdash{}{0pt}
\definecolor{dialinecolor}{rgb}{0.960784, 0.960784, 0.960784}
\pgfsetfillcolor{dialinecolor}
\pgfpathellipse{\pgfpoint{12.662500\du}{-1.262500\du}}{\pgfpoint{7.462500\du}{0\du}}{\pgfpoint{0\du}{7.462500\du}}
\pgfusepath{fill}
\definecolor{dialinecolor}{rgb}{0.000000, 0.000000, 0.000000}
\pgfsetstrokecolor{dialinecolor}
\pgfpathellipse{\pgfpoint{12.662500\du}{-1.262500\du}}{\pgfpoint{7.462500\du}{0\du}}{\pgfpoint{0\du}{7.462500\du}}
\pgfusepath{stroke}
\pgfsetbuttcap
\pgfsetmiterjoin
\pgfsetdash{}{0pt}
\definecolor{dialinecolor}{rgb}{0.000000, 0.000000, 0.000000}
\pgfsetstrokecolor{dialinecolor}
\pgfpathellipse{\pgfpoint{12.662500\du}{-1.262500\du}}{\pgfpoint{7.462500\du}{0\du}}{\pgfpoint{0\du}{7.462500\du}}
\pgfusepath{stroke}
\pgfsetlinewidth{0.000000\du}
\pgfsetdash{}{0pt}
\pgfsetdash{}{0pt}
\pgfsetbuttcap
\pgfsetmiterjoin
\pgfsetlinewidth{0.000000\du}
\pgfsetbuttcap
\pgfsetmiterjoin
\pgfsetdash{}{0pt}
\definecolor{dialinecolor}{rgb}{0.247059, 0.317647, 0.709804}
\pgfsetfillcolor{dialinecolor}
\pgfpathellipse{\pgfpoint{12.737500\du}{-8.837500\du}}{\pgfpoint{1.787500\du}{0\du}}{\pgfpoint{0\du}{1.787500\du}}
\pgfusepath{fill}
\definecolor{dialinecolor}{rgb}{0.188235, 0.247059, 0.623529}
\pgfsetstrokecolor{dialinecolor}
\pgfpathellipse{\pgfpoint{12.737500\du}{-8.837500\du}}{\pgfpoint{1.787500\du}{0\du}}{\pgfpoint{0\du}{1.787500\du}}
\pgfusepath{stroke}
\pgfsetbuttcap
\pgfsetmiterjoin
\pgfsetdash{}{0pt}
\definecolor{dialinecolor}{rgb}{0.188235, 0.247059, 0.623529}
\pgfsetstrokecolor{dialinecolor}
\pgfpathellipse{\pgfpoint{12.737500\du}{-8.837500\du}}{\pgfpoint{1.787500\du}{0\du}}{\pgfpoint{0\du}{1.787500\du}}
\pgfusepath{stroke}
\pgfsetlinewidth{0.000000\du}
\pgfsetdash{}{0pt}
\pgfsetdash{}{0pt}
\pgfsetbuttcap
\pgfsetmiterjoin
\pgfsetlinewidth{0.000000\du}
\pgfsetbuttcap
\pgfsetmiterjoin
\pgfsetdash{}{0pt}
\definecolor{dialinecolor}{rgb}{0.247059, 0.317647, 0.709804}
\pgfsetfillcolor{dialinecolor}
\pgfpathellipse{\pgfpoint{20.002500\du}{-1.262500\du}}{\pgfpoint{1.787500\du}{0\du}}{\pgfpoint{0\du}{1.787500\du}}
\pgfusepath{fill}
\definecolor{dialinecolor}{rgb}{0.188235, 0.247059, 0.623529}
\pgfsetstrokecolor{dialinecolor}
\pgfpathellipse{\pgfpoint{20.002500\du}{-1.262500\du}}{\pgfpoint{1.787500\du}{0\du}}{\pgfpoint{0\du}{1.787500\du}}
\pgfusepath{stroke}
\pgfsetbuttcap
\pgfsetmiterjoin
\pgfsetdash{}{0pt}
\definecolor{dialinecolor}{rgb}{0.188235, 0.247059, 0.623529}
\pgfsetstrokecolor{dialinecolor}
\pgfpathellipse{\pgfpoint{20.002500\du}{-1.262500\du}}{\pgfpoint{1.787500\du}{0\du}}{\pgfpoint{0\du}{1.787500\du}}
\pgfusepath{stroke}
\pgfsetlinewidth{0.000000\du}
\pgfsetdash{}{0pt}
\pgfsetdash{}{0pt}
\pgfsetbuttcap
\pgfsetmiterjoin
\pgfsetlinewidth{0.000000\du}
\pgfsetbuttcap
\pgfsetmiterjoin
\pgfsetdash{}{0pt}
\definecolor{dialinecolor}{rgb}{0.247059, 0.317647, 0.709804}
\pgfsetfillcolor{dialinecolor}
\pgfpathellipse{\pgfpoint{12.667500\du}{6.212500\du}}{\pgfpoint{1.787500\du}{0\du}}{\pgfpoint{0\du}{1.787500\du}}
\pgfusepath{fill}
\definecolor{dialinecolor}{rgb}{0.188235, 0.247059, 0.623529}
\pgfsetstrokecolor{dialinecolor}
\pgfpathellipse{\pgfpoint{12.667500\du}{6.212500\du}}{\pgfpoint{1.787500\du}{0\du}}{\pgfpoint{0\du}{1.787500\du}}
\pgfusepath{stroke}
\pgfsetbuttcap
\pgfsetmiterjoin
\pgfsetdash{}{0pt}
\definecolor{dialinecolor}{rgb}{0.188235, 0.247059, 0.623529}
\pgfsetstrokecolor{dialinecolor}
\pgfpathellipse{\pgfpoint{12.667500\du}{6.212500\du}}{\pgfpoint{1.787500\du}{0\du}}{\pgfpoint{0\du}{1.787500\du}}
\pgfusepath{stroke}
\pgfsetlinewidth{0.000000\du}
\pgfsetdash{}{0pt}
\pgfsetdash{}{0pt}
\pgfsetbuttcap
\pgfsetmiterjoin
\pgfsetlinewidth{0.000000\du}
\pgfsetbuttcap
\pgfsetmiterjoin
\pgfsetdash{}{0pt}
\definecolor{dialinecolor}{rgb}{0.247059, 0.317647, 0.709804}
\pgfsetfillcolor{dialinecolor}
\pgfpathellipse{\pgfpoint{5.282500\du}{-1.312500\du}}{\pgfpoint{1.787500\du}{0\du}}{\pgfpoint{0\du}{1.787500\du}}
\pgfusepath{fill}
\definecolor{dialinecolor}{rgb}{0.188235, 0.247059, 0.623529}
\pgfsetstrokecolor{dialinecolor}
\pgfpathellipse{\pgfpoint{5.282500\du}{-1.312500\du}}{\pgfpoint{1.787500\du}{0\du}}{\pgfpoint{0\du}{1.787500\du}}
\pgfusepath{stroke}
\pgfsetbuttcap
\pgfsetmiterjoin
\pgfsetdash{}{0pt}
\definecolor{dialinecolor}{rgb}{0.188235, 0.247059, 0.623529}
\pgfsetstrokecolor{dialinecolor}
\pgfpathellipse{\pgfpoint{5.282500\du}{-1.312500\du}}{\pgfpoint{1.787500\du}{0\du}}{\pgfpoint{0\du}{1.787500\du}}
\pgfusepath{stroke}
\pgfsetlinewidth{0.100000\du}
\pgfsetdash{{\pgflinewidth}{0.200000\du}}{0cm}
\pgfsetdash{{\pgflinewidth}{0.200000\du}}{0cm}
\pgfsetbuttcap
{
\definecolor{dialinecolor}{rgb}{0.000000, 0.000000, 0.000000}
\pgfsetfillcolor{dialinecolor}
% was here!!!
\definecolor{dialinecolor}{rgb}{0.000000, 0.000000, 0.000000}
\pgfsetstrokecolor{dialinecolor}
\draw (12.737500\du,-7.050000\du)--(12.667500\du,4.425000\du);
}
\pgfsetlinewidth{0.100000\du}
\pgfsetdash{{\pgflinewidth}{0.200000\du}}{0cm}
\pgfsetdash{{\pgflinewidth}{0.200000\du}}{0cm}
\pgfsetbuttcap
{
\definecolor{dialinecolor}{rgb}{0.000000, 0.000000, 0.000000}
\pgfsetfillcolor{dialinecolor}
% was here!!!
\definecolor{dialinecolor}{rgb}{0.000000, 0.000000, 0.000000}
\pgfsetstrokecolor{dialinecolor}
\draw (7.069954\du,-1.305589\du)--(18.215000\du,-1.262500\du);
}
\pgfsetlinewidth{0.100000\du}
\pgfsetdash{}{0pt}
\pgfsetdash{}{0pt}
\pgfsetbuttcap
\pgfsetmiterjoin
\pgfsetlinewidth{0.100000\du}
\pgfsetbuttcap
\pgfsetmiterjoin
\pgfsetdash{}{0pt}
\definecolor{dialinecolor}{rgb}{1.000000, 1.000000, 1.000000}
\pgfsetfillcolor{dialinecolor}
\fill (11.682258\du,-10.925000\du)--(11.682258\du,-8.965000\du)--(13.579032\du,-8.965000\du)--(13.579032\du,-10.925000\du)--cycle;
\definecolor{dialinecolor}{rgb}{0.000000, 0.000000, 0.000000}
\pgfsetstrokecolor{dialinecolor}
\draw (11.682258\du,-10.925000\du)--(11.682258\du,-8.965000\du)--(13.579032\du,-8.965000\du)--(13.579032\du,-10.925000\du)--cycle;
\pgfsetbuttcap
\pgfsetmiterjoin
\pgfsetdash{}{0pt}
\definecolor{dialinecolor}{rgb}{0.000000, 0.000000, 0.000000}
\pgfsetstrokecolor{dialinecolor}
\draw (11.682258\du,-10.925000\du)--(11.682258\du,-8.965000\du)--(13.579032\du,-8.965000\du)--(13.579032\du,-10.925000\du)--cycle;
\pgfsetlinewidth{0.100000\du}
\pgfsetdash{}{0pt}
\pgfsetdash{}{0pt}
\pgfsetbuttcap
\pgfsetmiterjoin
\pgfsetlinewidth{0.100000\du}
\pgfsetbuttcap
\pgfsetmiterjoin
\pgfsetdash{}{0pt}
\definecolor{dialinecolor}{rgb}{1.000000, 1.000000, 1.000000}
\pgfsetfillcolor{dialinecolor}
\fill (20.015000\du,-2.300000\du)--(20.015000\du,-0.340000\du)--(21.911774\du,-0.340000\du)--(21.911774\du,-2.300000\du)--cycle;
\definecolor{dialinecolor}{rgb}{0.000000, 0.000000, 0.000000}
\pgfsetstrokecolor{dialinecolor}
\draw (20.015000\du,-2.300000\du)--(20.015000\du,-0.340000\du)--(21.911774\du,-0.340000\du)--(21.911774\du,-2.300000\du)--cycle;
\pgfsetbuttcap
\pgfsetmiterjoin
\pgfsetdash{}{0pt}
\definecolor{dialinecolor}{rgb}{0.000000, 0.000000, 0.000000}
\pgfsetstrokecolor{dialinecolor}
\draw (20.015000\du,-2.300000\du)--(20.015000\du,-0.340000\du)--(21.911774\du,-0.340000\du)--(21.911774\du,-2.300000\du)--cycle;
\pgfsetlinewidth{0.100000\du}
\pgfsetdash{}{0pt}
\pgfsetdash{}{0pt}
\pgfsetbuttcap
\pgfsetmiterjoin
\pgfsetlinewidth{0.100000\du}
\pgfsetbuttcap
\pgfsetmiterjoin
\pgfsetdash{}{0pt}
\definecolor{dialinecolor}{rgb}{1.000000, 1.000000, 1.000000}
\pgfsetfillcolor{dialinecolor}
\fill (11.730000\du,6.225000\du)--(11.730000\du,8.185000\du)--(13.626774\du,8.185000\du)--(13.626774\du,6.225000\du)--cycle;
\definecolor{dialinecolor}{rgb}{0.000000, 0.000000, 0.000000}
\pgfsetstrokecolor{dialinecolor}
\draw (11.730000\du,6.225000\du)--(11.730000\du,8.185000\du)--(13.626774\du,8.185000\du)--(13.626774\du,6.225000\du)--cycle;
\pgfsetbuttcap
\pgfsetmiterjoin
\pgfsetdash{}{0pt}
\definecolor{dialinecolor}{rgb}{0.000000, 0.000000, 0.000000}
\pgfsetstrokecolor{dialinecolor}
\draw (11.730000\du,6.225000\du)--(11.730000\du,8.185000\du)--(13.626774\du,8.185000\du)--(13.626774\du,6.225000\du)--cycle;
\pgfsetlinewidth{0.100000\du}
\pgfsetdash{}{0pt}
\pgfsetdash{}{0pt}
\pgfsetbuttcap
\pgfsetmiterjoin
\pgfsetlinewidth{0.100000\du}
\pgfsetbuttcap
\pgfsetmiterjoin
\pgfsetdash{}{0pt}
\definecolor{dialinecolor}{rgb}{1.000000, 1.000000, 1.000000}
\pgfsetfillcolor{dialinecolor}
\fill (3.545000\du,-2.300000\du)--(3.545000\du,-0.340000\du)--(5.441774\du,-0.340000\du)--(5.441774\du,-2.300000\du)--cycle;
\definecolor{dialinecolor}{rgb}{0.000000, 0.000000, 0.000000}
\pgfsetstrokecolor{dialinecolor}
\draw (3.545000\du,-2.300000\du)--(3.545000\du,-0.340000\du)--(5.441774\du,-0.340000\du)--(5.441774\du,-2.300000\du)--cycle;
\pgfsetbuttcap
\pgfsetmiterjoin
\pgfsetdash{}{0pt}
\definecolor{dialinecolor}{rgb}{0.000000, 0.000000, 0.000000}
\pgfsetstrokecolor{dialinecolor}
\draw (3.545000\du,-2.300000\du)--(3.545000\du,-0.340000\du)--(5.441774\du,-0.340000\du)--(5.441774\du,-2.300000\du)--cycle;
% setfont left to latex
\definecolor{dialinecolor}{rgb}{0.000000, 0.000000, 0.000000}
\pgfsetstrokecolor{dialinecolor}
\node at (12.630645\du,-9.652500\du){0};
% setfont left to latex
\definecolor{dialinecolor}{rgb}{0.000000, 0.000000, 0.000000}
\pgfsetstrokecolor{dialinecolor}
\node at (20.963387\du,-1.027500\du){25};
% setfont left to latex
\definecolor{dialinecolor}{rgb}{0.000000, 0.000000, 0.000000}
\pgfsetstrokecolor{dialinecolor}
\node at (12.678387\du,7.497500\du){50};
% setfont left to latex
\definecolor{dialinecolor}{rgb}{0.000000, 0.000000, 0.000000}
\pgfsetstrokecolor{dialinecolor}
\node at (4.493387\du,-1.027500\du){75};
% setfont left to latex
\definecolor{dialinecolor}{rgb}{0.000000, 0.000000, 0.000000}
\pgfsetstrokecolor{dialinecolor}
\node[anchor=west] at (5.350000\du,11.575000\du){};
% setfont left to latex
\definecolor{dialinecolor}{rgb}{0.000000, 0.000000, 0.000000}
\pgfsetstrokecolor{dialinecolor}
\node[anchor=west] at (14.700000\du,-3.725000\du){Noeud 2};
% setfont left to latex
\definecolor{dialinecolor}{rgb}{0.000000, 0.000000, 0.000000}
\pgfsetstrokecolor{dialinecolor}
\node[anchor=west] at (14.700000\du,-2.925000\du){\ensuremath{[}1,25\ensuremath{]}};
% setfont left to latex
\definecolor{dialinecolor}{rgb}{0.000000, 0.000000, 0.000000}
\pgfsetstrokecolor{dialinecolor}
\node[anchor=west] at (15.750000\du,-3.925000\du){};
% setfont left to latex
\definecolor{dialinecolor}{rgb}{0.000000, 0.000000, 0.000000}
\pgfsetstrokecolor{dialinecolor}
\node[anchor=west] at (14.500000\du,1.675000\du){Noeud 3};
% setfont left to latex
\definecolor{dialinecolor}{rgb}{0.000000, 0.000000, 0.000000}
\pgfsetstrokecolor{dialinecolor}
\node[anchor=west] at (14.500000\du,2.475000\du){\ensuremath{[}26,50\ensuremath{]}};
% setfont left to latex
\definecolor{dialinecolor}{rgb}{0.000000, 0.000000, 0.000000}
\pgfsetstrokecolor{dialinecolor}
\node[anchor=west] at (8.450000\du,1.675000\du){Noeud 4};
% setfont left to latex
\definecolor{dialinecolor}{rgb}{0.000000, 0.000000, 0.000000}
\pgfsetstrokecolor{dialinecolor}
\node[anchor=west] at (8.450000\du,2.475000\du){\ensuremath{[}51,75\ensuremath{]}};
% setfont left to latex
\definecolor{dialinecolor}{rgb}{0.000000, 0.000000, 0.000000}
\pgfsetstrokecolor{dialinecolor}
\node[anchor=west] at (8.350000\du,-3.625000\du){Noeud 1};
% setfont left to latex
\definecolor{dialinecolor}{rgb}{0.000000, 0.000000, 0.000000}
\pgfsetstrokecolor{dialinecolor}
\node[anchor=west] at (8.350000\du,-2.825000\du){\ensuremath{[}76,0\ensuremath{]}};
\pgfsetlinewidth{0.100000\du}
\pgfsetdash{}{0pt}
\pgfsetdash{}{0pt}
\pgfsetbuttcap
\pgfsetmiterjoin
\pgfsetlinewidth{0.100000\du}
\pgfsetbuttcap
\pgfsetmiterjoin
\pgfsetdash{}{0pt}
\definecolor{dialinecolor}{rgb}{1.000000, 1.000000, 1.000000}
\pgfsetfillcolor{dialinecolor}
\fill (18.082258\du,-13.525000\du)--(18.082258\du,-11.566667\du)--(19.977419\du,-11.566667\du)--(19.977419\du,-13.525000\du)--cycle;
\definecolor{dialinecolor}{rgb}{0.000000, 0.000000, 0.000000}
\pgfsetstrokecolor{dialinecolor}
\draw (18.082258\du,-13.525000\du)--(18.082258\du,-11.566667\du)--(19.977419\du,-11.566667\du)--(19.977419\du,-13.525000\du)--cycle;
\pgfsetbuttcap
\pgfsetmiterjoin
\pgfsetdash{}{0pt}
\definecolor{dialinecolor}{rgb}{0.000000, 0.000000, 0.000000}
\pgfsetstrokecolor{dialinecolor}
\draw (18.082258\du,-13.525000\du)--(18.082258\du,-11.566667\du)--(19.977419\du,-11.566667\du)--(19.977419\du,-13.525000\du)--cycle;
% setfont left to latex
\definecolor{dialinecolor}{rgb}{0.000000, 0.000000, 0.000000}
\pgfsetstrokecolor{dialinecolor}
\node at (19.029839\du,-12.253333\du){\#};
% setfont left to latex
\definecolor{dialinecolor}{rgb}{0.000000, 0.000000, 0.000000}
\pgfsetstrokecolor{dialinecolor}
\node[anchor=west] at (19.977419\du,-12.323333\du){Token};
\pgfsetlinewidth{0.000000\du}
\pgfsetdash{}{0pt}
\pgfsetdash{}{0pt}
\pgfsetbuttcap
\pgfsetmiterjoin
\pgfsetlinewidth{0.000000\du}
\pgfsetbuttcap
\pgfsetmiterjoin
\pgfsetdash{}{0pt}
\definecolor{dialinecolor}{rgb}{0.247059, 0.317647, 0.709804}
\pgfsetfillcolor{dialinecolor}
\pgfpathellipse{\pgfpoint{19.005000\du}{-9.960000\du}}{\pgfpoint{0.840000\du}{0\du}}{\pgfpoint{0\du}{0.840000\du}}
\pgfusepath{fill}
\definecolor{dialinecolor}{rgb}{0.188235, 0.247059, 0.623529}
\pgfsetstrokecolor{dialinecolor}
\pgfpathellipse{\pgfpoint{19.005000\du}{-9.960000\du}}{\pgfpoint{0.840000\du}{0\du}}{\pgfpoint{0\du}{0.840000\du}}
\pgfusepath{stroke}
\pgfsetbuttcap
\pgfsetmiterjoin
\pgfsetdash{}{0pt}
\definecolor{dialinecolor}{rgb}{0.188235, 0.247059, 0.623529}
\pgfsetstrokecolor{dialinecolor}
\pgfpathellipse{\pgfpoint{19.005000\du}{-9.960000\du}}{\pgfpoint{0.840000\du}{0\du}}{\pgfpoint{0\du}{0.840000\du}}
\pgfusepath{stroke}
% setfont left to latex
\definecolor{dialinecolor}{rgb}{0.000000, 0.000000, 0.000000}
\pgfsetstrokecolor{dialinecolor}
\node[anchor=west] at (19.845000\du,-9.737500\du){Noeud};
\end{tikzpicture}

	\caption{Exemple de partitionnement des données dans une base de données distribuées\label{fig:partitionning}}
\end{figure}

\begin{figure}[p]
	\centering
		% Graphic for TeX using PGF
% Title: C:\Users\Kéké\Pictures\multi_hash_partitionning.dia
% Creator: Dia v0.97.2
% CreationDate: Fri Feb 06 11:02:31 2015
% For: Kéké
% \usepackage{tikz}
% The following commands are not supported in PSTricks at present
% We define them conditionally, so when they are implemented,
% this pgf file will use them.
\ifx\du\undefined
  \newlength{\du}
\fi
\setlength{\du}{15\unitlength}
\begin{tikzpicture}
\pgftransformxscale{1.000000}
\pgftransformyscale{-1.000000}
\definecolor{dialinecolor}{rgb}{0.000000, 0.000000, 0.000000}
\pgfsetstrokecolor{dialinecolor}
\definecolor{dialinecolor}{rgb}{1.000000, 1.000000, 1.000000}
\pgfsetfillcolor{dialinecolor}
\pgfsetlinewidth{0.100000\du}
\pgfsetdash{}{0pt}
\pgfsetdash{}{0pt}
\pgfsetbuttcap
\pgfsetmiterjoin
\pgfsetlinewidth{0.100000\du}
\pgfsetbuttcap
\pgfsetmiterjoin
\pgfsetdash{}{0pt}
\definecolor{dialinecolor}{rgb}{0.960784, 0.960784, 0.960784}
\pgfsetfillcolor{dialinecolor}
\pgfpathellipse{\pgfpoint{12.082500\du}{15.637500\du}}{\pgfpoint{7.462500\du}{0\du}}{\pgfpoint{0\du}{7.462500\du}}
\pgfusepath{fill}
\definecolor{dialinecolor}{rgb}{0.000000, 0.000000, 0.000000}
\pgfsetstrokecolor{dialinecolor}
\pgfpathellipse{\pgfpoint{12.082500\du}{15.637500\du}}{\pgfpoint{7.462500\du}{0\du}}{\pgfpoint{0\du}{7.462500\du}}
\pgfusepath{stroke}
\pgfsetbuttcap
\pgfsetmiterjoin
\pgfsetdash{}{0pt}
\definecolor{dialinecolor}{rgb}{0.000000, 0.000000, 0.000000}
\pgfsetstrokecolor{dialinecolor}
\pgfpathellipse{\pgfpoint{12.082500\du}{15.637500\du}}{\pgfpoint{7.462500\du}{0\du}}{\pgfpoint{0\du}{7.462500\du}}
\pgfusepath{stroke}
\pgfsetlinewidth{0.000000\du}
\pgfsetdash{}{0pt}
\pgfsetdash{}{0pt}
\pgfsetbuttcap
\pgfsetmiterjoin
\pgfsetlinewidth{0.000000\du}
\pgfsetbuttcap
\pgfsetmiterjoin
\pgfsetdash{}{0pt}
\definecolor{dialinecolor}{rgb}{0.247059, 0.317647, 0.709804}
\pgfsetfillcolor{dialinecolor}
\pgfpathellipse{\pgfpoint{12.157500\du}{8.062500\du}}{\pgfpoint{1.787500\du}{0\du}}{\pgfpoint{0\du}{1.787500\du}}
\pgfusepath{fill}
\definecolor{dialinecolor}{rgb}{0.188235, 0.247059, 0.623529}
\pgfsetstrokecolor{dialinecolor}
\pgfpathellipse{\pgfpoint{12.157500\du}{8.062500\du}}{\pgfpoint{1.787500\du}{0\du}}{\pgfpoint{0\du}{1.787500\du}}
\pgfusepath{stroke}
\pgfsetbuttcap
\pgfsetmiterjoin
\pgfsetdash{}{0pt}
\definecolor{dialinecolor}{rgb}{0.188235, 0.247059, 0.623529}
\pgfsetstrokecolor{dialinecolor}
\pgfpathellipse{\pgfpoint{12.157500\du}{8.062500\du}}{\pgfpoint{1.787500\du}{0\du}}{\pgfpoint{0\du}{1.787500\du}}
\pgfusepath{stroke}
\pgfsetlinewidth{0.000000\du}
\pgfsetdash{}{0pt}
\pgfsetdash{}{0pt}
\pgfsetbuttcap
\pgfsetmiterjoin
\pgfsetlinewidth{0.000000\du}
\pgfsetbuttcap
\pgfsetmiterjoin
\pgfsetdash{}{0pt}
\definecolor{dialinecolor}{rgb}{0.247059, 0.317647, 0.709804}
\pgfsetfillcolor{dialinecolor}
\pgfpathellipse{\pgfpoint{19.422500\du}{15.637500\du}}{\pgfpoint{1.787500\du}{0\du}}{\pgfpoint{0\du}{1.787500\du}}
\pgfusepath{fill}
\definecolor{dialinecolor}{rgb}{0.188235, 0.247059, 0.623529}
\pgfsetstrokecolor{dialinecolor}
\pgfpathellipse{\pgfpoint{19.422500\du}{15.637500\du}}{\pgfpoint{1.787500\du}{0\du}}{\pgfpoint{0\du}{1.787500\du}}
\pgfusepath{stroke}
\pgfsetbuttcap
\pgfsetmiterjoin
\pgfsetdash{}{0pt}
\definecolor{dialinecolor}{rgb}{0.188235, 0.247059, 0.623529}
\pgfsetstrokecolor{dialinecolor}
\pgfpathellipse{\pgfpoint{19.422500\du}{15.637500\du}}{\pgfpoint{1.787500\du}{0\du}}{\pgfpoint{0\du}{1.787500\du}}
\pgfusepath{stroke}
\pgfsetlinewidth{0.000000\du}
\pgfsetdash{}{0pt}
\pgfsetdash{}{0pt}
\pgfsetbuttcap
\pgfsetmiterjoin
\pgfsetlinewidth{0.000000\du}
\pgfsetbuttcap
\pgfsetmiterjoin
\pgfsetdash{}{0pt}
\definecolor{dialinecolor}{rgb}{0.247059, 0.317647, 0.709804}
\pgfsetfillcolor{dialinecolor}
\pgfpathellipse{\pgfpoint{12.087500\du}{23.112500\du}}{\pgfpoint{1.787500\du}{0\du}}{\pgfpoint{0\du}{1.787500\du}}
\pgfusepath{fill}
\definecolor{dialinecolor}{rgb}{0.188235, 0.247059, 0.623529}
\pgfsetstrokecolor{dialinecolor}
\pgfpathellipse{\pgfpoint{12.087500\du}{23.112500\du}}{\pgfpoint{1.787500\du}{0\du}}{\pgfpoint{0\du}{1.787500\du}}
\pgfusepath{stroke}
\pgfsetbuttcap
\pgfsetmiterjoin
\pgfsetdash{}{0pt}
\definecolor{dialinecolor}{rgb}{0.188235, 0.247059, 0.623529}
\pgfsetstrokecolor{dialinecolor}
\pgfpathellipse{\pgfpoint{12.087500\du}{23.112500\du}}{\pgfpoint{1.787500\du}{0\du}}{\pgfpoint{0\du}{1.787500\du}}
\pgfusepath{stroke}
\pgfsetlinewidth{0.000000\du}
\pgfsetdash{}{0pt}
\pgfsetdash{}{0pt}
\pgfsetbuttcap
\pgfsetmiterjoin
\pgfsetlinewidth{0.000000\du}
\pgfsetbuttcap
\pgfsetmiterjoin
\pgfsetdash{}{0pt}
\definecolor{dialinecolor}{rgb}{0.247059, 0.317647, 0.709804}
\pgfsetfillcolor{dialinecolor}
\pgfpathellipse{\pgfpoint{4.702500\du}{15.587500\du}}{\pgfpoint{1.787500\du}{0\du}}{\pgfpoint{0\du}{1.787500\du}}
\pgfusepath{fill}
\definecolor{dialinecolor}{rgb}{0.188235, 0.247059, 0.623529}
\pgfsetstrokecolor{dialinecolor}
\pgfpathellipse{\pgfpoint{4.702500\du}{15.587500\du}}{\pgfpoint{1.787500\du}{0\du}}{\pgfpoint{0\du}{1.787500\du}}
\pgfusepath{stroke}
\pgfsetbuttcap
\pgfsetmiterjoin
\pgfsetdash{}{0pt}
\definecolor{dialinecolor}{rgb}{0.188235, 0.247059, 0.623529}
\pgfsetstrokecolor{dialinecolor}
\pgfpathellipse{\pgfpoint{4.702500\du}{15.587500\du}}{\pgfpoint{1.787500\du}{0\du}}{\pgfpoint{0\du}{1.787500\du}}
\pgfusepath{stroke}
\pgfsetlinewidth{0.100000\du}
\pgfsetdash{{\pgflinewidth}{0.200000\du}}{0cm}
\pgfsetdash{{\pgflinewidth}{0.200000\du}}{0cm}
\pgfsetbuttcap
{
\definecolor{dialinecolor}{rgb}{0.000000, 0.000000, 0.000000}
\pgfsetfillcolor{dialinecolor}
% was here!!!
\definecolor{dialinecolor}{rgb}{0.000000, 0.000000, 0.000000}
\pgfsetstrokecolor{dialinecolor}
\draw (12.157500\du,9.850000\du)--(12.087500\du,21.325000\du);
}
\pgfsetlinewidth{0.100000\du}
\pgfsetdash{{\pgflinewidth}{0.200000\du}}{0cm}
\pgfsetdash{{\pgflinewidth}{0.200000\du}}{0cm}
\pgfsetbuttcap
{
\definecolor{dialinecolor}{rgb}{0.000000, 0.000000, 0.000000}
\pgfsetfillcolor{dialinecolor}
% was here!!!
\definecolor{dialinecolor}{rgb}{0.000000, 0.000000, 0.000000}
\pgfsetstrokecolor{dialinecolor}
\draw (6.489954\du,15.594411\du)--(17.635000\du,15.637500\du);
}
\pgfsetlinewidth{0.100000\du}
\pgfsetdash{}{0pt}
\pgfsetdash{}{0pt}
\pgfsetbuttcap
\pgfsetmiterjoin
\pgfsetlinewidth{0.100000\du}
\pgfsetbuttcap
\pgfsetmiterjoin
\pgfsetdash{}{0pt}
\definecolor{dialinecolor}{rgb}{1.000000, 1.000000, 1.000000}
\pgfsetfillcolor{dialinecolor}
\fill (11.102300\du,5.975000\du)--(11.102300\du,7.935000\du)--(12.999074\du,7.935000\du)--(12.999074\du,5.975000\du)--cycle;
\definecolor{dialinecolor}{rgb}{0.000000, 0.000000, 0.000000}
\pgfsetstrokecolor{dialinecolor}
\draw (11.102300\du,5.975000\du)--(11.102300\du,7.935000\du)--(12.999074\du,7.935000\du)--(12.999074\du,5.975000\du)--cycle;
\pgfsetbuttcap
\pgfsetmiterjoin
\pgfsetdash{}{0pt}
\definecolor{dialinecolor}{rgb}{0.000000, 0.000000, 0.000000}
\pgfsetstrokecolor{dialinecolor}
\draw (11.102300\du,5.975000\du)--(11.102300\du,7.935000\du)--(12.999074\du,7.935000\du)--(12.999074\du,5.975000\du)--cycle;
\pgfsetlinewidth{0.100000\du}
\pgfsetdash{}{0pt}
\pgfsetdash{}{0pt}
\pgfsetbuttcap
\pgfsetmiterjoin
\pgfsetlinewidth{0.100000\du}
\pgfsetbuttcap
\pgfsetmiterjoin
\pgfsetdash{}{0pt}
\definecolor{dialinecolor}{rgb}{1.000000, 1.000000, 1.000000}
\pgfsetfillcolor{dialinecolor}
\fill (19.435000\du,14.600000\du)--(19.435000\du,16.560000\du)--(21.331774\du,16.560000\du)--(21.331774\du,14.600000\du)--cycle;
\definecolor{dialinecolor}{rgb}{0.000000, 0.000000, 0.000000}
\pgfsetstrokecolor{dialinecolor}
\draw (19.435000\du,14.600000\du)--(19.435000\du,16.560000\du)--(21.331774\du,16.560000\du)--(21.331774\du,14.600000\du)--cycle;
\pgfsetbuttcap
\pgfsetmiterjoin
\pgfsetdash{}{0pt}
\definecolor{dialinecolor}{rgb}{0.000000, 0.000000, 0.000000}
\pgfsetstrokecolor{dialinecolor}
\draw (19.435000\du,14.600000\du)--(19.435000\du,16.560000\du)--(21.331774\du,16.560000\du)--(21.331774\du,14.600000\du)--cycle;
\pgfsetlinewidth{0.100000\du}
\pgfsetdash{}{0pt}
\pgfsetdash{}{0pt}
\pgfsetbuttcap
\pgfsetmiterjoin
\pgfsetlinewidth{0.100000\du}
\pgfsetbuttcap
\pgfsetmiterjoin
\pgfsetdash{}{0pt}
\definecolor{dialinecolor}{rgb}{1.000000, 1.000000, 1.000000}
\pgfsetfillcolor{dialinecolor}
\fill (11.150000\du,23.125000\du)--(11.150000\du,25.085000\du)--(13.046774\du,25.085000\du)--(13.046774\du,23.125000\du)--cycle;
\definecolor{dialinecolor}{rgb}{0.000000, 0.000000, 0.000000}
\pgfsetstrokecolor{dialinecolor}
\draw (11.150000\du,23.125000\du)--(11.150000\du,25.085000\du)--(13.046774\du,25.085000\du)--(13.046774\du,23.125000\du)--cycle;
\pgfsetbuttcap
\pgfsetmiterjoin
\pgfsetdash{}{0pt}
\definecolor{dialinecolor}{rgb}{0.000000, 0.000000, 0.000000}
\pgfsetstrokecolor{dialinecolor}
\draw (11.150000\du,23.125000\du)--(11.150000\du,25.085000\du)--(13.046774\du,25.085000\du)--(13.046774\du,23.125000\du)--cycle;
\pgfsetlinewidth{0.100000\du}
\pgfsetdash{}{0pt}
\pgfsetdash{}{0pt}
\pgfsetbuttcap
\pgfsetmiterjoin
\pgfsetlinewidth{0.100000\du}
\pgfsetbuttcap
\pgfsetmiterjoin
\pgfsetdash{}{0pt}
\definecolor{dialinecolor}{rgb}{1.000000, 1.000000, 1.000000}
\pgfsetfillcolor{dialinecolor}
\fill (2.965000\du,14.600000\du)--(2.965000\du,16.560000\du)--(4.861774\du,16.560000\du)--(4.861774\du,14.600000\du)--cycle;
\definecolor{dialinecolor}{rgb}{0.000000, 0.000000, 0.000000}
\pgfsetstrokecolor{dialinecolor}
\draw (2.965000\du,14.600000\du)--(2.965000\du,16.560000\du)--(4.861774\du,16.560000\du)--(4.861774\du,14.600000\du)--cycle;
\pgfsetbuttcap
\pgfsetmiterjoin
\pgfsetdash{}{0pt}
\definecolor{dialinecolor}{rgb}{0.000000, 0.000000, 0.000000}
\pgfsetstrokecolor{dialinecolor}
\draw (2.965000\du,14.600000\du)--(2.965000\du,16.560000\du)--(4.861774\du,16.560000\du)--(4.861774\du,14.600000\du)--cycle;
% setfont left to latex
\definecolor{dialinecolor}{rgb}{0.000000, 0.000000, 0.000000}
\pgfsetstrokecolor{dialinecolor}
\node at (12.050600\du,7.247500\du){0};
% setfont left to latex
\definecolor{dialinecolor}{rgb}{0.000000, 0.000000, 0.000000}
\pgfsetstrokecolor{dialinecolor}
\node at (20.383400\du,15.872500\du){25};
% setfont left to latex
\definecolor{dialinecolor}{rgb}{0.000000, 0.000000, 0.000000}
\pgfsetstrokecolor{dialinecolor}
\node at (12.098400\du,24.397500\du){50};
% setfont left to latex
\definecolor{dialinecolor}{rgb}{0.000000, 0.000000, 0.000000}
\pgfsetstrokecolor{dialinecolor}
\node at (3.913390\du,15.872500\du){75};
% setfont left to latex
\definecolor{dialinecolor}{rgb}{0.000000, 0.000000, 0.000000}
\pgfsetstrokecolor{dialinecolor}
\node[anchor=west] at (14.120000\du,13.175000\du){Noeud 2};
% setfont left to latex
\definecolor{dialinecolor}{rgb}{0.000000, 0.000000, 0.000000}
\pgfsetstrokecolor{dialinecolor}
\node[anchor=west] at (14.120000\du,13.975000\du){\ensuremath{[}1,25\ensuremath{]}};
% setfont left to latex
\definecolor{dialinecolor}{rgb}{0.000000, 0.000000, 0.000000}
\pgfsetstrokecolor{dialinecolor}
\node[anchor=west] at (15.170000\du,12.975000\du){};
% setfont left to latex
\definecolor{dialinecolor}{rgb}{0.000000, 0.000000, 0.000000}
\pgfsetstrokecolor{dialinecolor}
\node[anchor=west] at (13.920000\du,18.575000\du){Noeud 3};
% setfont left to latex
\definecolor{dialinecolor}{rgb}{0.000000, 0.000000, 0.000000}
\pgfsetstrokecolor{dialinecolor}
\node[anchor=west] at (13.920000\du,19.375000\du){\ensuremath{[}26,50\ensuremath{]}};
% setfont left to latex
\definecolor{dialinecolor}{rgb}{0.000000, 0.000000, 0.000000}
\pgfsetstrokecolor{dialinecolor}
\node[anchor=west] at (7.870000\du,18.575000\du){Noeud 4};
% setfont left to latex
\definecolor{dialinecolor}{rgb}{0.000000, 0.000000, 0.000000}
\pgfsetstrokecolor{dialinecolor}
\node[anchor=west] at (7.870000\du,19.375000\du){\ensuremath{[}51,75\ensuremath{]}};
% setfont left to latex
\definecolor{dialinecolor}{rgb}{0.000000, 0.000000, 0.000000}
\pgfsetstrokecolor{dialinecolor}
\node[anchor=west] at (7.770000\du,13.275000\du){Noeud 1};
% setfont left to latex
\definecolor{dialinecolor}{rgb}{0.000000, 0.000000, 0.000000}
\pgfsetstrokecolor{dialinecolor}
\node[anchor=west] at (7.770000\du,14.075000\du){\ensuremath{[}76,0\ensuremath{]}};
\pgfsetlinewidth{0.000000\du}
\pgfsetdash{}{0pt}
\pgfsetdash{}{0pt}
\pgfsetbuttcap
\pgfsetmiterjoin
\pgfsetlinewidth{0.000000\du}
\pgfsetbuttcap
\pgfsetmiterjoin
\pgfsetdash{}{0pt}
\definecolor{dialinecolor}{rgb}{1.000000, 0.341176, 0.133333}
\pgfsetfillcolor{dialinecolor}
\pgfpathellipse{\pgfpoint{18.112500\du}{3.412500\du}}{\pgfpoint{0.612500\du}{0\du}}{\pgfpoint{0\du}{0.612500\du}}
\pgfusepath{fill}
\definecolor{dialinecolor}{rgb}{1.000000, 1.000000, 1.000000}
\pgfsetstrokecolor{dialinecolor}
\pgfpathellipse{\pgfpoint{18.112500\du}{3.412500\du}}{\pgfpoint{0.612500\du}{0\du}}{\pgfpoint{0\du}{0.612500\du}}
\pgfusepath{stroke}
\pgfsetbuttcap
\pgfsetmiterjoin
\pgfsetdash{}{0pt}
\definecolor{dialinecolor}{rgb}{1.000000, 1.000000, 1.000000}
\pgfsetstrokecolor{dialinecolor}
\pgfpathellipse{\pgfpoint{18.112500\du}{3.412500\du}}{\pgfpoint{0.612500\du}{0\du}}{\pgfpoint{0\du}{0.612500\du}}
\pgfusepath{stroke}
% setfont left to latex
\definecolor{dialinecolor}{rgb}{0.000000, 0.000000, 0.000000}
\pgfsetstrokecolor{dialinecolor}
\node[anchor=west] at (18.725000\du,3.635000\du){Hash1(CleObjet) = 10};
\pgfsetlinewidth{0.000000\du}
\pgfsetdash{}{0pt}
\pgfsetdash{}{0pt}
\pgfsetbuttcap
\pgfsetmiterjoin
\pgfsetlinewidth{0.000000\du}
\pgfsetbuttcap
\pgfsetmiterjoin
\pgfsetdash{}{0pt}
\definecolor{dialinecolor}{rgb}{0.298039, 0.686275, 0.313726}
\pgfsetfillcolor{dialinecolor}
\pgfpathellipse{\pgfpoint{18.077500\du}{5.387500\du}}{\pgfpoint{0.612500\du}{0\du}}{\pgfpoint{0\du}{0.612500\du}}
\pgfusepath{fill}
\definecolor{dialinecolor}{rgb}{1.000000, 1.000000, 1.000000}
\pgfsetstrokecolor{dialinecolor}
\pgfpathellipse{\pgfpoint{18.077500\du}{5.387500\du}}{\pgfpoint{0.612500\du}{0\du}}{\pgfpoint{0\du}{0.612500\du}}
\pgfusepath{stroke}
\pgfsetbuttcap
\pgfsetmiterjoin
\pgfsetdash{}{0pt}
\definecolor{dialinecolor}{rgb}{1.000000, 1.000000, 1.000000}
\pgfsetstrokecolor{dialinecolor}
\pgfpathellipse{\pgfpoint{18.077500\du}{5.387500\du}}{\pgfpoint{0.612500\du}{0\du}}{\pgfpoint{0\du}{0.612500\du}}
\pgfusepath{stroke}
% setfont left to latex
\definecolor{dialinecolor}{rgb}{0.000000, 0.000000, 0.000000}
\pgfsetstrokecolor{dialinecolor}
\node[anchor=west] at (18.690000\du,5.610000\du){Hash2(CleObjet) = 28};
\pgfsetlinewidth{0.000000\du}
\pgfsetdash{}{0pt}
\pgfsetdash{}{0pt}
\pgfsetbuttcap
\pgfsetmiterjoin
\pgfsetlinewidth{0.000000\du}
\pgfsetbuttcap
\pgfsetmiterjoin
\pgfsetdash{}{0pt}
\definecolor{dialinecolor}{rgb}{0.611765, 0.152941, 0.690196}
\pgfsetfillcolor{dialinecolor}
\pgfpathellipse{\pgfpoint{18.042500\du}{7.362500\du}}{\pgfpoint{0.612500\du}{0\du}}{\pgfpoint{0\du}{0.612500\du}}
\pgfusepath{fill}
\definecolor{dialinecolor}{rgb}{1.000000, 1.000000, 1.000000}
\pgfsetstrokecolor{dialinecolor}
\pgfpathellipse{\pgfpoint{18.042500\du}{7.362500\du}}{\pgfpoint{0.612500\du}{0\du}}{\pgfpoint{0\du}{0.612500\du}}
\pgfusepath{stroke}
\pgfsetbuttcap
\pgfsetmiterjoin
\pgfsetdash{}{0pt}
\definecolor{dialinecolor}{rgb}{1.000000, 1.000000, 1.000000}
\pgfsetstrokecolor{dialinecolor}
\pgfpathellipse{\pgfpoint{18.042500\du}{7.362500\du}}{\pgfpoint{0.612500\du}{0\du}}{\pgfpoint{0\du}{0.612500\du}}
\pgfusepath{stroke}
% setfont left to latex
\definecolor{dialinecolor}{rgb}{0.000000, 0.000000, 0.000000}
\pgfsetstrokecolor{dialinecolor}
\node[anchor=west] at (18.655000\du,7.585000\du){Hash3(CleObjet) = 88};
\pgfsetlinewidth{0.000000\du}
\pgfsetdash{}{0pt}
\pgfsetdash{}{0pt}
\pgfsetbuttcap
\pgfsetmiterjoin
\pgfsetlinewidth{0.000000\du}
\pgfsetbuttcap
\pgfsetmiterjoin
\pgfsetdash{}{0pt}
\definecolor{dialinecolor}{rgb}{1.000000, 0.341176, 0.133333}
\pgfsetfillcolor{dialinecolor}
\pgfpathellipse{\pgfpoint{15.727500\du}{9.337500\du}}{\pgfpoint{0.612500\du}{0\du}}{\pgfpoint{0\du}{0.612500\du}}
\pgfusepath{fill}
\definecolor{dialinecolor}{rgb}{1.000000, 1.000000, 1.000000}
\pgfsetstrokecolor{dialinecolor}
\pgfpathellipse{\pgfpoint{15.727500\du}{9.337500\du}}{\pgfpoint{0.612500\du}{0\du}}{\pgfpoint{0\du}{0.612500\du}}
\pgfusepath{stroke}
\pgfsetbuttcap
\pgfsetmiterjoin
\pgfsetdash{}{0pt}
\definecolor{dialinecolor}{rgb}{1.000000, 1.000000, 1.000000}
\pgfsetstrokecolor{dialinecolor}
\pgfpathellipse{\pgfpoint{15.727500\du}{9.337500\du}}{\pgfpoint{0.612500\du}{0\du}}{\pgfpoint{0\du}{0.612500\du}}
\pgfusepath{stroke}
\pgfsetlinewidth{0.000000\du}
\pgfsetdash{}{0pt}
\pgfsetdash{}{0pt}
\pgfsetbuttcap
\pgfsetmiterjoin
\pgfsetlinewidth{0.000000\du}
\pgfsetbuttcap
\pgfsetmiterjoin
\pgfsetdash{}{0pt}
\definecolor{dialinecolor}{rgb}{0.298039, 0.686275, 0.313726}
\pgfsetfillcolor{dialinecolor}
\pgfpathellipse{\pgfpoint{18.977500\du}{18.137500\du}}{\pgfpoint{0.612500\du}{0\du}}{\pgfpoint{0\du}{0.612500\du}}
\pgfusepath{fill}
\definecolor{dialinecolor}{rgb}{1.000000, 1.000000, 1.000000}
\pgfsetstrokecolor{dialinecolor}
\pgfpathellipse{\pgfpoint{18.977500\du}{18.137500\du}}{\pgfpoint{0.612500\du}{0\du}}{\pgfpoint{0\du}{0.612500\du}}
\pgfusepath{stroke}
\pgfsetbuttcap
\pgfsetmiterjoin
\pgfsetdash{}{0pt}
\definecolor{dialinecolor}{rgb}{1.000000, 1.000000, 1.000000}
\pgfsetstrokecolor{dialinecolor}
\pgfpathellipse{\pgfpoint{18.977500\du}{18.137500\du}}{\pgfpoint{0.612500\du}{0\du}}{\pgfpoint{0\du}{0.612500\du}}
\pgfusepath{stroke}
\pgfsetlinewidth{0.000000\du}
\pgfsetdash{}{0pt}
\pgfsetdash{}{0pt}
\pgfsetbuttcap
\pgfsetmiterjoin
\pgfsetlinewidth{0.000000\du}
\pgfsetbuttcap
\pgfsetmiterjoin
\pgfsetdash{}{0pt}
\definecolor{dialinecolor}{rgb}{0.611765, 0.152941, 0.690196}
\pgfsetfillcolor{dialinecolor}
\pgfpathellipse{\pgfpoint{6.327500\du}{11.087500\du}}{\pgfpoint{0.612500\du}{0\du}}{\pgfpoint{0\du}{0.612500\du}}
\pgfusepath{fill}
\definecolor{dialinecolor}{rgb}{1.000000, 1.000000, 1.000000}
\pgfsetstrokecolor{dialinecolor}
\pgfpathellipse{\pgfpoint{6.327500\du}{11.087500\du}}{\pgfpoint{0.612500\du}{0\du}}{\pgfpoint{0\du}{0.612500\du}}
\pgfusepath{stroke}
\pgfsetbuttcap
\pgfsetmiterjoin
\pgfsetdash{}{0pt}
\definecolor{dialinecolor}{rgb}{1.000000, 1.000000, 1.000000}
\pgfsetstrokecolor{dialinecolor}
\pgfpathellipse{\pgfpoint{6.327500\du}{11.087500\du}}{\pgfpoint{0.612500\du}{0\du}}{\pgfpoint{0\du}{0.612500\du}}
\pgfusepath{stroke}
% setfont left to latex
\definecolor{dialinecolor}{rgb}{0.000000, 0.000000, 0.000000}
\pgfsetstrokecolor{dialinecolor}
\node[anchor=west] at (3.100000\du,3.650000\du){Objet = \{CleObjet, Données\}};
\pgfsetlinewidth{0.100000\du}
\pgfsetdash{}{0pt}
\pgfsetdash{}{0pt}
\pgfsetbuttcap
\pgfsetmiterjoin
\pgfsetlinewidth{0.100000\du}
\pgfsetbuttcap
\pgfsetmiterjoin
\pgfsetdash{}{0pt}
\definecolor{dialinecolor}{rgb}{1.000000, 1.000000, 1.000000}
\pgfsetfillcolor{dialinecolor}
\fill (21.364919\du,20.125033\du)--(21.364919\du,22.083367\du)--(23.260081\du,22.083367\du)--(23.260081\du,20.125033\du)--cycle;
\definecolor{dialinecolor}{rgb}{0.000000, 0.000000, 0.000000}
\pgfsetstrokecolor{dialinecolor}
\draw (21.364919\du,20.125033\du)--(21.364919\du,22.083367\du)--(23.260081\du,22.083367\du)--(23.260081\du,20.125033\du)--cycle;
\pgfsetbuttcap
\pgfsetmiterjoin
\pgfsetdash{}{0pt}
\definecolor{dialinecolor}{rgb}{0.000000, 0.000000, 0.000000}
\pgfsetstrokecolor{dialinecolor}
\draw (21.364919\du,20.125033\du)--(21.364919\du,22.083367\du)--(23.260081\du,22.083367\du)--(23.260081\du,20.125033\du)--cycle;
% setfont left to latex
\definecolor{dialinecolor}{rgb}{0.000000, 0.000000, 0.000000}
\pgfsetstrokecolor{dialinecolor}
\node at (22.312500\du,21.396700\du){\#};
% setfont left to latex
\definecolor{dialinecolor}{rgb}{0.000000, 0.000000, 0.000000}
\pgfsetstrokecolor{dialinecolor}
\node[anchor=west] at (23.260081\du,21.326700\du){Token};
\pgfsetlinewidth{0.000000\du}
\pgfsetdash{}{0pt}
\pgfsetdash{}{0pt}
\pgfsetbuttcap
\pgfsetmiterjoin
\pgfsetlinewidth{0.000000\du}
\pgfsetbuttcap
\pgfsetmiterjoin
\pgfsetdash{}{0pt}
\definecolor{dialinecolor}{rgb}{0.247059, 0.317647, 0.709804}
\pgfsetfillcolor{dialinecolor}
\pgfpathellipse{\pgfpoint{22.287619\du}{23.690033\du}}{\pgfpoint{0.840000\du}{0\du}}{\pgfpoint{0\du}{0.840000\du}}
\pgfusepath{fill}
\definecolor{dialinecolor}{rgb}{0.188235, 0.247059, 0.623529}
\pgfsetstrokecolor{dialinecolor}
\pgfpathellipse{\pgfpoint{22.287619\du}{23.690033\du}}{\pgfpoint{0.840000\du}{0\du}}{\pgfpoint{0\du}{0.840000\du}}
\pgfusepath{stroke}
\pgfsetbuttcap
\pgfsetmiterjoin
\pgfsetdash{}{0pt}
\definecolor{dialinecolor}{rgb}{0.188235, 0.247059, 0.623529}
\pgfsetstrokecolor{dialinecolor}
\pgfpathellipse{\pgfpoint{22.287619\du}{23.690033\du}}{\pgfpoint{0.840000\du}{0\du}}{\pgfpoint{0\du}{0.840000\du}}
\pgfusepath{stroke}
% setfont left to latex
\definecolor{dialinecolor}{rgb}{0.000000, 0.000000, 0.000000}
\pgfsetstrokecolor{dialinecolor}
\node[anchor=west] at (23.127619\du,23.912533\du){Noeud};
\end{tikzpicture}

	\caption{Partitionnement des réplicas d'un objet avec une fonction de hachage pour chaque réplica\label{fig:multi_hash_partitionning}}
\end{figure}

\begin{figure}[p]
	\centering
		% Graphic for TeX using PGF
% Title: C:\Users\Kéké\Pictures\request.dia
% Creator: Dia v0.97.2
% CreationDate: Thu Feb 05 11:17:12 2015
% For: Kéké
% \usepackage{tikz}
% The following commands are not supported in PSTricks at present
% We define them conditionally, so when they are implemented,
% this pgf file will use them.
\ifx\du\undefined
  \newlength{\du}
\fi
\setlength{\du}{15\unitlength}
\begin{tikzpicture}
\pgftransformxscale{1.000000}
\pgftransformyscale{-1.000000}
\definecolor{dialinecolor}{rgb}{0.000000, 0.000000, 0.000000}
\pgfsetstrokecolor{dialinecolor}
\definecolor{dialinecolor}{rgb}{1.000000, 1.000000, 1.000000}
\pgfsetfillcolor{dialinecolor}
\pgfsetlinewidth{0.100000\du}
\pgfsetdash{}{0pt}
\pgfsetdash{}{0pt}
\pgfsetbuttcap
\pgfsetmiterjoin
\pgfsetlinewidth{0.100000\du}
\pgfsetbuttcap
\pgfsetmiterjoin
\pgfsetdash{}{0pt}
\definecolor{dialinecolor}{rgb}{0.960784, 0.960784, 0.960784}
\pgfsetfillcolor{dialinecolor}
\pgfpathellipse{\pgfpoint{14.750000\du}{15.250000\du}}{\pgfpoint{6.700000\du}{0\du}}{\pgfpoint{0\du}{6.700000\du}}
\pgfusepath{fill}
\definecolor{dialinecolor}{rgb}{0.000000, 0.000000, 0.000000}
\pgfsetstrokecolor{dialinecolor}
\pgfpathellipse{\pgfpoint{14.750000\du}{15.250000\du}}{\pgfpoint{6.700000\du}{0\du}}{\pgfpoint{0\du}{6.700000\du}}
\pgfusepath{stroke}
\pgfsetbuttcap
\pgfsetmiterjoin
\pgfsetdash{}{0pt}
\definecolor{dialinecolor}{rgb}{0.000000, 0.000000, 0.000000}
\pgfsetstrokecolor{dialinecolor}
\pgfpathellipse{\pgfpoint{14.750000\du}{15.250000\du}}{\pgfpoint{6.700000\du}{0\du}}{\pgfpoint{0\du}{6.700000\du}}
\pgfusepath{stroke}
\pgfsetlinewidth{0.000000\du}
\pgfsetdash{}{0pt}
\pgfsetdash{}{0pt}
\pgfsetbuttcap
\pgfsetmiterjoin
\pgfsetlinewidth{0.000000\du}
\pgfsetbuttcap
\pgfsetmiterjoin
\pgfsetdash{}{0pt}
\definecolor{dialinecolor}{rgb}{0.247059, 0.317647, 0.709804}
\pgfsetfillcolor{dialinecolor}
\pgfpathellipse{\pgfpoint{12.750000\du}{9.050000\du}}{\pgfpoint{1.300000\du}{0\du}}{\pgfpoint{0\du}{1.300000\du}}
\pgfusepath{fill}
\definecolor{dialinecolor}{rgb}{0.000000, 0.000000, 0.000000}
\pgfsetstrokecolor{dialinecolor}
\pgfpathellipse{\pgfpoint{12.750000\du}{9.050000\du}}{\pgfpoint{1.300000\du}{0\du}}{\pgfpoint{0\du}{1.300000\du}}
\pgfusepath{stroke}
\pgfsetbuttcap
\pgfsetmiterjoin
\pgfsetdash{}{0pt}
\definecolor{dialinecolor}{rgb}{0.000000, 0.000000, 0.000000}
\pgfsetstrokecolor{dialinecolor}
\pgfpathellipse{\pgfpoint{12.750000\du}{9.050000\du}}{\pgfpoint{1.300000\du}{0\du}}{\pgfpoint{0\du}{1.300000\du}}
\pgfusepath{stroke}
\pgfsetlinewidth{0.000000\du}
\pgfsetdash{}{0pt}
\pgfsetdash{}{0pt}
\pgfsetbuttcap
\pgfsetmiterjoin
\pgfsetlinewidth{0.000000\du}
\pgfsetbuttcap
\pgfsetmiterjoin
\pgfsetdash{}{0pt}
\definecolor{dialinecolor}{rgb}{0.247059, 0.317647, 0.709804}
\pgfsetfillcolor{dialinecolor}
\pgfpathellipse{\pgfpoint{16.445000\du}{8.875000\du}}{\pgfpoint{1.300000\du}{0\du}}{\pgfpoint{0\du}{1.300000\du}}
\pgfusepath{fill}
\definecolor{dialinecolor}{rgb}{0.000000, 0.000000, 0.000000}
\pgfsetstrokecolor{dialinecolor}
\pgfpathellipse{\pgfpoint{16.445000\du}{8.875000\du}}{\pgfpoint{1.300000\du}{0\du}}{\pgfpoint{0\du}{1.300000\du}}
\pgfusepath{stroke}
\pgfsetbuttcap
\pgfsetmiterjoin
\pgfsetdash{}{0pt}
\definecolor{dialinecolor}{rgb}{0.000000, 0.000000, 0.000000}
\pgfsetstrokecolor{dialinecolor}
\pgfpathellipse{\pgfpoint{16.445000\du}{8.875000\du}}{\pgfpoint{1.300000\du}{0\du}}{\pgfpoint{0\du}{1.300000\du}}
\pgfusepath{stroke}
\pgfsetlinewidth{0.000000\du}
\pgfsetdash{}{0pt}
\pgfsetdash{}{0pt}
\pgfsetbuttcap
\pgfsetmiterjoin
\pgfsetlinewidth{0.000000\du}
\pgfsetbuttcap
\pgfsetmiterjoin
\pgfsetdash{}{0pt}
\definecolor{dialinecolor}{rgb}{0.247059, 0.317647, 0.709804}
\pgfsetfillcolor{dialinecolor}
\pgfpathellipse{\pgfpoint{19.740000\du}{11.050000\du}}{\pgfpoint{1.300000\du}{0\du}}{\pgfpoint{0\du}{1.300000\du}}
\pgfusepath{fill}
\definecolor{dialinecolor}{rgb}{0.000000, 0.000000, 0.000000}
\pgfsetstrokecolor{dialinecolor}
\pgfpathellipse{\pgfpoint{19.740000\du}{11.050000\du}}{\pgfpoint{1.300000\du}{0\du}}{\pgfpoint{0\du}{1.300000\du}}
\pgfusepath{stroke}
\pgfsetbuttcap
\pgfsetmiterjoin
\pgfsetdash{}{0pt}
\definecolor{dialinecolor}{rgb}{0.000000, 0.000000, 0.000000}
\pgfsetstrokecolor{dialinecolor}
\pgfpathellipse{\pgfpoint{19.740000\du}{11.050000\du}}{\pgfpoint{1.300000\du}{0\du}}{\pgfpoint{0\du}{1.300000\du}}
\pgfusepath{stroke}
\pgfsetlinewidth{0.000000\du}
\pgfsetdash{}{0pt}
\pgfsetdash{}{0pt}
\pgfsetbuttcap
\pgfsetmiterjoin
\pgfsetlinewidth{0.000000\du}
\pgfsetbuttcap
\pgfsetmiterjoin
\pgfsetdash{}{0pt}
\definecolor{dialinecolor}{rgb}{0.247059, 0.317647, 0.709804}
\pgfsetfillcolor{dialinecolor}
\pgfpathellipse{\pgfpoint{21.235000\du}{15.425000\du}}{\pgfpoint{1.300000\du}{0\du}}{\pgfpoint{0\du}{1.300000\du}}
\pgfusepath{fill}
\definecolor{dialinecolor}{rgb}{0.000000, 0.000000, 0.000000}
\pgfsetstrokecolor{dialinecolor}
\pgfpathellipse{\pgfpoint{21.235000\du}{15.425000\du}}{\pgfpoint{1.300000\du}{0\du}}{\pgfpoint{0\du}{1.300000\du}}
\pgfusepath{stroke}
\pgfsetbuttcap
\pgfsetmiterjoin
\pgfsetdash{}{0pt}
\definecolor{dialinecolor}{rgb}{0.000000, 0.000000, 0.000000}
\pgfsetstrokecolor{dialinecolor}
\pgfpathellipse{\pgfpoint{21.235000\du}{15.425000\du}}{\pgfpoint{1.300000\du}{0\du}}{\pgfpoint{0\du}{1.300000\du}}
\pgfusepath{stroke}
\pgfsetlinewidth{0.000000\du}
\pgfsetdash{}{0pt}
\pgfsetdash{}{0pt}
\pgfsetbuttcap
\pgfsetmiterjoin
\pgfsetlinewidth{0.000000\du}
\pgfsetbuttcap
\pgfsetmiterjoin
\pgfsetdash{}{0pt}
\definecolor{dialinecolor}{rgb}{0.247059, 0.317647, 0.709804}
\pgfsetfillcolor{dialinecolor}
\pgfpathellipse{\pgfpoint{19.830000\du}{19.050000\du}}{\pgfpoint{1.300000\du}{0\du}}{\pgfpoint{0\du}{1.300000\du}}
\pgfusepath{fill}
\definecolor{dialinecolor}{rgb}{0.000000, 0.000000, 0.000000}
\pgfsetstrokecolor{dialinecolor}
\pgfpathellipse{\pgfpoint{19.830000\du}{19.050000\du}}{\pgfpoint{1.300000\du}{0\du}}{\pgfpoint{0\du}{1.300000\du}}
\pgfusepath{stroke}
\pgfsetbuttcap
\pgfsetmiterjoin
\pgfsetdash{}{0pt}
\definecolor{dialinecolor}{rgb}{0.000000, 0.000000, 0.000000}
\pgfsetstrokecolor{dialinecolor}
\pgfpathellipse{\pgfpoint{19.830000\du}{19.050000\du}}{\pgfpoint{1.300000\du}{0\du}}{\pgfpoint{0\du}{1.300000\du}}
\pgfusepath{stroke}
\pgfsetlinewidth{0.000000\du}
\pgfsetdash{}{0pt}
\pgfsetdash{}{0pt}
\pgfsetbuttcap
\pgfsetmiterjoin
\pgfsetlinewidth{0.000000\du}
\pgfsetbuttcap
\pgfsetmiterjoin
\pgfsetdash{}{0pt}
\definecolor{dialinecolor}{rgb}{0.247059, 0.317647, 0.709804}
\pgfsetfillcolor{dialinecolor}
\pgfpathellipse{\pgfpoint{16.325000\du}{21.375000\du}}{\pgfpoint{1.300000\du}{0\du}}{\pgfpoint{0\du}{1.300000\du}}
\pgfusepath{fill}
\definecolor{dialinecolor}{rgb}{0.000000, 0.000000, 0.000000}
\pgfsetstrokecolor{dialinecolor}
\pgfpathellipse{\pgfpoint{16.325000\du}{21.375000\du}}{\pgfpoint{1.300000\du}{0\du}}{\pgfpoint{0\du}{1.300000\du}}
\pgfusepath{stroke}
\pgfsetbuttcap
\pgfsetmiterjoin
\pgfsetdash{}{0pt}
\definecolor{dialinecolor}{rgb}{0.000000, 0.000000, 0.000000}
\pgfsetstrokecolor{dialinecolor}
\pgfpathellipse{\pgfpoint{16.325000\du}{21.375000\du}}{\pgfpoint{1.300000\du}{0\du}}{\pgfpoint{0\du}{1.300000\du}}
\pgfusepath{stroke}
\pgfsetlinewidth{0.000000\du}
\pgfsetdash{}{0pt}
\pgfsetdash{}{0pt}
\pgfsetbuttcap
\pgfsetmiterjoin
\pgfsetlinewidth{0.000000\du}
\pgfsetbuttcap
\pgfsetmiterjoin
\pgfsetdash{}{0pt}
\definecolor{dialinecolor}{rgb}{0.247059, 0.317647, 0.709804}
\pgfsetfillcolor{dialinecolor}
\pgfpathellipse{\pgfpoint{12.270000\du}{21.550000\du}}{\pgfpoint{1.300000\du}{0\du}}{\pgfpoint{0\du}{1.300000\du}}
\pgfusepath{fill}
\definecolor{dialinecolor}{rgb}{0.000000, 0.000000, 0.000000}
\pgfsetstrokecolor{dialinecolor}
\pgfpathellipse{\pgfpoint{12.270000\du}{21.550000\du}}{\pgfpoint{1.300000\du}{0\du}}{\pgfpoint{0\du}{1.300000\du}}
\pgfusepath{stroke}
\pgfsetbuttcap
\pgfsetmiterjoin
\pgfsetdash{}{0pt}
\definecolor{dialinecolor}{rgb}{0.000000, 0.000000, 0.000000}
\pgfsetstrokecolor{dialinecolor}
\pgfpathellipse{\pgfpoint{12.270000\du}{21.550000\du}}{\pgfpoint{1.300000\du}{0\du}}{\pgfpoint{0\du}{1.300000\du}}
\pgfusepath{stroke}
\pgfsetlinewidth{0.000000\du}
\pgfsetdash{}{0pt}
\pgfsetdash{}{0pt}
\pgfsetbuttcap
\pgfsetmiterjoin
\pgfsetlinewidth{0.000000\du}
\pgfsetbuttcap
\pgfsetmiterjoin
\pgfsetdash{}{0pt}
\definecolor{dialinecolor}{rgb}{0.247059, 0.317647, 0.709804}
\pgfsetfillcolor{dialinecolor}
\pgfpathellipse{\pgfpoint{9.265000\du}{18.975000\du}}{\pgfpoint{1.300000\du}{0\du}}{\pgfpoint{0\du}{1.300000\du}}
\pgfusepath{fill}
\definecolor{dialinecolor}{rgb}{0.000000, 0.000000, 0.000000}
\pgfsetstrokecolor{dialinecolor}
\pgfpathellipse{\pgfpoint{9.265000\du}{18.975000\du}}{\pgfpoint{1.300000\du}{0\du}}{\pgfpoint{0\du}{1.300000\du}}
\pgfusepath{stroke}
\pgfsetbuttcap
\pgfsetmiterjoin
\pgfsetdash{}{0pt}
\definecolor{dialinecolor}{rgb}{0.000000, 0.000000, 0.000000}
\pgfsetstrokecolor{dialinecolor}
\pgfpathellipse{\pgfpoint{9.265000\du}{18.975000\du}}{\pgfpoint{1.300000\du}{0\du}}{\pgfpoint{0\du}{1.300000\du}}
\pgfusepath{stroke}
\pgfsetlinewidth{0.000000\du}
\pgfsetdash{}{0pt}
\pgfsetdash{}{0pt}
\pgfsetbuttcap
\pgfsetmiterjoin
\pgfsetlinewidth{0.000000\du}
\pgfsetbuttcap
\pgfsetmiterjoin
\pgfsetdash{}{0pt}
\definecolor{dialinecolor}{rgb}{0.247059, 0.317647, 0.709804}
\pgfsetfillcolor{dialinecolor}
\pgfpathellipse{\pgfpoint{8.010000\du}{15.050000\du}}{\pgfpoint{1.300000\du}{0\du}}{\pgfpoint{0\du}{1.300000\du}}
\pgfusepath{fill}
\definecolor{dialinecolor}{rgb}{0.000000, 0.000000, 0.000000}
\pgfsetstrokecolor{dialinecolor}
\pgfpathellipse{\pgfpoint{8.010000\du}{15.050000\du}}{\pgfpoint{1.300000\du}{0\du}}{\pgfpoint{0\du}{1.300000\du}}
\pgfusepath{stroke}
\pgfsetbuttcap
\pgfsetmiterjoin
\pgfsetdash{}{0pt}
\definecolor{dialinecolor}{rgb}{0.000000, 0.000000, 0.000000}
\pgfsetstrokecolor{dialinecolor}
\pgfpathellipse{\pgfpoint{8.010000\du}{15.050000\du}}{\pgfpoint{1.300000\du}{0\du}}{\pgfpoint{0\du}{1.300000\du}}
\pgfusepath{stroke}
\pgfsetlinewidth{0.000000\du}
\pgfsetdash{}{0pt}
\pgfsetdash{}{0pt}
\pgfsetbuttcap
\pgfsetmiterjoin
\pgfsetlinewidth{0.000000\du}
\pgfsetbuttcap
\pgfsetmiterjoin
\pgfsetdash{}{0pt}
\definecolor{dialinecolor}{rgb}{0.247059, 0.317647, 0.709804}
\pgfsetfillcolor{dialinecolor}
\pgfpathellipse{\pgfpoint{9.405000\du}{11.275000\du}}{\pgfpoint{1.300000\du}{0\du}}{\pgfpoint{0\du}{1.300000\du}}
\pgfusepath{fill}
\definecolor{dialinecolor}{rgb}{0.000000, 0.000000, 0.000000}
\pgfsetstrokecolor{dialinecolor}
\pgfpathellipse{\pgfpoint{9.405000\du}{11.275000\du}}{\pgfpoint{1.300000\du}{0\du}}{\pgfpoint{0\du}{1.300000\du}}
\pgfusepath{stroke}
\pgfsetbuttcap
\pgfsetmiterjoin
\pgfsetdash{}{0pt}
\definecolor{dialinecolor}{rgb}{0.000000, 0.000000, 0.000000}
\pgfsetstrokecolor{dialinecolor}
\pgfpathellipse{\pgfpoint{9.405000\du}{11.275000\du}}{\pgfpoint{1.300000\du}{0\du}}{\pgfpoint{0\du}{1.300000\du}}
\pgfusepath{stroke}
\pgfsetlinewidth{0.100000\du}
\pgfsetdash{}{0pt}
{\pgfsetcornersarced{\pgfpoint{0.500000\du}{0.500000\du}}\definecolor{dialinecolor}{rgb}{1.000000, 1.000000, 1.000000}
\pgfsetfillcolor{dialinecolor}
\fill (3.100000\du,5.950000\du)--(3.100000\du,7.750000\du)--(7.440000\du,7.750000\du)--(7.440000\du,5.950000\du)--cycle;
}{\pgfsetcornersarced{\pgfpoint{0.500000\du}{0.500000\du}}\definecolor{dialinecolor}{rgb}{0.000000, 0.000000, 0.000000}
\pgfsetstrokecolor{dialinecolor}
\draw (3.100000\du,5.950000\du)--(3.100000\du,7.750000\du)--(7.440000\du,7.750000\du)--(7.440000\du,5.950000\du)--cycle;
}% setfont left to latex
\definecolor{dialinecolor}{rgb}{0.000000, 0.000000, 0.000000}
\pgfsetstrokecolor{dialinecolor}
\node at (5.270000\du,7.045000\du){Requête R};
\pgfsetlinewidth{0.100000\du}
\pgfsetdash{}{0pt}
\pgfsetdash{}{0pt}
\pgfsetbuttcap
{
\definecolor{dialinecolor}{rgb}{1.000000, 0.000000, 0.000000}
\pgfsetfillcolor{dialinecolor}
% was here!!!
\pgfsetarrowsend{latex}
\definecolor{dialinecolor}{rgb}{1.000000, 0.000000, 0.000000}
\pgfsetstrokecolor{dialinecolor}
\draw (6.150303\du,7.792041\du)--(8.517378\du,10.325127\du);
}
\pgfsetlinewidth{0.100000\du}
\pgfsetdash{}{0pt}
\pgfsetdash{}{0pt}
\pgfsetbuttcap
{
\definecolor{dialinecolor}{rgb}{1.000000, 0.000000, 0.000000}
\pgfsetfillcolor{dialinecolor}
% was here!!!
\pgfsetarrowsend{stealth}
\definecolor{dialinecolor}{rgb}{1.000000, 0.000000, 0.000000}
\pgfsetstrokecolor{dialinecolor}
\draw (10.704760\du,11.246703\du)--(18.440240\du,11.078297\du);
}
\pgfsetlinewidth{0.100000\du}
\pgfsetdash{}{0pt}
\pgfsetdash{}{0pt}
\pgfsetbuttcap
{
\definecolor{dialinecolor}{rgb}{1.000000, 0.000000, 0.000000}
\pgfsetfillcolor{dialinecolor}
% was here!!!
\pgfsetarrowsend{stealth}
\definecolor{dialinecolor}{rgb}{1.000000, 0.000000, 0.000000}
\pgfsetstrokecolor{dialinecolor}
\draw (10.631034\du,11.705096\du)--(20.008966\du,14.994904\du);
}
\pgfsetlinewidth{0.100000\du}
\pgfsetdash{}{0pt}
\pgfsetdash{}{0pt}
\pgfsetbuttcap
{
\definecolor{dialinecolor}{rgb}{1.000000, 0.000000, 0.000000}
\pgfsetfillcolor{dialinecolor}
% was here!!!
\pgfsetarrowsend{stealth}
\definecolor{dialinecolor}{rgb}{1.000000, 0.000000, 0.000000}
\pgfsetstrokecolor{dialinecolor}
\draw (9.405000\du,12.575000\du)--(11.756433\du,20.302345\du);
}
\pgfsetlinewidth{0.100000\du}
\pgfsetdash{}{0pt}
\pgfsetdash{}{0pt}
\pgfsetbuttcap
\pgfsetmiterjoin
\pgfsetlinewidth{0.100000\du}
\pgfsetbuttcap
\pgfsetmiterjoin
\pgfsetdash{}{0pt}
\definecolor{dialinecolor}{rgb}{1.000000, 1.000000, 1.000000}
\pgfsetfillcolor{dialinecolor}
\fill (8.850000\du,21.400000\du)--(8.850000\du,23.800000\du)--(10.500000\du,23.800000\du)--(10.500000\du,21.400000\du)--cycle;
\pgfsetbuttcap
\pgfsetmiterjoin
\pgfsetdash{}{0pt}
\definecolor{dialinecolor}{rgb}{0.000000, 0.000000, 0.000000}
\pgfsetstrokecolor{dialinecolor}
\draw (8.850000\du,21.400000\du)--(8.850000\du,23.800000\du)--(10.500000\du,23.800000\du)--(10.500000\du,21.400000\du);
\pgfsetlinewidth{0.100000\du}
\pgfsetdash{}{0pt}
\pgfsetdash{}{0pt}
\pgfsetbuttcap
{
\definecolor{dialinecolor}{rgb}{0.000000, 0.000000, 0.000000}
\pgfsetfillcolor{dialinecolor}
% was here!!!
\definecolor{dialinecolor}{rgb}{0.000000, 0.000000, 0.000000}
\pgfsetstrokecolor{dialinecolor}
\draw (8.850000\du,22.600000\du)--(10.500000\du,22.600000\du);
}
% setfont left to latex
\definecolor{dialinecolor}{rgb}{0.000000, 0.000000, 0.000000}
\pgfsetstrokecolor{dialinecolor}
\node at (9.650000\du,23.195000\du){R};
\pgfsetlinewidth{0.100000\du}
\pgfsetdash{}{0pt}
\pgfsetdash{}{0pt}
\pgfsetbuttcap
\pgfsetmiterjoin
\pgfsetlinewidth{0.100000\du}
\pgfsetbuttcap
\pgfsetmiterjoin
\pgfsetdash{}{0pt}
\definecolor{dialinecolor}{rgb}{1.000000, 1.000000, 1.000000}
\pgfsetfillcolor{dialinecolor}
\fill (20.095000\du,6.925000\du)--(20.095000\du,9.325000\du)--(21.745000\du,9.325000\du)--(21.745000\du,6.925000\du)--cycle;
\pgfsetbuttcap
\pgfsetmiterjoin
\pgfsetdash{}{0pt}
\definecolor{dialinecolor}{rgb}{0.000000, 0.000000, 0.000000}
\pgfsetstrokecolor{dialinecolor}
\draw (20.095000\du,6.925000\du)--(20.095000\du,9.325000\du)--(21.745000\du,9.325000\du)--(21.745000\du,6.925000\du);
\pgfsetlinewidth{0.100000\du}
\pgfsetdash{}{0pt}
\pgfsetdash{}{0pt}
\pgfsetbuttcap
{
\definecolor{dialinecolor}{rgb}{0.000000, 0.000000, 0.000000}
\pgfsetfillcolor{dialinecolor}
% was here!!!
\definecolor{dialinecolor}{rgb}{0.000000, 0.000000, 0.000000}
\pgfsetstrokecolor{dialinecolor}
\draw (20.095000\du,8.125000\du)--(21.745000\du,8.125000\du);
}
% setfont left to latex
\definecolor{dialinecolor}{rgb}{0.000000, 0.000000, 0.000000}
\pgfsetstrokecolor{dialinecolor}
\node at (20.895000\du,8.720000\du){R};
\pgfsetlinewidth{0.100000\du}
\pgfsetdash{}{0pt}
\pgfsetdash{}{0pt}
\pgfsetbuttcap
\pgfsetmiterjoin
\pgfsetlinewidth{0.100000\du}
\pgfsetbuttcap
\pgfsetmiterjoin
\pgfsetdash{}{0pt}
\definecolor{dialinecolor}{rgb}{1.000000, 1.000000, 1.000000}
\pgfsetfillcolor{dialinecolor}
\fill (22.840000\du,12.350000\du)--(22.840000\du,14.750000\du)--(24.490000\du,14.750000\du)--(24.490000\du,12.350000\du)--cycle;
\pgfsetbuttcap
\pgfsetmiterjoin
\pgfsetdash{}{0pt}
\definecolor{dialinecolor}{rgb}{0.000000, 0.000000, 0.000000}
\pgfsetstrokecolor{dialinecolor}
\draw (22.840000\du,12.350000\du)--(22.840000\du,14.750000\du)--(24.490000\du,14.750000\du)--(24.490000\du,12.350000\du);
\pgfsetlinewidth{0.100000\du}
\pgfsetdash{}{0pt}
\pgfsetdash{}{0pt}
\pgfsetbuttcap
{
\definecolor{dialinecolor}{rgb}{0.000000, 0.000000, 0.000000}
\pgfsetfillcolor{dialinecolor}
% was here!!!
\definecolor{dialinecolor}{rgb}{0.000000, 0.000000, 0.000000}
\pgfsetstrokecolor{dialinecolor}
\draw (22.840000\du,13.550000\du)--(24.490000\du,13.550000\du);
}
% setfont left to latex
\definecolor{dialinecolor}{rgb}{0.000000, 0.000000, 0.000000}
\pgfsetstrokecolor{dialinecolor}
\node at (23.640000\du,14.195000\du){R};
\pgfsetlinewidth{0.100000\du}
\pgfsetdash{}{0pt}
\pgfsetdash{}{0pt}
\pgfsetbuttcap
{
\definecolor{dialinecolor}{rgb}{0.298039, 0.686275, 0.313726}
\pgfsetfillcolor{dialinecolor}
% was here!!!
\pgfsetarrowsend{stealth}
\definecolor{dialinecolor}{rgb}{0.298039, 0.686275, 0.313726}
\pgfsetstrokecolor{dialinecolor}
\draw (12.270000\du,20.250000\du)--(10.016976\du,12.458150\du);
}
\pgfsetlinewidth{0.100000\du}
\pgfsetdash{}{0pt}
\pgfsetdash{}{0pt}
\pgfsetbuttcap
{
\definecolor{dialinecolor}{rgb}{0.298039, 0.686275, 0.313726}
\pgfsetfillcolor{dialinecolor}
% was here!!!
\pgfsetarrowsend{stealth}
\definecolor{dialinecolor}{rgb}{0.298039, 0.686275, 0.313726}
\pgfsetstrokecolor{dialinecolor}
\draw (9.267378\du,10.125127\du)--(6.900303\du,7.592041\du);
}
\pgfsetlinewidth{0.100000\du}
\pgfsetdash{}{0pt}
\pgfsetdash{}{0pt}
\pgfsetbuttcap
{
\definecolor{dialinecolor}{rgb}{1.000000, 0.341176, 0.133333}
\pgfsetfillcolor{dialinecolor}
% was here!!!
\pgfsetarrowsend{stealth}
\definecolor{dialinecolor}{rgb}{1.000000, 0.341176, 0.133333}
\pgfsetstrokecolor{dialinecolor}
\draw (13.343020\du,20.816899\du)--(20.161980\du,16.158101\du);
}
\pgfsetlinewidth{0.100000\du}
\pgfsetdash{}{0pt}
\pgfsetdash{}{0pt}
\pgfsetbuttcap
{
\definecolor{dialinecolor}{rgb}{1.000000, 0.341176, 0.133333}
\pgfsetfillcolor{dialinecolor}
% was here!!!
\pgfsetarrowsend{stealth}
\definecolor{dialinecolor}{rgb}{1.000000, 0.341176, 0.133333}
\pgfsetstrokecolor{dialinecolor}
\draw (13.023657\du,20.490643\du)--(18.986343\du,12.109357\du);
}
\pgfsetlinewidth{0.100000\du}
\pgfsetdash{}{0pt}
\pgfsetdash{}{0pt}
\pgfsetbuttcap
{
\definecolor{dialinecolor}{rgb}{1.000000, 0.341176, 0.133333}
\pgfsetfillcolor{dialinecolor}
% was here!!!
\definecolor{dialinecolor}{rgb}{1.000000, 0.341176, 0.133333}
\pgfsetstrokecolor{dialinecolor}
\draw (20.095000\du,8.125000\du)--(21.475000\du,9.200000\du);
}
\pgfsetlinewidth{0.100000\du}
\pgfsetdash{}{0pt}
\pgfsetdash{}{0pt}
\pgfsetbuttcap
{
\definecolor{dialinecolor}{rgb}{1.000000, 0.341176, 0.133333}
\pgfsetfillcolor{dialinecolor}
% was here!!!
\definecolor{dialinecolor}{rgb}{1.000000, 0.341176, 0.133333}
\pgfsetstrokecolor{dialinecolor}
\draw (21.520000\du,8.325000\du)--(20.225000\du,9.150000\du);
}
\pgfsetlinewidth{0.100000\du}
\pgfsetdash{}{0pt}
\pgfsetdash{}{0pt}
\pgfsetbuttcap
{
\definecolor{dialinecolor}{rgb}{1.000000, 0.341176, 0.133333}
\pgfsetfillcolor{dialinecolor}
% was here!!!
\definecolor{dialinecolor}{rgb}{1.000000, 0.341176, 0.133333}
\pgfsetstrokecolor{dialinecolor}
\draw (23.015165\du,13.595165\du)--(24.340000\du,14.600000\du);
}
\pgfsetlinewidth{0.100000\du}
\pgfsetdash{}{0pt}
\pgfsetdash{}{0pt}
\pgfsetbuttcap
{
\definecolor{dialinecolor}{rgb}{1.000000, 0.341176, 0.133333}
\pgfsetfillcolor{dialinecolor}
% was here!!!
\definecolor{dialinecolor}{rgb}{1.000000, 0.341176, 0.133333}
\pgfsetstrokecolor{dialinecolor}
\draw (24.265165\du,13.795165\du)--(23.020000\du,14.600000\du);
}
\end{tikzpicture}

	\caption{Cheminement d'une requête dans une base de données distribuées avec la prise en charge de l'affectation (un seul noeud traite la requête)\label{fig:request}}
\end{figure}

\begin{figure}[p]
	\centering
		% Graphic for TeX using PGF
% Title: C:\Users\Kéké\Pictures\Diagramme1.dia
% Creator: Dia v0.97.2
% CreationDate: Fri Jan 30 16:46:06 2015
% For: Kéké
% \usepackage{tikz}
% The following commands are not supported in PSTricks at present
% We define them conditionally, so when they are implemented,
% this pgf file will use them.
\ifx\du\undefined
  \newlength{\du}
\fi
\setlength{\du}{15\unitlength}
\begin{tikzpicture}
\pgftransformxscale{1.000000}
\pgftransformyscale{-1.000000}
\definecolor{dialinecolor}{rgb}{0.000000, 0.000000, 0.000000}
\pgfsetstrokecolor{dialinecolor}
\definecolor{dialinecolor}{rgb}{1.000000, 1.000000, 1.000000}
\pgfsetfillcolor{dialinecolor}
\definecolor{dialinecolor}{rgb}{1.000000, 1.000000, 1.000000}
\pgfsetfillcolor{dialinecolor}
\pgfpathellipse{\pgfpoint{9.291856\du}{4.819228\du}}{\pgfpoint{4.128876\du}{0\du}}{\pgfpoint{0\du}{3.973660\du}}
\pgfusepath{fill}
\pgfsetlinewidth{0.100000\du}
\pgfsetdash{}{0pt}
\pgfsetdash{}{0pt}
\pgfsetmiterjoin
\definecolor{dialinecolor}{rgb}{0.000000, 0.000000, 0.000000}
\pgfsetstrokecolor{dialinecolor}
\pgfpathellipse{\pgfpoint{9.291856\du}{4.819228\du}}{\pgfpoint{4.128876\du}{0\du}}{\pgfpoint{0\du}{3.973660\du}}
\pgfusepath{stroke}
% setfont left to latex
\definecolor{dialinecolor}{rgb}{0.000000, 0.000000, 0.000000}
\pgfsetstrokecolor{dialinecolor}
\node at (9.291856\du,4.614228\du){Choix des paramètres};
% setfont left to latex
\definecolor{dialinecolor}{rgb}{0.000000, 0.000000, 0.000000}
\pgfsetstrokecolor{dialinecolor}
\node at (9.291856\du,5.414228\du){ de la simulation};
\definecolor{dialinecolor}{rgb}{1.000000, 1.000000, 1.000000}
\pgfsetfillcolor{dialinecolor}
\pgfpathellipse{\pgfpoint{9.346636\du}{16.723318\du}}{\pgfpoint{4.003364\du}{0\du}}{\pgfpoint{0\du}{4.026682\du}}
\pgfusepath{fill}
\pgfsetlinewidth{0.100000\du}
\pgfsetdash{}{0pt}
\pgfsetdash{}{0pt}
\pgfsetmiterjoin
\definecolor{dialinecolor}{rgb}{0.000000, 0.000000, 0.000000}
\pgfsetstrokecolor{dialinecolor}
\pgfpathellipse{\pgfpoint{9.346636\du}{16.723318\du}}{\pgfpoint{4.003364\du}{0\du}}{\pgfpoint{0\du}{4.026682\du}}
\pgfusepath{stroke}
% setfont left to latex
\definecolor{dialinecolor}{rgb}{0.000000, 0.000000, 0.000000}
\pgfsetstrokecolor{dialinecolor}
\node at (9.346636\du,16.518318\du){Lancement de};
% setfont left to latex
\definecolor{dialinecolor}{rgb}{0.000000, 0.000000, 0.000000}
\pgfsetstrokecolor{dialinecolor}
\node at (9.346636\du,17.318318\du){la simulation};
\pgfsetlinewidth{0.100000\du}
\pgfsetdash{}{0pt}
\pgfsetdash{}{0pt}
\pgfsetbuttcap
{
\definecolor{dialinecolor}{rgb}{0.000000, 0.000000, 0.000000}
\pgfsetfillcolor{dialinecolor}
% was here!!!
\pgfsetarrowsend{latex}
\definecolor{dialinecolor}{rgb}{0.000000, 0.000000, 0.000000}
\pgfsetstrokecolor{dialinecolor}
\draw (9.291856\du,8.792888\du)--(9.346636\du,12.696636\du);
}
\definecolor{dialinecolor}{rgb}{1.000000, 1.000000, 1.000000}
\pgfsetfillcolor{dialinecolor}
\fill (2.450000\du,25.150000\du)--(2.450000\du,35.150000\du)--(16.150000\du,35.150000\du)--(16.150000\du,25.150000\du)--cycle;
\pgfsetlinewidth{0.100000\du}
\pgfsetdash{}{0pt}
\pgfsetdash{}{0pt}
\pgfsetmiterjoin
\definecolor{dialinecolor}{rgb}{0.000000, 0.000000, 0.000000}
\pgfsetstrokecolor{dialinecolor}
\draw (2.450000\du,25.150000\du)--(2.450000\du,35.150000\du)--(16.150000\du,35.150000\du)--(16.150000\du,25.150000\du)--cycle;
% setfont left to latex
\definecolor{dialinecolor}{rgb}{0.000000, 0.000000, 0.000000}
\pgfsetstrokecolor{dialinecolor}
\node at (9.300000\du,29.945000\du){ENVIRONNEMENT DE};
% setfont left to latex
\definecolor{dialinecolor}{rgb}{0.000000, 0.000000, 0.000000}
\pgfsetstrokecolor{dialinecolor}
\node at (9.300000\du,30.745000\du){SIMULATION};
\pgfsetlinewidth{0.100000\du}
\pgfsetdash{}{0pt}
\pgfsetdash{}{0pt}
\pgfsetbuttcap
{
\definecolor{dialinecolor}{rgb}{0.000000, 0.000000, 0.000000}
\pgfsetfillcolor{dialinecolor}
% was here!!!
\pgfsetarrowsend{latex}
\definecolor{dialinecolor}{rgb}{0.000000, 0.000000, 0.000000}
\pgfsetstrokecolor{dialinecolor}
\draw (9.346636\du,20.750000\du)--(9.300000\du,25.150000\du);
}
% setfont left to latex
\definecolor{dialinecolor}{rgb}{0.000000, 0.000000, 0.000000}
\pgfsetstrokecolor{dialinecolor}
\node[anchor=west] at (9.323318\du,23.172500\du){Paramètres};
\definecolor{dialinecolor}{rgb}{1.000000, 1.000000, 1.000000}
\pgfsetfillcolor{dialinecolor}
\pgfpathellipse{\pgfpoint{9.296636\du}{43.498318\du}}{\pgfpoint{4.203364\du}{0\du}}{\pgfpoint{0\du}{4.151682\du}}
\pgfusepath{fill}
\pgfsetlinewidth{0.100000\du}
\pgfsetdash{}{0pt}
\pgfsetdash{}{0pt}
\pgfsetmiterjoin
\definecolor{dialinecolor}{rgb}{0.000000, 0.000000, 0.000000}
\pgfsetstrokecolor{dialinecolor}
\pgfpathellipse{\pgfpoint{9.296636\du}{43.498318\du}}{\pgfpoint{4.203364\du}{0\du}}{\pgfpoint{0\du}{4.151682\du}}
\pgfusepath{stroke}
% setfont left to latex
\definecolor{dialinecolor}{rgb}{0.000000, 0.000000, 0.000000}
\pgfsetstrokecolor{dialinecolor}
\node at (9.296636\du,42.893318\du){Stockage de};
% setfont left to latex
\definecolor{dialinecolor}{rgb}{0.000000, 0.000000, 0.000000}
\pgfsetstrokecolor{dialinecolor}
\node at (9.296636\du,43.693318\du){l'état final de};
% setfont left to latex
\definecolor{dialinecolor}{rgb}{0.000000, 0.000000, 0.000000}
\pgfsetstrokecolor{dialinecolor}
\node at (9.296636\du,44.493318\du){la simulation};
\pgfsetlinewidth{0.100000\du}
\pgfsetdash{}{0pt}
\pgfsetdash{}{0pt}
\pgfsetbuttcap
{
\definecolor{dialinecolor}{rgb}{0.000000, 0.000000, 0.000000}
\pgfsetfillcolor{dialinecolor}
% was here!!!
\pgfsetarrowsend{latex}
\definecolor{dialinecolor}{rgb}{0.000000, 0.000000, 0.000000}
\pgfsetstrokecolor{dialinecolor}
\draw (9.300000\du,35.150000\du)--(9.296636\du,39.346636\du);
}
% setfont left to latex
\definecolor{dialinecolor}{rgb}{0.000000, 0.000000, 0.000000}
\pgfsetstrokecolor{dialinecolor}
\node[anchor=west] at (9.298318\du,37.070818\du){Etat final de};
% setfont left to latex
\definecolor{dialinecolor}{rgb}{0.000000, 0.000000, 0.000000}
\pgfsetstrokecolor{dialinecolor}
\node[anchor=west] at (9.298318\du,37.870818\du){la simulation};
\pgfsetlinewidth{0.100000\du}
\pgfsetdash{}{0pt}
\definecolor{dialinecolor}{rgb}{1.000000, 1.000000, 1.000000}
\pgfsetfillcolor{dialinecolor}
\pgfpathellipse{\pgfpoint{26.375000\du}{28.115217\du}}{\pgfpoint{0.332609\du}{0\du}}{\pgfpoint{0\du}{0.332609\du}}
\pgfusepath{fill}
\definecolor{dialinecolor}{rgb}{0.000000, 0.000000, 0.000000}
\pgfsetstrokecolor{dialinecolor}
\pgfpathellipse{\pgfpoint{26.375000\du}{28.115217\du}}{\pgfpoint{0.332609\du}{0\du}}{\pgfpoint{0\du}{0.332609\du}}
\pgfusepath{stroke}
\definecolor{dialinecolor}{rgb}{0.000000, 0.000000, 0.000000}
\pgfsetstrokecolor{dialinecolor}
\draw (25.044565\du,28.780435\du)--(27.705435\du,28.780435\du);
\definecolor{dialinecolor}{rgb}{0.000000, 0.000000, 0.000000}
\pgfsetstrokecolor{dialinecolor}
\draw (26.375000\du,28.447826\du)--(26.375000\du,30.110870\du);
\definecolor{dialinecolor}{rgb}{0.000000, 0.000000, 0.000000}
\pgfsetstrokecolor{dialinecolor}
\draw (26.375000\du,30.110870\du)--(25.044565\du,31.584783\du);
\definecolor{dialinecolor}{rgb}{0.000000, 0.000000, 0.000000}
\pgfsetstrokecolor{dialinecolor}
\draw (26.375000\du,30.110870\du)--(27.705435\du,31.584783\du);
% setfont left to latex
\definecolor{dialinecolor}{rgb}{0.000000, 0.000000, 0.000000}
\pgfsetstrokecolor{dialinecolor}
\node at (26.375000\du,32.845000\du){Client};
\pgfsetlinewidth{0.100000\du}
\pgfsetdash{{\pgflinewidth}{0.200000\du}}{0cm}
\pgfsetdash{{\pgflinewidth}{0.200000\du}}{0cm}
\pgfsetbuttcap
{
\definecolor{dialinecolor}{rgb}{0.000000, 0.000000, 0.000000}
\pgfsetfillcolor{dialinecolor}
% was here!!!
\pgfsetarrowsend{latex}
\definecolor{dialinecolor}{rgb}{0.000000, 0.000000, 0.000000}
\pgfsetstrokecolor{dialinecolor}
\draw (24.550000\du,30.100000\du)--(16.150000\du,30.150000\du);
}
% setfont left to latex
\definecolor{dialinecolor}{rgb}{0.000000, 0.000000, 0.000000}
\pgfsetstrokecolor{dialinecolor}
\node at (20.350000\du,30.720000\du){Observe pendant};
% setfont left to latex
\definecolor{dialinecolor}{rgb}{0.000000, 0.000000, 0.000000}
\pgfsetstrokecolor{dialinecolor}
\node at (20.350000\du,31.520000\du){l'exécution};
% setfont left to latex
\definecolor{dialinecolor}{rgb}{0.000000, 0.000000, 0.000000}
\pgfsetstrokecolor{dialinecolor}
\node[anchor=west] at (27.250000\du,20.600000\du){};
% setfont left to latex
\definecolor{dialinecolor}{rgb}{0.000000, 0.000000, 0.000000}
\pgfsetstrokecolor{dialinecolor}
\node[anchor=west] at (9.296636\du,43.498318\du){};
% setfont left to latex
\definecolor{dialinecolor}{rgb}{0.000000, 0.000000, 0.000000}
\pgfsetstrokecolor{dialinecolor}
\node[anchor=west] at (9.319246\du,10.967262\du){Paramètres};
\end{tikzpicture}

	\caption{Scénario d'utilisation du projet par le client\label{fig:user_scenario}}
\end{figure}

\bibliography{identifications_besoins}

\end{document}
