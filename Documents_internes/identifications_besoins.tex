\documentclass[12pt]{article}
\usepackage[utf8]{inputenc}
\usepackage[french]{babel}
\usepackage[tikz]{bclogo}
\usepackage{geometry}
\geometry{hmargin=2.5cm,vmargin=1.5cm}

\setlength{\parskip}{1ex plus 2ex minus 1ex}

\title{
 \begin{minipage}\linewidth
        \centering
        Simulation d'algorithmes d'équilibrage de charge dans un environnement distribué 
        \vskip3pt
        \large Identifications des besoins
    \end{minipage}
 }
    
\author{Kevin Barreau \and Guillaume Marques \and Corentin Salingue}

\begin{document}

\maketitle

\abstract
Ce document dégage une première identification des besoins.
Il s'agit d'un document support pour l'élaboration du cahier des charges

\newpage

\section{Définition du projet}

\subsection{Définitions}

\paragraph{}
L'expansion, au cours des deux dernières décennies, des réseaux et notamment d'Internet a engendré une importante création de données, massives par leur nombre et leur taille.
Stocker cette information sur un seul point de stockage (ordinateur par exemple) n'est bien sûr plus envisageable, que ce soit pour des raisons  techniques ou pour des raisons de sécurité (pannes potentielles par exemple).
Pour cela des systèmes de stockages dit distribués sont utilisés en pratique afin des les répartir sur dfférentes unités de stockages.

\paragraph{Définition} Un environnement distribué est constitué de plusieurs machines (ordinateurs), appelées \textit{noeuds}, sur lesquelles sont stockées des données.

\paragraph{Définition} Une donnée est une suite binaire de $0$ et de $1$ dont le contenu n'est pas important pour l'application.

\paragraph{} Le client souhaite répartir toutes ces données de manières équitable entre les noeuds. De plus, il souhaite que ces données soient accessible afin de pouvoir les requêter et récuperer de l'information.

\paragraph{Définition} Une requête est un message envoyé à une machine (ou plusieurs machines) afin de récuperer de l'information sur des données.
Nous noterons que la nature de l'information est inutile pour le bon fonctionnement de l'application.


\paragraph{} Pour répartir toutes ces données, notre client a développer de nouveaux algorithmes d'équilibrage de charges et de réplication qu'il souhaite tester dans un environnement distribué.

\subsection{Finalité}

\paragraph{} Nous devons développer une solution logicielle permettant de tester ces nouveaux algorithmes d'équilibrage de charge et de réplication.


\section{Besoins fonctionnels}

L'environnement de simulation voulu est un système distribué constitué de $n$ noeuds de stockage dans lequel on souhaite stocker $m$ objets.

\subsection{Environnement distribué}

Un environnement distribué est constitué de plusieurs machines, appelées \textit{noeuds}, sur lesquelles sont stockées les données et sont distribuées les tâches à effectuer, appelées \textit{requêtes}.
	
\vspace{0.5cm}
%Le client souhaite l'utilisation de l'environnement distribué Cassandra car il pense que tous les aspects qu'il souhaite tester sont facilement implémentables ici.
% Faux - Cassandra car c'est la BDD qu'ils connaissent le mieux.
Utilisation de Cassandra. Pourquoi ? \newline
\begin{itemize}
 \item Le client est plus à l'aise avec cette solution (il a quelques connaissances sur le logiciel)
 \item Pourquoi pas une autre BDD ? (développer)
\end{itemize}



\subsection{Gestion des noeuds}


\begin{itemize}
 \item Creer un noeud
 \item Supprimer un noeud (inutile d'après Kévin)s
 \item Sauvegarder un ensemble de noeuds avec leurs objets
 \item Importer un ensemble de noeuds avec leurs objets afin de simuler sur un environemment précedemment crée.
\end{itemize}

\vspace{0.5cm}

\textbf{Question } Quelle format pour le stockage d'un ensemble de noeuds ? \newline
\textbf{Question } Combien de noeuds maximum ? \newline
\textbf{Question } Position des noeuds sur le ring ? \newline


\subsection{Gestion des objets}

\textbf{Question } Sur quel type d'objet allons-nous travailler ? \newline

\begin{itemize}
 \item Creer un objet
 \item Supprimer un objet
 \item Gérer la popularité d'un objet
 \begin{itemize}
  \item Implémenter l'algorithme d'approximation Space-Saving Algorithm
  \item Ajouter un vecteur de taille $n$ à chaque noeud dans le lequel on stockera les objets les plus populaires
  \item Protocole de communication pour le calcul des objets les plus populaires \textbf{(Modifier gossip? Nouveau ?)}
 \end{itemize}
\end{itemize}

\vspace{0.5cm}

\textbf{Question } Gestion de la popularité : quelle période (fixe ou fenêtre glissante) ? \newline
\textbf{Question } Position de l'objet sur le noeud ? \newline
\textbf{Note } Bien définir l'algorithme. \newline

\subsection{Gestion des requêtes}

\begin{itemize}
 \item Creer une requête
 \item Envoyer une requête sur le réseau
 \item Sauvegarder un jeu de requête
 \item Importer un jeu de requête
\end{itemize}

\vspace{0.5cm}

\textbf{Question } Quelle format pour le stockage d'un jeu de requêtes ? \newline

\subsection{Visualisation des données}

\begin{itemize}
 \item Temps de réponse moyen sur les requêtes passées.
 \item Charge d'un noeud
 \item Popularité des objets
\end{itemize}

\vspace{0.5cm}

\textbf{Note } Bien définir ces items. \newline
\end{document}
