\documentclass[12 pt]{article}
\usepackage[utf8]{inputenc}
\usepackage[french]{babel}
\usepackage{url}

\bibliographystyle{alpha}

\title{Éléments bibliographiques}
\begin{document}

\maketitle
\section{Système distribué}

\begin{itemize}
\item La référence \cite{RefOzsu2011} apporte les principes sur les base de données distribuées, allant de l'architecture aux requêtes, en passant par l'optimisation, les transactions ou encore la concurrence.
\end{itemize}


\section{Algorithmes d'équilibrage de charges}


\begin{itemize}
\item La référence \cite{RefArtbalancedalloc} traite des algorithmes d'équilibrage en général.

\item La référence \cite{RefRand} traite des algorithmes d'allocation aléatoire.

\item La référence \cite{RefLoadBalancingPeertoPeer} traite des algorithmes d'équilibrage des charges dans réseau peer-to-peer.
\end{itemize}

\section{Cassandra}

\begin{itemize}
\item La référence \cite{RefHewitt2010} renseigne sur la compréhension du fonctionnement de Cassandra, ainsi que les structures particulières de cette base de données.

\item La référence \cite{RefApacheCassandra} permet d'accéder à l'ensemble de la documentation technique et aux sources de la base de données Cassandra.

\item La référence \cite{RefArtFb} 

\item La référence \cite{RefDataStaxfctCassandra}

\end{itemize}

\section{Visualisation}

\begin{itemize}
\item La référence \cite{RefDataStax}

\item La référence \cite{RefTulip}
\end{itemize} 

\bibliography{bibliography}


\end{document}



